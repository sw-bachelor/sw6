\section{Presentation from Kindergarden Birken 27/2}
Birken is a kindergarden for children with autism and ADHD.
The representeres from the kindergarden is Kristine and Susanne.
The are children from the age of 0 to 7, but usually at least 2 years old.
\\
During the presentation they told a lot about what autism is and how it influences the children with the diagnose of autism.
Children with autism lack communicative skills and lack stucture. 
One example they gave, was that a child could sit in front of a lunch box for 15 minutes without start eating, because they didn't know how to begin.
Another example was if they had to go to the bathroom, they needed to know the order they have to complete the task.
Birken solved this by having a column of small pictograms, which showed the order they had to follow. 
Hereafter they needed to know the order to wash their hands.
There would be a column of pictograms by the toilet and one by the sink.
\\\\
After talking about what autism and ADHD is, they followed to talk about the visions for the weekplanner that they had.
They emphasize strongly that they wanted a stable program and that it should be usable offline.
Children with autism often like predictability and repetitiveness. 
If is not possible to log into the weekplanner because it is either not stable or not connected to the internet, the children would quickly get frustrated.
Therefore a stable program and a usable program when being offline was the most important thing for them this project.

\subsection{Useful features for weekplanner}
Kristine and Susanne also talked about other useful features, that would enhance the usability in weekplanner for them.

\begin{itemize}
    \item Be able to take pictures and insert into the weekplanner
    \item Be able to duplicate a week to other weeks
    \item Be able to make templates for every week
    \item Be able to change background to other colors to help new children like the kindergarden 
\end{itemize}
\todo{Skal det være user stories i stedet for?}

Being able to take pictures and inserting it into the weekplanner would enhance the flexibility for the program.
If they suddenly have to change plans the pedagogue should be able to insert a picture if there is not an existing pictogram for the new plans.
\\
Being able to duplicate a week to other weeks and make templates is to save time for the pedagogue. 
It currently already takes them a long time to plan the new weeks for the children. 
If they need to manually insert a pictogram every time for almost identical weeks on different iPads it would be very time consuming for them.
\\
The need to change the background is to accommodate new children arriving in the kindergarden. 
They change color to something that the child likes. 
Hereby the child sees something that it reconizes and likes, and this can give them a better introduction to the kindergarden.
