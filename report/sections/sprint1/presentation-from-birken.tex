\section{Presentation from Kindergarten Birken 27/2}\label{presentation-from-birken}
Birken is a kindergarten for children with autism and ADHD.
Kristine and Susanne are the representatives from the kindergarten and are the ones who conducted the presentation.
The children range from the age of 0 to 7 where most of them are at least 2 years old.
\\\\
During the presentation Kristine and Susanne discussed autism and how it influenced the children that had the diagnosis.
The main relevant point for the GIRAF project was that children with autism lack communicative skills and structure. 
They emphasized strongly that they wanted a stable application and that it should be usable offline.
A point of consideration was that children with autism often liked predictability and repetitiveness. 

\subsubsection{Useful features for the weekplanner}
Kristine and Susanne also talked about useful features that would enhance the usability of the weekplanner:

\begin{itemize}
    \item Be able to take pictures and insert them into the weekplanner
    \item Be able to duplicate a week plan
    \item Be able to create template schedules for the week plans
\end{itemize}
\noindent
Being able to take pictures and inserting them into the weekplanner would enhance the flexibility of the program.
\\\\
Being able to duplicate week plans or make week plan templates could be utilized to save time for the pedagogue and avoid unnecessary workloads.
Instead of creating a week plan from scratch every time, they could create a new week plan to use as a template from which to make smaller changes. 
