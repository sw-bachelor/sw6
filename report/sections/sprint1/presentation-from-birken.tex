\section{Presentation from Kindergarten Birken 27/2}
Birken is a kindergarten for children with autism and ADHD.
Kristine and Susanne are the representatives from the kindergarten and are the ones who conducted the presentation.
The children range from the age of 0 to 7, but are usually at least 2 years old.
\\
During the presentation Kristine and Susanne talked a lot about what autism is and how it influences the children with the diagnosis.
The main relevant point for the GIRAF project is that children with autism lack communicative skills and stucture. 
One example they gave was that a child could sit in front of a lunch box for 15 minutes without starting to eat, because they might not know how to begin.
Another example given was, if they had to go to the bathroom, they needed to know the order in wihch they have to complete the task.
Birken solved this by having columns of small pictograms hanging in the bathroom, which showed the order they had to follow. 
For example, there would be a column of pictograms byu the toilet.
After this they needed to know the order to wash their hands.
In the same was as the column at the toilet, there would be a column of pictograms by the sink detailing the order of events.
\\\\
After talking about what autism and ADHD is, they continued by talking about the visions they have for the weekplanner.
They emphasize strongly that they want a stable program and that it should be usable offline.
A point of consideration is that children with autism often like predictability and repetitiveness. 
If it is not possible to log in to the weekplanner because it is either not stable or not connected to the internet, the children would quickly get frustrated.
Therefore a stable and a usable program while being offline was the most important thing for them.

\subsection{Useful features for weekplanner}
Kristine and Susanne also talked about other useful features that would enhance the usability in the weekplanner:

\begin{itemize}
    \item Be able to take pictures and insert them into the weekplanner
    \item Be able to duplicate a week to other weeks
    \item Be able to make template schedules for the weeks
    \item Be able to change the background colour to other colours
\end{itemize}
\todo{Skal det være user stories i stedet for?}

Being able to take pictures and inserting them into the weekplanner would enhance the flexibility of the program.
If they suddenly have to change plans the pedagogue should be able to insert a picture if there is not an existing pictogram for the new plans.
\\
Being able to duplicate a week to other weeks and make templates is to save time for the pedagogue and avoid unnecessary workloads. 
It currently already takes them a long time to plan the new weeks for the children. 
If they need to manually insert a pictogram every time for almost identical weeks on different iPads it would be very time consuming for them.
In the same vein, being able to make template schedules is another way to save time. 
Simply importing a template to base a week off of is a valuable feature.
\\
The need to change the background is to accommodate new children arriving in the kindergarten. 
They want to change color to the childs favorite color. 
Because of this, the child sees something that it reconizes and likes, and this can give them a better introduction and first impression to the kindergarten.
It also helps as it can conform to the child's preferences.
