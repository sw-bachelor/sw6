\section{Sprint review}
At the end of the sprint all groups participated in a meeting called the sprint review, as mentioned in \autoref{subsec:SoS-sprint-review}.
Every group had a representative present, while the PO- and process group had all their members at the meeting.
\\
As per the protocol, each group presented the work they had completed during the sprint. The user stories and their status at the end of the sprint can be seen on \autoref{table:user-stories-sprint-1-review}.

\begin{table}[H]
    \begin{tabular}{|p{2.8cm}|p{7cm}|p{2cm}|p{1.5cm}|}
    \hline
    Issue ID        & User story name                                                                                                                                                          & Group assigned  & Status     \\ \hline
    Weekplanner\#4  & As a guardian, I would like to be able to mark activity(s)                                                                                                               & Group 13        & Invalid    \\ \hline
    Weekplanner\#6  & As a guardian I would like the user selection screen to look better so it makes it easier for me to find the correct user                                                & Group 9         & Invalid    \\ \hline
    Weekplanner\#8  & As a citizen I would like the ability to choose how my day is represented (horizontally or vertically) so that it fits my personal preference                            & Group 2         & Invalid    \\ \hline
    Weekplanner\#11 & As a guardian, I would like a way to add pictograms directly from google so that I can quickly improvise if the system does not have the activity I want                 & Group 13        & Invalid    \\ \hline
    Weekplanner\#14 & As a citizen I would like the icons to be consistent throughout the system so that I instinctively know their meaning                                                    & Group 11        & Invalid    \\ \hline
    Weekplanner\#15 & As a citizen I would like to be able to choose how many days I see at a time on my weekplanner, so that it fits my personal preference                                   & Group 11        & Invalid    \\ \hline
    Weekplanner\#16 & As a guardian I would like to be able to see results as I'm typing the name of a pictogram so that I can see if there are any results instead of just seeing my keyboard & Group 8         & Invalid    \\ \hline
    Weekplanner\#17 & As a guardian I would like to confirm with a password that the system is changing to guardian mode so that a citizen cannot gain access to it                            & Group 12        & Resolved   \\ \hline
    Weekplanner\#19 & As a user I would like the icons to be updated so that they are modern and easy to understand                                                                            & Group 10        & Invalid    \\ \hline
    \end{tabular}
    \caption{User stories for all development groups in sprint 1.}\label{table:user-stories-sprint-1-review}
\end{table}

\begin{itemize}
    \item Group 2 were not completely done, they still needed tests for their code.
    \item Group 8 ran into a dependency problem as they needed a feature that another group was working on. Therefore they did not finish their user story.
    \item Group 9 had completed their user story. Maybe they could do some small design tweaks, but the functionality was complete.
    \item Group 10, our group, did not finish the user stories we were given. Since we are the PO-group we also had other responsibilities which were completed, leaving us without the necessary time to complete the programming. We had customer interviews and meetings, we created a new design guide, we corrected outdated prototypes and created new ones as well as edited user stories on top of the programming.
    \item Group 11 was the group responsible for fixing errors people encountered with the new \texttt{Flutter} API. Therefore, they had done a lot of bug fixing which they were still working on. Their user story was not completely done as a result of this.
    \item Group 12 had implemented their user story. There seemed to be a problem with the backend implementation, however. As their user story was only frontend related it was considered done, and a new issue related to the backend was created instead.
    \item Group 13 were also almost done with their user story.
\end{itemize}
\noindent
This sprint review was a bit unusual, as many user stories were left not fully implemented. This was a result of the decision to change the frontend framework which meant that the groups had to start from scratch on new issues and user stories in the middle of the sprint.
Therefore it was expected that most groups would not have completed their tasks.

\subsubsection{PO goal review}
All our goals for sprint 1 have been resolved. 
However, the choice of changing framework to flutter required us to do a lot more work for some of the goals.
It required us to create more user stories because we needed to reimplement features that were previously implemented in Xamarin.
New prototypes were however not needed because of this, as we had already committed to creating new prototypes for everything.

\noindent
This is the status for the goals in sprint 1:
\begin{table}[H]
    \centering
    \begin{tabular}{|l|l|}
    \hline
    Goals:                                 & Status   \\ \hline
    Create user stories                    & Resolved \\ \hline
    Interview customers                    & Resolved \\ \hline
    Create prototypes for all user stories & Resolved \\ \hline
    Create a new design guide                & Resolved \\ \hline
    Prepare sprint 2                       & Resolved \\ \hline
    \end{tabular}
    \caption{Status of all the goals the PO group had in sprint 1}
    \label{PO-goal-review-sprint-1}
\end{table}

