\section{Sprint review}\label{sprint-1-review}
The sprint review concerns itself with how the sprint ended in terms of user stories and implemented features.
At the end of the sprint all groups participated in a meeting called the sprint review, as mentioned in \autoref{subsec:SoS-sprint-review}.
Every group had a representative present, while the PO- and process group had all their members at the meeting.
\\
As per the protocol, each group presented the work they had completed during the sprint. The user stories and their status at the end of the sprint can be seen on \autoref{table:user-stories-sprint-1-review}.

\begin{table}[H]
    \begin{tabular}{|p{2.8cm}|p{7cm}|p{1.6cm}|p{2cm}|}
    \hline
    Issue ID        & User story name                                                                                                                                                          & Group assigned  & Status     \\ \hline
    Weekplanner\#23 & As a guardian I would like to be able to access parts of the system through the top bar so that I can easily go to the screen I want                                     & Group 2         & Incomplete  \\ \hline
    Weekplanner\#43 & As a guardian I would like to be able to view a given citizens week plan so that I can get an overview of what is happening this week for them                           & Group 10        & Incomplete     \\ \hline
    Weekplanner\#44 & As a citizen I would like to be able to view my week plan so that I know what is going to happen                                                                         & Group 10        & Incomplete       \\ \hline
    Weekplanner\#46 & As a guardian I would like to be able to search for pictograms when adding a new activity so that I can find a pictogram that suits the activity                         & Group 11        & Incomplete      \\ \hline
    Weekplanner\#47 & As a guardian I would like to be able to choose a citizen so that I can choose who I’m editing the week plan for                                                         & Group 9         & Mostly complete    \\ \hline
    Weekplanner\#48 & As a guardian I would like to be able to log into the system using a username and password, so that I can see my associated citizens                                     & Group 9         & Mostly complete    \\ \hline
    Weekplanner\#49 & As a guardian I would like to be able to log out so that the system is not left logged in after I’m done using it.                                                       & Group 12        & Complete     \\ \hline
    Weekplanner\#50 & As a guardian I would like to be able to choose which week plan to show the citizen so that I can have multiple plans available to each citizen                          & Group 13        & Mostly complete     \\ \hline
    Weekplanner\#51 & As a guardian I would like to be able to create a new week plan so that it is suited to the given citizens week                                                          & Group 8         & Incomplete    \\ \hline
    Wiki\#3         & Migrate pages under project management                                                                                                                                   & Group 2         & Complete    \\ \hline
    Wiki\#6         & Migrate pages under REST API Development from "Backend architecture" to "Future work"                                                                                    & Group 12        & Complete    \\ \hline
    \end{tabular}
   \caption{User stories for all development groups in sprint 1 after Flutter was implemented.}\label{table:user-stories-sprint-1-review}
\end{table}

\begin{itemize}
    \item Group 2 were not completely done, they still needed tests for their code. They did finish their wiki assignment, however.
    \item Group 8 ran into a dependency problem as they needed a feature that another group was working on. Therefore they did not finish their user story.
    \item Group 9 had completed their user stories. Maybe they could do some small design tweaks, but the functionality was complete.
    \item Group 10, our group, did not finish the user stories we were given. Since we are the PO-group we also had other responsibilities leaving us without the necessary time to complete the programming tasks.
    \item Group 11 was the group responsible for fixing errors people encountered with the new \texttt{Flutter} API. Therefore, they had done a lot of bug fixing which they were still working on. Their user story was not completely done as a result of this.
    \item Group 12 had implemented their user story. There seemed to be a problem with the backend implementation, however. As their user story was only frontend related it was considered done, and a new issue related to the backend was created instead. Their wiki assignment was also completed.
    \item Group 13 were also almost done with their user story.
\end{itemize}
\noindent
This sprint review was a bit unusual, as many user stories were left not fully implemented. This was a result of the decision to change the frontend framework which meant that the groups had to start from scratch on new issues and user stories in the middle of the sprint.
Therefore it was expected that most groups would not have completed their tasks due to inexperience with Flutter and simply not having enough time to finish their user stories.

\subsubsection{PO goal review}
All our goals for sprint 1 have been resolved. 
However, the choice of changing framework to flutter required us to do a lot more work for some of the goals.
It required us to create more user stories because we needed to reimplement features that were previously implemented in Xamarin.
New prototypes were however not needed because of this change of framework, as we had already committed to creating new prototypes when Xamarin was used.
We were assigned two user stories after changing framework, which we did not complete in this sprint.

\noindent
This is the status for the goals in sprint 1:
\begin{table}[H]
    \centering
    \begin{tabular}{|l|l|}
    \hline
    Goals:                                 & Status   \\ \hline
    Create user stories                    & Resolved \\ \hline
    Interview customers                    & Resolved \\ \hline
    Create prototypes for highly prioritized user stories & Resolved \\ \hline
    Create a new design guide                & Resolved \\ \hline
    Prepare sprint 2                       & Resolved \\ \hline
    Weekplanner \#43 and Weekplanner\#44                       & Not resolved \\ \hline
    \end{tabular}
    \caption{Status of all the goals the PO group had in sprint 1.}
    \label{PO-goal-review-sprint-1}
\end{table}

\section{Sprint retrospective}\label{sprint-1-retrospective}
To close out the sprint we held a retrospective meeting for all of the developers on GIRAF.
The retrospective concerns itself with the processes used during the sprint.
The retrospective explores whether there were any major issues, or how the process could be improved to improve productivity.
This led to changes in the process for the whole team.
We also reflected internally on the process within our group, and examined how the shortcomings from sprint 1 could be avoided in the upcoming sprint.
These retrospectives are detailed below.

\subsubsection{The retrospective for the GIRAF project}\label{retro1giraf}
Immediately after the sprint review had concluded all groups started the sprint retrospective.
This is a meeting that every member of the GIRAF project should attend, such that everyone can give their opinion on how they thought the processes of the sprint had been conducted.
After everyone had arrived, each group got five minutes to talk about the sprint process and whether there was anything that they felt should be changed, or if there was something they felt worked really well.
\\\\
After this discussion, people were split up into different mixed groups with one representative from each GIRAF group.
These new groups started to discuss the processes and talked about the shortcomings of the sprint.
This resulted in each of these new groups coming up with some new ideas of what could be done differently.
When everyone was done discussing, all these new ideas were shared on a website where each person could vote anonymously on the three ideas they liked best.
\\\\
The ideas presented have been compiled into a list as shown below:
\begin{itemize}
    \item Not everyone needs to be at the sprint planning.
    \item Remember to come ask the PO group if in doubt about anything related to a user story.
    \item Less meetings on days without lectures.
    \item Get some of the developers to do technical explanations of the user stories so that there is a customer and a developer perspective on each user story.
    \item More systematic standup meetings. It should be enforced that every group has a representative present.
    \item Stop having sprint planning meetings.
    \item Use F-Klubben's facilities as a more relaxed space for discussion. 
    \item Standup meetings should not be placed in the middle of \textit{Advanced Algorithms} exercise sessions.
    \item Acquire new tablets since some of the current ones are in bad condition or are too old to run the newest versions of their operating system.
    \item Consider doing online standup meetings or allow people to attend the meetings virtually.
    \item Less meetings on days with lectures.
    \item Standup meetings should not be right after or before a lecture.
    \item More hackathons.
    \item Consider when the PO group should review functionality. 
    \item The PO group should go around and ask how far the different groups are with their tasks.
    \item More activities across the different groups.
\end{itemize}
Afterwards the sprint retrospective was over.
The process group then took the ideas and started considering how they could incorporate them into the next sprint.
\\
\noindent
The following ideas were the ones that received the most votes and were considered the most relevant by the process group.
\begin{itemize}
    \item Get some of the developers to do technical explanations of the user stories so that there is a customer and a developer perspective on each user story.
    \item Remember to come ask the PO group if in doubt about anything related to the user story.
    \item Acquire new tablets since some of the current ones are in bad condition or are too old to run the newest versions of their operating system.
\end{itemize}
The idea that each user story should have a technical explanation was the idea with the most votes. 
Following the meeting it was quickly introduced as a new standard for new user stories and existing user stories got a technical explanation added.
Furthermore, other development groups were encouraged to come see the PO group more often in case of doubt.
The process group decided that the sprint planning meeting should be changed to the sprint introduction.
These processes are described in \autoref{subsec:sprint-planning} and \autoref{subsec:sprint-introduction} respectively.
The process group also tried to get iPads and newer tablets but was not successful in accomplishing this.

\subsubsection{A retrospective for our process}
Within our group we employed a modified scrum model.
While a GIRAF sprint usually lasts three weeks, our internal sprints last just one week.
This is done to ensure we have something new to present to the supervisor every week, so we can continually get feedback and improve our work.
We define a backlog of smaller issues and distribute them between the members based on expected workload.
This works well, and will not be changed.
In terms of our interaction with the project as a whole, some changes are made.
As described in \autoref{retro1giraf}, there were some proposed changes that relate directly to the PO group.
Specifically, this was the points that emphasized how developers in doubt should approach the PO group for clarification, and how the PO could benefit from sending representatives to the different groups intermittently. 
This would help to generate an overview for both the other development groups and the PO group, and would let both parties keep up to date with the status of the project.
Because of this, we decided to implement the approach of visiting the different groups to keep up to date.
