\section{Sprint review}\label{sprint-1-review}
The sprint review concerns itself with how the sprint ended in terms of user stories and implemented features.
At the end of the sprint all groups participated in a meeting called the sprint review.
Every group had a representative present, while the PO- and process groups had all their members at the meeting.
\\
As per the protocol, each group presented the work they had completed during the sprint. The user stories and their status at the end of the sprint can be seen on \autoref{table:user-stories-sprint-1-review}.

\begin{longtable}{|p{2.8cm}|p{7cm}|p{1.6cm}|p{2cm}|}
    \hline
    Issue ID        & User story name                                                                                                                                                          & Group assigned  & Status     \\ \hline
    Weekplanner\#23 & As a guardian I would like to be able to access parts of the system through the top bar so that I can easily go to the screen I want                                     & Group 2         & Incomplete  \\ \hline
    Weekplanner\#43 & As a guardian I would like to be able to view a given citizens week plan so that I can get an overview of what is happening this week for them                           & Group 10        & Incomplete     \\ \hline
    Weekplanner\#44 & As a citizen I would like to be able to view my week plan so that I know what is going to happen                                                                         & Group 10        & Incomplete       \\ \hline
    Weekplanner\#46 & As a guardian I would like to be able to search for pictograms when adding a new activity so that I can find a pictogram that suits the activity                         & Group 11        & Incomplete      \\ \hline
    Weekplanner\#47 & As a guardian I would like to be able to choose a citizen so that I can choose who I’m editing the week plan for                                                         & Group 9         & Mostly complete    \\ \hline
    Weekplanner\#48 & As a guardian I would like to be able to log into the system using a username and password, so that I can see my associated citizens                                     & Group 9         & Mostly complete    \\ \hline
    Weekplanner\#49 & As a guardian I would like to be able to log out so that the system is not left logged in after I’m done using it.                                                       & Group 12        & Complete     \\ \hline
    Weekplanner\#50 & As a guardian I would like to be able to choose which week plan to show the citizen so that I can have multiple plans available to each citizen                          & Group 13        & Mostly complete     \\ \hline
    Weekplanner\#51 & As a guardian I would like to be able to create a new week plan so that it is suited to the given citizens week                                                          & Group 8         & Incomplete    \\ \hline
    Wiki\#3         & Migrate pages under project management                                                                                                                                   & Group 2         & Complete    \\ \hline
    Wiki\#6         & Migrate pages under REST API Development from "Backend architecture" to "Future work"                                                                                    & Group 12        & Complete    \\ \hline
   \caption{User stories for all development groups in sprint 1 after Flutter was implemented.}\label{table:user-stories-sprint-1-review}
\end{longtable}

\begin{itemize}
    \item Group 2 were not completely done, they still needed tests for their code. They did finish their wiki assignment, however.
    \item Group 8 ran into a dependency problem as they needed a feature that another group was working on. Because of this they did not finish their user story.
    \item Group 9 had completed their user stories.
    \item Group 10, our group, did not finish the user stories we were given due to a lack of time.
    \item Group 11 was the group responsible for fixing errors people encountered with the new Flutter API. Therefore, they had done a lot of bug fixing which they were still working on. Their user story was not completely done as a result of this.
    \item Group 12 had implemented their user story. There seemed to be a problem with the backend implementation, however. As their user story was only frontend related it was considered done, and a new issue related to the backend was created instead. Their wiki assignment was also completed.
    \item Group 13 were also almost done with their user story.
\end{itemize}
\noindent
This sprint review was a bit unusual, as many user stories were left not fully implemented. This was a result of the decision to change the front end framework which meant that the groups had to start from scratch on new user stories in the middle of the sprint.

\subsubsection{PO goal review} 
The choice of changing framework to Flutter required us to do a lot more work for some of the goals.
It required us to create more user stories because we needed to reimplement features that were previously implemented in Xamarin.
New prototypes were not needed because of this change of framework, as we had already committed to creating new prototypes when Xamarin was used.
We were assigned two user stories after changing framework, which we did not complete in this sprint.

\noindent
This is the status for the goals in sprint 1:
\begin{table}[H]
    \centering
    \begin{tabular}{|l|l|}
    \hline
    Goals:                                 & Status   \\ \hline
    Create user stories                    & Resolved \\ \hline
    Interview customers                    & Resolved \\ \hline
    Create prototypes for highly prioritized user stories & Resolved \\ \hline
    Create a new design guide                & Resolved \\ \hline
    Prepare sprint 2                       & Resolved \\ \hline
    Weekplanner \#43 and Weekplanner\#44                       & Not resolved \\ \hline
    \end{tabular}
    \caption{Status of all the goals the PO group had in sprint 1.}
    \label{PO-goal-review-sprint-1}
\end{table}

\section{Sprint retrospective}\label{sprint-1-retrospective}\label{retro1giraf}
Immediately after the sprint review had concluded all groups started the sprint retrospective.
Each group had five minutes to talk about the sprint process and whether there was anything that they felt should be changed.
\\\\
After this discussion, people were split into different mixed groups with one representative from each GIRAF group.
These new groups discussed the processes and talked about the shortcomings of the sprint.
When everyone was done discussing, new ideas were shared on a website where each person could vote anonymously on the three ideas they liked best.
\\\\
The following ideas were the ones that received the most votes:
\begin{itemize}
    \item Get some of the developers to do technical explanations of the user stories so that there is a customer and a developer perspective on each user story.
    \item Remember to come ask the PO group if in doubt about anything related to the user story.
    \item Acquire new tablets since some of the current ones are in bad condition or are too old to run the newest versions of their operating system.
\end{itemize}
The idea that each user story should have a technical explanation was the idea with the most votes. 
Following the meeting it was quickly introduced as a new standard for new user stories and existing user stories got a technical explanation added.
Furthermore, other development groups were encouraged to come see the PO group more often in case of doubt.
The process group decided that the sprint planning meeting should be changed to the sprint introduction.

\subsubsection{A retrospective for our process}
In terms of our interaction with the project as a whole, some changes were made.
As described in \autoref{retro1giraf}, there were some proposed changes that related directly to our group.
Specifically, these were the points that emphasized how developers in doubt should approach us for clarification, and how we could benefit from sending representatives to the different groups intermittently. 
This helped to generate an overview for both the other development groups and us, and would let both parties keep up to date with the status of the project.
Because of this, we decided to implement the approach of visiting the different groups to keep up to date.
