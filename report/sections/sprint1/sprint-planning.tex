\section{Sprint planning for sprint 1}\label{sprint-1-planning}
The first sprint began on February 25th and lasted until March 18th.
To commence it, all participants of the project gathered for a meeting.
This meeting was organized to determine how development should proceed for the sprint, distributing user stories between groups to maximize value for the customer.
As the PO group, we had certain responsibilities in relation to this.
To facilitate a smooth start of the sprint, we defined the goals and vision for the sprint, and prepared a backlog of user stories.

\subsection{The information and user stories available}
The PO group of last year had compiled a short list of user stories that defined features that the customers had pointed out as being attractive.
Combining the user stories in \autoref{prepared-work-from-previous-years} with the most essential key points of the interview with Emil discussed in \autoref{interview-with-emil} we produced an initial product backlog of user stories.
Unfortunately we could not meet with any of the other customers prior to starting the sprint, so only Emil's point of view had an impact on the initial stories.
The different user stories were prioritized into one of the five categories:
\begin{itemize}
    \item Highest
    \item High
    \item Medium
    \item Low
    \item Lowest
\end{itemize}
\noindent
The user stories in the product backlog were all features that were deemed as relevant, however the ones Emil had expressed the most enthusiasm for, as well as those from the previous PO group were carried forward into the sprint backlog.
This sprint backlog was the foundation for the meeting - all groups should be assigned at least one user story to develop.
For their very first user story, the groups were allowed to choose three stories they would prefer while disregarding the priority, as per the definition of the scrum process manual developed by the process group.

\subsection{The goals and vision for the GIRAF project}
To motivate the developers, and to create a clear vision of the focus and goals of the sprint we prepared a short presentation.
This presentation was based on the sprint backlog we constructed for the sprint.
The overall productivity of the GIRAF project was expected to be fairly low in the first sprint, as each group would need to get acquainted with the legacy code base, new groups and the new working environment.
As such, we wanted a sprint that seemed manageable by the developers, while providing them the opportunity to dive into the code base.
\\\\
As such, we defined the goals of the sprint:
\begin{itemize}
    \item Migrate the wiki from Phabricator to GitHub
    \item Implement the timer functionality for the week planner
    \item Improve the user interface
\end{itemize}
Along with this, the user stories we had prepared were introduced.

\subsection{The goal for the PO group}
Our goal for sprint 1 is to continue the work that started prior to sprint 1 and finish it.
This sprint we want to focus on getting an understanding of the needs of the customers, and then to pass on the knowledge to the development groups using user stories and prototypes.
This is done to reduce the amount of errors and misunderstandings that often occur when handing out development tasks.
Preparing sprint 2 includes having user stories ready and prioritized for sprint 2.
Every user story needs a prototype and a priority, so that the highest prioritized gets picked first.

\noindent
The following is the list of goals we have for sprint 1:
\begin{table}[H]
    \centering
    \begin{tabular}{|l|l|}
    \hline
    Goals:                                \\ \hline
    Create user stories                   \\ \hline
    Interview customers                    \\ \hline
    Create prototypes for all user stories \\ \hline
    Create a new design guide                \\ \hline
    Prepare sprint 2                       \\ \hline
    \end{tabular}
    \caption{Goals for the PO group in sprint 1}
    \label{PO-goal-sprint-1}
\end{table}

\subsection{User story selection}
All groups gave a prioritized list of user stories they would like to work on, and these were used when distributing stories.
After every group received a story, they would need to decompose the story into smaller tasks, estimate the workload and determine how to implement their specific story.
To assist with this, each member of the PO group was assigned one of the other six groups that work on the project. 
The PO representative would discuss the story with the assigned group, and answer any questions they would have to the best of their abilities.
Upon completing the estimation of the story, the meeting was concluded and the groups split to start working.
The PO group did not select a user story to implement as they were expected to be busy with other tasks during this sprint, such as preparing future sprints, getting prototypes ready and communicating with the customers.
