\section{Sprint planning}\label{sprint-1-planning}
The first sprint began on February 25th and lasted until March 18th.
To commence it, all participants of the project gathered for a meeting.
We had certain responsibilities in relation to this.
To facilitate a smooth start of the sprint, we defined the goals and vision for the sprint, and prepared a backlog of user stories.

\subsection{The prepared user stories}
We prepared a backlog of user stories for the first sprint.
The different user stories were prioritized into one of the five categories:
\begin{itemize}
    \item Highest
    \item High
    \item Medium
    \item Low
    \item Lowest
\end{itemize}
\noindent
For their very first user story, the groups were allowed to choose three stories they would prefer while disregarding the priority, as per the definition of the process manual developed by the process group.
The user stories that where chosen for the sprint backlog for sprint 1 can be seen on \autoref{table:user-stories-sprint-1-updated}.

\begin{longtable}{|p{2.8cm}|p{8cm}|p{2cm}|}
    \hline
    Issue ID        & User story name                                                                                                                                                          & Group assigned       \\ \hline
    Weekplanner\#4  & As a guardian, I would like to be able to mark activity(s)                                                                                                               & Group 13             \\ \hline
    Weekplanner\#6  & As a guardian I would like the user selection screen to look better so it makes it easier for me to find the correct user                                                & Group 9              \\ \hline
    Weekplanner\#8  & As a citizen I would like the ability to choose how my day is represented (horizontally or vertically) so that it fits my personal preference                            & Group 2              \\ \hline
    Weekplanner\#11 & As a guardian, I would like a way to add pictograms directly from google so that I can quickly improvise if the system does not have the activity I want                 & Group 13             \\ \hline
    Weekplanner\#14 & As a citizen I would like the icons to be consistent throughout the system so that I instinctively know their meaning                                                    & Group 11             \\ \hline
    Weekplanner\#15 & As a citizen I would like to be able to choose how many days I see at a time on my weekplanner, so that it fits my personal preference                                   & Group 11             \\ \hline
    Weekplanner\#16 & As a guardian I would like to be able to see results as I'm typing the name of a pictogram so that I can see if there are any results instead of just seeing my keyboard & Group 8              \\ \hline
    Weekplanner\#17 & As a guardian I would like to confirm with a password that the system is changing to guardian mode so that a citizen cannot gain access to it                            & Group 12             \\ \hline
    Wiki\#3         & Migrate pages under project management                                                                                                                                   & Group 2              \\ \hline
    Wiki\#6         & Migrate pages under REST API Development from "Backend architecture" to "Future work"                                                                                    & Group 12             \\ \hline
    \caption{User stories for all development groups in sprint 1.}\label{table:user-stories-sprint-1-updated}
\end{longtable}


\subsection{The goals and vision for the GIRAF project}
To motivate the developers, and to create a clear vision of the focus and goals of the sprint we prepared a short presentation.
This presentation was based on the sprint backlog we constructed for the sprint.
The overall productivity of the GIRAF project was expected to be fairly low in the first sprint, as each group would need to get acquainted with the legacy code base, new groups and the new working environment.
As such, we wanted a sprint that seemed manageable by the developers, while providing them the opportunity to dive into the code base.
\\\\
We defined the following goals for the sprint:
\begin{itemize}
    \item Migrate the wiki from Phabricator to GitHub
    \item Implement the timer functionality for the weekplanner
    \item Improve the user interface
\end{itemize}
Along with this, the user stories we had prepared were introduced.

\subsection{User story selection}
All groups gave a prioritized list of user stories they would like to work on, and these were used when distributing stories.
After every group received a story, they would need to decompose the story into smaller tasks, estimate the workload and determine how to implement their specific story.
To assist with this, each member of our group was assigned one of the other six groups.
We discussed the story with the assigned group, and answered any questions they had.
Upon completing the estimation of the story, the meeting was concluded.
We did not select a user story to implement as we were expected to be busy with other tasks during this sprint, such as preparing future sprints, getting prototypes ready and communicating with the customers.

\subsection{Our internal goals}
Our goals for sprint 1 were mainly to continue the work that started prior to the sprint.
In this sprint we wanted to focus on getting an understanding of the needs of the customers, and then to pass on that knowledge to the development groups using user stories and prototypes.

\noindent
The following is the list of goals we had for sprint 1:
\begin{table}[H]
    \centering
    \begin{tabular}{|l|l|}
    \hline
    Goals:                                \\ \hline
    Create user stories                   \\ \hline
    Interview customers                    \\ \hline
    Create prototypes for highly prioritized user stories \\ \hline
    Create a new design guide                \\ \hline
    Prepare sprint 2                       \\ \hline
    \end{tabular}
    \caption{Our internal goals for sprint 1.}
    \label{PO-goal-sprint-1}
\end{table}
\noindent
Preparing sprint 2 included having user stories ready and prioritized for sprint 2.
