\section{Change of framework}
The developers of last year had made the choice of changing the development framework from using Android Studio to Xamarin in order to support building the application for iOS.
However, with the introduction of full-stack teams this lead to a series of issues, as approximately 1/4th of the developers on the GIRAF project this year were using some variation of Linux as their operating system of choice, which Xamarin does not officially support.
Previously, this had not been a problem, as only the dedicated front-end development group would be working with Xamarin. 
Only the people in this group would need to be on Windows or MacOS in order to compile the front-end part of the application.
With full-stack groups, more or less every group would be affected by this problem, having group members who could not compile the front-end of the application.
This led to one group spending a lot of time investigating different ways to make the Xamarin project compile on Linux, but as time progressed it was deemed infeasible to implement a series of workarounds to make the project compile.
Some people got Xamarin to compile on Linux, but it seemed like that the process and methods to make Xamarin compile was not identical to other versions of Linux.
Instead, the front-end skill group held a meeting to discuss the pros and cons of making the Xamarin project work versus changing to another framework, which would natively support Linux.
During the meeting, a series of pros and cons for changing the framework were worked out:

\begin{itemize}
    \item [\textbf{Pros}]
    \item The most difficult part to reimplement is authorization
    \item It is difficult to maintain the Linux compatibility for Xamarin, as the problems varied between machines
    \item The Swagger API generates bad code, which can be fixed while changing the framework
    \item It would be possible to implement a cleaner UI while re-working the front-end
    \item It will be easier for future students to work on GIRAF if it works on Linux
    \item [\textbf{Cons}]
    \item Changing will require a new language that the developers may not have worked with before
    \item Some people have managed to get Xamarin to compile
    \item It is possible to compile Xamarin using Windows, which is offered for free by the university
    \item It will increase the time that is spent before a product can be released to the customers
\end{itemize}
\noindent
After discussing the pros and cons internally in the skill group, all the developers on the GIRAF project were invited to an open discussion. 
This discussion was held to hear the input of people who were not in the front-end skill group, as it felt like a decision that was too big to make without consulting everyone.
After the open meeting, it was clear that the majority of the developers were ready to make the change, and it was decided that \texttt{Flutter} would be used as a new framework. 
\texttt{Flutter} is an open-source mobile application development framework by Google, that uses the \texttt{Dart} language.
One of the major reasons for choosing Flutter over alternative frameworks such as \texttt{React Native} or \texttt{Ionic} is the availability of tutorials and guides, for both Flutter and Dart. 
Likewise, it was deemed that, due to Flutter being developed by Google, the chance that it would become outdated anytime soon was highly unlikely.
In order to ease the transition from Xamarin to Flutter, a Flutter team was assembled.
This team was assigned the task of spending the upcoming weekend with setting up the base of the Flutter project, and to help the other students get started by running a Flutter workshop in the following week.
