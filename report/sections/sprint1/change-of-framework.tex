\section{Change of framework}\label{change-of-framework}
The developers of last year had made the choice of changing the development framework from using Android Studio to Xamarin in order to support building the application for iOS.
However, with the introduction of full stack teams this lead to a series of issues, as approximately 1/4th of the developers on the GIRAF project this year were using some variation of Linux as their operating system of choice, which Xamarin does not officially support.
Previously, this had not been a problem, as only the dedicated front-end development group would be working with Xamarin. 
Only the people in this group would need to be on Windows or MacOS in order to compile the front-end part of the application.
With full stack groups, more or less every group would be affected by this problem, having group members who could not compile the front-end of the application.
This led to one group spending a lot of time investigating different ways to make the Xamarin project compile on Linux, but as time progressed it was deemed infeasible to implement a series of workarounds to make the project compile.
Some people got Xamarin to compile on Linux, but it seemed like that the process and methods to make Xamarin compile was not identical to other versions of Linux.
Instead, the front-end skill group held a meeting to discuss the pros and cons of making the Xamarin project work versus changing to another framework, which would natively support Linux.
During the meeting, a series of pros and cons for changing the framework were worked out:

\begin{itemize}
    \item [\textbf{Pros}]
    \item The most difficult part to reimplement is authorization
    \item It is difficult to maintain the Linux compatibility for Xamarin, as the problems varied between machines
    \item The Swagger API generates bad code, which can be fixed while changing the framework
    \item It would be possible to implement a cleaner UI while re-working the front-end
    \item It will be easier for future students to work on GIRAF if it works on Linux
    \item [\textbf{Cons}]
    \item Changing will require a new language that the developers may not have worked with before
    \item Some people have managed to get Xamarin to compile
    \item It is possible to compile Xamarin using Windows, which is offered for free by the university
    \item It will increase the time that is spent before a product can be released to the customers
\end{itemize}
\noindent
After discussing the pros and cons internally in the skill group, all the developers on the GIRAF project were invited to an open discussion. 
This discussion was held to hear the input of people who were not in the front-end skill group, as it felt like a decision that was too big to make without consulting everyone.
After the open meeting, it was clear that the majority of the developers were ready to make the change, and it was decided that \texttt{Flutter} would be used as a new framework. 
\texttt{Flutter} is an open-source mobile application development framework by Google, that uses the \texttt{Dart} language.
One of the major reasons for choosing Flutter over alternative frameworks such as \texttt{React Native} or \texttt{Ionic} is the availability of tutorials and guides, for both Flutter and Dart. 
Likewise, it was deemed that, due to Flutter being developed by Google, the chance that it would become outdated anytime soon was highly unlikely.
In order to ease the transition from Xamarin to Flutter, a Flutter team was assembled.
This team was assigned the task of spending the upcoming weekend with setting up the base of the Flutter project, and to help the other students get started by running a Flutter workshop in the following week.

\subsection{Flutter weekend}
To make the transition to Flutter from Xamarin smoother for everyone in the GIRAF project, a group of volunteers were assembled that spent the following weekend laying the ground work for switching to Flutter. 
The team spent the weekend implementing the core part of the application, so that the rest of the groups in the GIRAF project had something to work from. 
The team consisted of eight people, where four of them were from group sw611f19, two from the PO group, one from group sw612f19 and one from the process group.
Half of them spent the weekend implementing the API in Flutter and re-creating all the models on the frontend part of the application. \\\\
Previously, when the application was implemented in Xamarin, they had made use of Swagger to automatically generate the API for the applications frontend.
This did not work optimally, due to faulty code generation from Swagger, and the team decided to implement it manually instead, to ensure that it worked as intended.
The other half of the weekend the team spent the weekend focusing on the graphical user interface of the application. \\\\
A base design of the login screen, the week plan screen and the settings screen were created to get some insight into how applications are designed in Flutter, and to serve as a foundation such that the rest of the groups had some examples to look at when they were to implement user stories in the coming week.
A big part of the weekend was also spent looking at best practices and standards regarding Flutter development, and at which architectures could be used when implementing the application.
The team settled on using the \texttt{BLoC} structure and some ground work was done to facilitate this, and an example \texttt{BLoC} was implemented as an example of what a \texttt{BLoC} should look like.
The PO group created new user stories that focused on implementing the whole application from scratch and getting the application to the same state as it was before the change of framework.\\\\
The following Monday, one of the group members of group sw611f19 assisted the product owners in writing more technical descriptions of the user stories that were to be implemented in the following week, to make it easier for the groups who did not participate in the weekend to get started with the development relating to their user stories.
A workshop was held on the following day for all the groups in the GIRAF project, where a brief introduction of Flutter was given, and some live coding was done to give an example of the workflow in Flutter.
At the workshop, all groups were then assigned user stories that they could start to work on.
