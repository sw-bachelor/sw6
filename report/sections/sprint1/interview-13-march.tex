\section{Interview with customer representatives 13/3}\label{interview13/3}
During pre-sprint 1 and the beginning of sprint 1 a lot of changes and design updates were considered for the interface of GIRAF.
This resulted in new prototypes, icons and a revision of the previous design guide used for the project.
To evaluate the changes made, we invited customer representatives to come for an interview and a presentation of the changes.
This interview was scheduled to take place on March 13th.
Two representatives from different partners were able to make this meeting - Susanne from Birken and Mette from Egebakken.

\subsection{The interview structure}
The primary goal of the interview was to confirm that the changes made to the design were acceptable. 
The secondary goal was to clarify certain uncertainties we encountered. 
To reach these goals, we structured the interview to consist of a presentation and some prepared questions targeting the areas of uncertainty.
The presentation was based on the most relevant new prototypes, and we encouraged discussions pertaining to the given prototype alongside the presentation.
Specifically, the prototypes had changed and needed feedback on were related to the following functionalities of GIRAF:
\begin{itemize}
    \item The timer functionality
    \item Choosing how to represent time
    \item Pictogram search
    \item Different ways of marking activities as done
    \item Copying a week plan to a different citizen
    \item The redone login screen with citizen selection
    \item Greyscale functionality
    \item Showing different amounts of days
    \item Horizontal view of week plan
    \item Marking multiple activities at the same time
\end{itemize}

\subsection{The key points of the presentation}
The discussions that arose based on the presentation led to valuable information. The most essential will be discussed below.

\subsubsection{Copying week plans between citizens and the login screen with citizen selection}
These two prototypes illustrated the fundamental functionality of the week planner. The guardian logs in, chooses a citizen from a list and is redirected to the week plan for the citizen.
The representatives were happy with the design as shown on the prototypes, however they had some concerns with the selection of citizens.
As it stands, the prototype makes use of two columns and however many rows are needed to show all citizen.
Each citizen is illustrated with a name and a picture.
Susanne and Mette were worried there might be too many layers to this functionality.
If they would have to go through multiple layers to insert pictures to get it to display as shown on the prototype, they would not be thrilled.
They want it to be foolproof and fast to do.
Another concern brought up is the need to be able to divide the citizens in a way that makes sense for the guardians.
Generally, the guardians are only concerned with a select number of citizens.
However, citizens cannot just be linked to guardians, as guardians might have to step in as substitute teachers for a different class, or other events could occur that neceesitate that one particular guardian has to make changes to a citizen they normally do not concern themselves with.
Proposed solutions to this as discussed with the representatives is to either add a search functionality for selecting citizens, or add a new layer to the prototype that works like a folder system, each folder specifying a class.
Opening the class folder would then lead to the current implementation of choosing a citizen.
Adding onto this functionality, Susanne and Mette questioned how we had thought to implement login functionality.
Because of the concerns raised above that not every guardian needs to necessarily act upon every citizen, the merits of personalised logins were brought to question.
They were uncertain whether these personalized logins would work, or if a team login for all guardians at one location would be necesarry. 
However, as long as guardians are connected to a location, and citizens distributed into class folders, private logins are a good solution.
This would let the guardian personalize their password, and allow them to step in in case of having to act as a substitute.
In terms of creating guardian an citizen profiles they were interested in both a web-solution for guardian profiles, or just an application functionality to do this - thy would prefer the simplest method.
For citizens they would like to see a simple way of doing this through the application that makes taking a picture easy.
\\\\
In regards to copying activities, we discussed whether every citizen has a unique week plan or if one week plan could be assigned to multiple citizens.
Susanne and Mette remarked that usually a foundation can be used. 
Specific activities will alays play a role in some capacity, such as \textit{wash and and eat} or \textit{playground}.
Mette also specified that they work on a template that changes based on weeks. 
On even weeks they would ride horses, while on odd weeks they do not.
As such, they would like to be able to operate with templates to reduce the necesarry workload. 
As an extension of this, some citizens will have the same activities, and copying activities to multiple citizens is a desirable functionality.

\todo{Overvej at indsætte billede af login.}

\subsubsection{Greyscale functionality, horizontal view and how a week plan is updated}
Next the prototypes of greyscale functionality and the horizontal view of the week planner were shown.
Greyscale had some requested changes in the form of colors.
They would like to see a darker grey used, and the division of days should be clearer with a more visible lines between days.
The horizontal view of the week planner was mostly fine. The way the names of the days was shown was not good enough, and this either needs to be redone or removed.
The main point is that the pictogram should always be facing the right side up, which the design does.
In terms of how changes are made in the week planner, it was unclear whether they would make these through their own tablet or the citizen's.
Upon being posed this question, Susanne and Mette confirmed that, generally, the way it will work is that they make the changes on the citizen's tablet.
Another concern is that the program needs a way to display the fact that changes have been made, to avoid confusion. 

\subsubsection{Showing different amounts of days}
The functionality of showing different amounts of days posed some uncertainties. 
GIRAF should be able to show seven days, five days, three days or just one day at a time.
But if the program were to show five days ahead, and the current day were friday, how would it handle the overlap with the next week?
This question was discussed for a while, eventually leading to the conclusion that if this were the case, they would want the program to show the next days, even if they might overlap with next week.
The problem that these days might be empty, as a week might not have been planned for the coming week, is ultimately a problem they would have to deal with on their end.

\subsection{Marking activities as completed, manipulating multiple activities and searching for pictograms}
In terms of marking activities as completed, they were happy with the options we presented - checkmarks, removal and greyscale. 
We posed the question of how to handle removal of completed activities.
If one were to be removed upon completion, an empty space would appear on the plan - should the system account for this and automatically move all the other activities up to remove the empty space?
According to Susanne and Mette, this could cause confusion, but ultimately it would juts depend on how the system was taught to the citizens.
\\\\
The ability to manipulate mulitple activities at the same time would be nice to have. 
The prototypes we had made prepared for this functionality were acceptable, however this gave functionality gave rise to an important discussion.
The institutions want the citizens to be as independent as possible - this means they are the ones that mark activitiesas completed.
Based on our impressions of the work of previous years of the GIRAF project and the prototypes that were handed down to us, we had thought the guardians would be performing this task.
On top of this, adding the ability for citizens to marks activities that were previously marked as completed as not completed, this would not pose a problem.
The representatives did not expect that the citizens would abuse this functionality, however they would like GIRAF to restrict this to only being able to change the status of only the previously completed activity.   
The prototype for pictrogram search was good, but they were also interested in either being able to take their own pictures to use, or to search Google for new ones, like Emil mentioned in \ref{interview-with-emil}.

\subsubsection{The timer functionality and cancelling activities}
The timer functionality is one of the fundamental functionalities we want to add to the GIRAF project.
The requirements of the functionaity were discussed alongside the prototypes.
Generally, they will add the timer as they add the activites to the plan rather than when the activity starts. 
The only exception to this is when the citizen completely shuts down, then they add it manually, or even do it physically.
The different representations we had designed were proposed and evaluated.
Generally they wanted to use an hourglass or a circle that gradually changed color.
When adding a timer, they would rather just input the time the activity takes in minutes rather than the time period it takes place in.
\todo{måske indsæt billede af repræsentationerne.}
Cancelling activities is a guardian only functionality, but the citizens need to be able to see that an activity has been cancelled. 

\subsection{The key points of the clarifying questions}
In terms of citizens having their own login or having guardians login, entering citizen mode and showing the week plan, Susanne and Mette would like to see a login functionaloty specific to the citizens.
They did not think citizens being able to look at other citizen's plan would be a large issue.
When asked if the citizens were likely to use GIRAF at home, the representatives answered that this was not known.
They could receive help to plan activities for the weekend, but they will generally do different things at home compared to the school. 
In terms of the information needed for a new citizen, only names and a photo would be necesarry. 

\subsection{Summary}
The most essential piece of information gained was the fact they we ad misunderstood the way activities are marked as done. 
We thought the guardians would perform this action, but it is actually the citizens. 
This means the prototypes needed changing.
When selecting a citizen, an additional layer to divide them into classes would be preferable for the representatives.
Generally, the designs proposed on the prototypes were acceptable, and the icons used were satisfactory.