\section{Sprint retrospective}
To close out the sprint we held a retrospective meeting for all of the developers on GIRAF.
This led to changes in the process for the whole team.
We also reflected internally on the process within our group, and examined how the shortcomings from sprint 1 could be avoided in the upcoming sprint.
These retrospectives are detailed below.

\subsection{The retrospective for the GIRAF project}\label{retro1giraf}
Immediately after the sprint review had concluded all groups started the sprint retrospective.
This is a meeting that every member of the GIRAF project should attend, such that everyone can give their opinion on how they thought the processes of the sprint had been conducted.
After everyone had arrived, each group got five minutes to talk about the sprint process and whether there was anything that they felt should be changed, or if there was something they felt worked really well.
\\\\
After this discussion, people were split up into different mixed groups with one representative from each GIRAF group.
These new groups started to discus the processes and talked about the shortcomings of the sprint.
This resulted in each of these new groups coming up with some new ideas of what could be done differently.
When everyone was done discussing, all these new ideas were put up on a website where each person could vote anonymously on the three ideas they liked best.
\\\\
The ideas presented have been compiled into a list as shown below:
\begin{itemize}
    \item Not everyone needs to be at the sprint planning.
    \item Remember to come ask the PO group if in doubt about anything related to the user story.
    \item Less meetings on days without lectures.
    \item Get some of the developers to do technical explanations of the user stories so that there is a customer and a developer perspective on each user story.
    \item More systematic standup meetings. It should be enforced that every group has a representative present.
    \item Stop having sprint planning meetings.
    \item Use F-Klubben's facilities as a more relaxed space for discussion. 
    \item Standup meetings should not be placed in the middle of \textit{Advanced Algorithms} exercise sessions.
    \item Acquire new tablets since some of the current ones are in bad condition or are too old to run the newest versions of their operating system.
    \item Consider doing online standup meetings or allow people to attend the meetings virtually.
    \item Less meetings on days with lectures.
    \item Standup meetings should not be right after or before a lecture.
    \item More hackathons.
    \item Consider when the PO group should review functionality. 
    \item The PO group should go around and ask how far the different groups are with their tasks.
    \item More activities across the different groups.
\end{itemize}
Afterwards the sprint retrospective was over.
The process group then took the ideas and started considering how they could incorporate them into the next sprint.
\\
\noindent
The following ideas were the things that were considered the most by the process group.
\begin{itemize}
    \item Get some of the developers to do technical explanations of the user stories so that there is a customer and a developer perspective on each user story.
    \item Remember to come ask the PO group if in doubt about anything related to the user story.
    \item Acquire new tablets since some of the current ones are in bad condition or are too old to run the newest versions of their operating system.
\end{itemize}
The idea that each user story should have an technical explanation was the idea with the most votes. 
Following the meeting it was quickly introduced as a new standard for new user stories and existing user stories got a technical explanation added.
Furthermore, development groups were encouraged to come see the PO group more in case of doubt which seemed to have worked.
The process group decided that the sprint planning meeting should be changed to the sprint introduction.
These processes are described in \autoref{subsec:SoS-sprint-planning} and \autoref{subsec:SoS-sprint-introduction} respectively.
The process group also tried to get iPads and newer tablets but was not successful in accomplishing this.

\subsection{A retrospective for our process}
Within our group we employed a regular scrum model.
While a GIRAF sprint usually lasts three weeks, our internal sprints last just one week.
This is done to ensure we have something new to present to the supervisor every week, so we can continually get feedback and improve our work.
We define a backlog of smaller issues and distribute them between the members.
This works well, and will not be changed.
In terms of our interaction with the project as a whole, some changes are made.
As described in \autoref{retro1giraf}, there were some proposed changes that relate directly to the PO group.
Specifically, this was the points that emphasized how developers in doubt should approach the PO group for clarification, and how the PO could benefit from sending representatives to the different groups intermittently. 
This would help to generate an overview for both the development groups and the PO group, and would let both parties keep up to date with the status of the project.
Because of this, we decided to implement the approach of visiting the different groups to keep up to date.