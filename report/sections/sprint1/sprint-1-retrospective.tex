\section{Sprint retrospective}
Immediately after the sprint review had concluded we started the sprint retrospective.
This is a meeting that every member of the GIRAF project should attend, such that everyone can give their opinion on how they thought the processes of the sprint had been conducted.
After everyone had arrived, each group got five minutes to talk about the sprint process and whether there was anything that they felt should be changed, or if there was something they felt worked really well.
\\\\
After this discussion, people were split up into different mixed groups with one representative from each GIRAF group.
These new groups started to discus the processes and talked about the shortcomings of the sprint.
This resulted in each of these new groups coming up with some new ideas of what could be done differently.
When everyone was done discussing, all these new ideas were put up on a website where each person could vote anonymously on the three ideas they liked best.
\\\\
\begin{itemize}
    \item Idea1
\end{itemize}
Afterwards the sprint retrospective was over.
The process group then took the ideas and started considering how they could incorporate these into the next sprint.
\\\\
The idea that each user story should have an technical explanation was the most voted for idea. 
Following the meeting it was quickly introduced as a new standard for new user stories.
