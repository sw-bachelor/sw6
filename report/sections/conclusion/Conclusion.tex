\chapter{Conclusion}
In this chapter we will be evaluating how well the process worked for us, the state of the application and whether we achieved the goals set in our product vision.

\section{Evaluating the process}
As mentioned in \todo{referer til proces afsnit}, we used a modified version of scrum where we used some of the concepts from scrum. Overall the impression was that the process worked out quite well for us since we did not make any drastical changes during the semester. We also finished most of the assigned user stories for each sprint and the productivity was genrally good throughout the semester. Only minor changes were made from sprint to sprint based on the feedback from each retrospective, and we ended up with a very efficient process in the final sprint. 

\section{Final state of GIRAF}
As mentioned in \autoref{change-of-framework} we changed framework to Flutter and rebuild the application.
Therefore we spent a lot of time reimplementing the same functionalities as they had in 2018. In the final sprint we implemented some new features such as:
\begin{itemize}
    \item Manipulating multiple activities at the same time.
    \item Adding pictograms from gallery to the database.
    \item Deleting weekplans
    \item Adding timers to the activities
    \item Faster pictogram loading
\end{itemize}

\section{Evaluating the product vision}
We would have liked to add even more new features to the application, but as it was expressed numerous times by the customers, stability was important very important to them if they were ever to use the application. This was also one of the goals we set in \autoref{sec:introduction-product-vision}. Achieving a stable and usable weekplanner was the most important goal for us, and from the usability test in \autoref{usability-test-14-05} we learned that the customers was very satifies with the endproduct. We can therefore conclude that we achieved this goal. 
\\\\
The next goal from the product vision was to achieve a more intuitive design. The customers were impressed with how easily they understood how to use the different functionalities available in the application, so this goal was also achieved.\todo{Referer måske til tider her} 
\\\\
The next goal from our product vision was to to have the application available on iOS since many citizens are using iPads. This goal is achieved since we switched to Flutter which can compile to iOS
\\\\
The next goal was to have the application be fully available offline. This goal quickly became unrealistic after we switched the framework to Flutter. We simply did not have time to focus on this goal and thus it was not completed.
\\\\
The final goal was to make it easy and accessible for the next year's GIRAF students to get started. For the final sprint we focused a lot on documenting every important part of the project and the \texttt{wiki} page on GitHub is now filled with information about the project. 
