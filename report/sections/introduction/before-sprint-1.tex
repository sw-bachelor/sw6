\section{Before sprint 1}
Before the first semester wide sprint could start, our group had some work to do as preparation hereof.
First of all, during the readthrough of the reports from previous years, it was discovered that the PO group of 2018 had left us a suggestion for content in the first sprint:

\subsection*{Suggested sprint 1}
\subsubsection*{Relevant user stories}

\begin{itemize}
    \item T1005: As a guardian, I would like to be able to mark activity(s). 
    \item T1242: Setting - Change the way an activity is marked as completed. 
\end{itemize}

\subsubsection{Requirements}
This release focuses on one small guardian quality-of-life feature, as well as a setting for the citizens.

\paragraph{Mark activities}
The guardian must be able to enter a mark state from a button on the master-detail page. The mark state enables the guardian to mark one or multiple activities (much like how it works with email inboxes, where you can mark the emails you wish to interact with.) Then after they've entered this mark state, the button to enter marked mode must be replaced by a return button that leaves mark mode and removes all marks. Also, the ability to delete all marked activities should be implemented.

\paragraph{Completed marker}
The completed marker is the ability to change the way an activity is marked as completed. The following options must be implemented for this release:

\begin{itemize}
    \item Checkmark: A checkmark is placed on top of the activity. (This is already implemented, but must be an option once more options are added)
    \item Hide activity: The activity disappears from the citizens perspective but must still be visible to guardians so they can remove the completion marker.
    \item  Move to the right: Move the activity further to the right in the day-column. This should only be available if the citizen also only prefers to have the weekplanner shown in portrait mode, and thus prefers to have a single day shown.
\end{itemize}
This must be implemented on the settingspage.

\subsection*{Suggested sprint 2}
\subsubsection*{Relevant user stories}

\begin{itemize}
    \item T913: As a user, I would like to be able to time tasks using a timer.
    \item T922: As a user, I would like to be able change the visual representation of the timer
\end{itemize}

\subsubsection*{Requirements}
The release focuses on a timer for the activitypage, which is the page you get when tapping an activity. This timer is used by the citizens to keep track of time for those activities that need it, and thus the guardians must be able to place a timer on the activity page, which the citizens can then access.

\paragraph{The timer}
The guardian must be able to add a timer to an activity. The purpose of the timer is to time the activities that need to be timed, to remove the necessity of the guardians using a physical timer. If the guardian chooses to time an activity, the timer must be placed on the activitypage of the given activity. The guardian must also be able to set the specific length of the timer, with the visualization of a timer adapting to the length. Furthermore, the guardian must be able to add timers to future activities, to prevent them having to constantly enter the application to add a timer to an activity.

A citizen must be able to start the timer by entering the activity page for an activity marked as active and pressing a “Start” button. If the activity is not marked as active, the timer cannot be started.

\paragraph{Setting for visualization of timer}

Citizens visualize time in different ways. It is, therefore, necessary for the guardians to be able to choose the specific timer that a citizen wants to use. The citizens must, therefore, have a setting that specifies the visualization of the timer. As a start, the visualization setting must include the following different timers.

\begin{itemize}
    \item Digital clock
    \item Egg-clock
    \item Hourglass
\end{itemize}
The chosen visualization must then be used as the timer in the activity.



\subsection{Interview with Emil from Egebakken}


\subsection{User stories}
Based on the suggested sprints from the previous PO group and the interview with Emil, we defined a series of user stories with suggested features.


\subsection{Design guide}
% nye ikoner

\subsection{Producing Prototypes in Adobe XD}
% old prototypes in powerpoint
% new in XD

