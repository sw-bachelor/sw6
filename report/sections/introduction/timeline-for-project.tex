\section{Timeline for project}
For our group this semester project could be split into 6 parts, as seen on \autoref{fig:timeline-for-project}.
There were 4 sprints in total for development of the application. 
For the PO- and process groups there was a lot of work to prepare the first sprint.
This is what we considered \textit{Pre sprint 1}.
The process group planned the sprints such that the last sprint would end two weeks before the deadline for handing in the project.
Those last two weeks were spent on usability tests with the customers and finishing the report.
What was common for every sprint was that they followed the process guidelines previously described in \autoref{the-giraf-process}.
\todo{fix ref if scrum of scrum section changes}

\begin{figure}[H]
    \center{\includegraphics[width=\textwidth]
    {figures/timeline-for-entire-project.JPG}}
    \caption{\label{fig:timeline-for-project} Timeline for the project.}
\end{figure}

\noindent The light blue boxes show the highlights of what was focused on during each sprint by our group.
The different parts will be explained in the following sections.

\subsubsection{Pre sprint 1}
During\textit{Pre sprint 1} the preparation for the first sprint took place.
During this time, the groups of the GIRAF project did not fully understand the project, so it was mostly spent gathering information.
We contacted and interviewed customers to gain insight into what the customers wanted.
Based on these interviews, user stories and prototypes were created so that they were ready for sprint 1.
In \textit{pre sprint 1}, an interview with Emil from Egebakken was conducted. 
This interview is described in \autoref{interview-with-emil}.
The process group focused on establishing the process guidelines, while the development groups were researching the existing codebase.

\subsubsection{Sprint 1}
A presentation from the kindergarten Birken was given on February 27th.
This presentation is described in \autoref{presentation-from-birken}.
The focus of this sprint started out as fixing bugs and making the application more stable.
It turned out that there were a lot of problems with the frontend with Xamarin, and it was decided that the frontend should be migrated to Flutter during an extraordinary meeting involving most participants of the project. \\
The pros and cons of the decision are described in \autoref{change-of-framework}.
This also resulted in us needing to look into the user stories, as new user stories for previously implemented features needed to be implemented again and many of the old user stories had to gain a lower priority.
A meeting was planned with the customers on March 13th to get feedback on the prototypes. 
This interview resulted in some of the prototypes getting reworked to better fit the customers needs.

\subsubsection{Sprint 2}
As there was no release at the end of sprint 1 due to the decision to migrate to Flutter, several of the user stories assigned in sprint 1 were not completed and as a result of this, there was no reason to conduct a usability test in sprint 1.
Instead, sprint 2 was mostly spent developing user stories planned in sprint 1 and some additional user stories for sprint 2.
During this sprint we also implemented some user stories, which are described in \autoref{our-assigned-user-stories-sprint-2}.
At the end of this sprint the first release was completed.

\subsubsection{Sprint 3}
There were still some  user stories from sprint 1 and sprint 2 that were not completed due to some dependencies. 
At this point the project started to have a some of the most essential features implemented, but it still needed a lot of features before it was usable. 
In this sprint we focused mostly on implementing user stories.
However, new prototypes were also designed during this sprint. 
The sprint ended with a usability test which is described in \autoref{usability-test-sprint-3}.

\subsubsection{Sprint 4}
As sprint 4 was our final sprint, the sprint mostly focused on completing the user stories that were not completed in the previous sprints and implementing the most essential user stories to have a minimal viable product.
Our group and the process group did not implement any user stories during this sprint, because we focused on documentation and making the handover to next year students as thorough as possible.
In the end of the sprint the final usability test was conducted.
The purpose of this usability test was to get a better understanding of what the application was lacking and what the next year students should focus on.

\subsubsection{Deadline}
All development on the weekplanner ended May 13th. After this all groups were expected to only focus on finishing their report.
We still needed to finish some documentation for next year, as the last usability test was held on May 14th.
Project hand in was on May 28th.
