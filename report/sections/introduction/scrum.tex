\section{Scrum}
This semester scrum is used as the project management process. 
The scrum master group created the \textit{GIRAF Process Manual} and this section describes the most important things in this manual.

\todo{Hvordan henviser vi til GIRAF Process Manualen?}

\subsection{Scrum of Scrums}
The previous semesters suggest that Scrum of Scrums is used as a project management process. 
Scrum of scrums is used to scale scrum to be used for bigger teams.

\subsection{Sprints}
This subsection focuses on how the initial process planning of sprints are conducted. 

The stucture of a sprint is:
\begin{enumerate}
    \item Sprint planning
    \item Scrum of Scrums Stand ups
    \item Skill group meetings
    \item Release Preparation
    \item Sprint Review
    \item Sprint Retrospective
    \item Release Party
\end{enumerate}

\subsubsection{Sprint planning}
\subsubsection{Scrum of scrums stand up}
\subsubsection{Skill group meetings}
\subsubsection{Release preparation}
\subsubsection{Sprint review}
\subsubsection{Sprint retrospective}
\subsubsection{Release party}

\subsection{Sprint scheduling}
These are 4 intial sprints this semester:
\begin{itemize}
    \item Sprint 1: 25/2 - 18/3, 3 weeks
    \item Sprint 2: 18/3 - 8/4, 3 weeks
    \item Sprint 3: 8/4 - 29/4, 3 weeks
    \item Sprint 4: 29/4 - 13/5, 2 weeks
\end{itemize}
\noindent
The last 4 days before a sprint ends is used to prepare a working release build. 
The release then have to be approved by product owners.
The remaining time before project delivery will be used to write on the report and to prepare information and material to future semesters. 

\subsection{Semester roles}
Different from previous semesters, this semester the groups are fullstack teams. 
This means that all groups have responsibility in frontend, backend and servers.
Previous semesters there had been a frontend -, backend -, server -, scrum master - and product owner group. 
According to the previous semesters, this had not worked optimally, and they recommended that the groups became fullstack teams.
Even though that they are fullstack teams, it has been decided that there will still be a scrum master - and product owner group. 
The fullstack groups will get user stories to implement at every sprint.

\subsection{Skill groups}
In every fullstack group there are people with different responsibilities.
There are an accountable person for frontend, backend and server.
The people with the responsibility with one of these areas will have meetings with the accountable from the other groups to share knowledge and discuss problems that have been encountered.
These groups are called skill groups.
Skill groups are suggested to have at least one weekly meeting.

\subsection{Our role}
This semester our role is to function as product owner and be in contact with the customers.
As product owners we strive to have interviews with customers at every sprint to get feedback on the product.
\\
There are some well defined tasks for the product owner group, which are:
\begin{itemize}
    \item Interview customers
    \item Create user stories    
    \item Refine backlog regularly and prioritise user stories
    \item Create prototypes
    \item Create sprint vision and sprint goals
    \item Ensure that the fullstack teams understand the user stories
    \item Conduct usability tests
\end{itemize}
\noindent
The interviews are used to create user stories. 
From user stories prototypes are created, which are later shown to the customers.
When the prototypes are approved they are added to the user stories in the backlog.
\\
At every sprint we make a vision for the sprint, sprint goals and update the backlog. 
The vision and sprint goal is made to ensure that every team knows what they are working towards.
When a sprints ends a usability test is conducted to test new features and get feedback on it.
