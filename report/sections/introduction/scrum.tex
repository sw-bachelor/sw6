\section{The GIRAF process}\label{the-giraf-process}
Scrum is a framework that is used extensively in software development projects.
It is an agile approach to working with complex and changing problems where a normal waterfall model does not work optimally.
\\\\
In this project all groups used a modified version of scrum to structure the groups across the whole GIRAF team.
We originally used Scrum of Scrums which was already a modified version of scrum, but the process group ended up modifying it even further to better fit our workflow.
The sprint structure for our process worked like this:

\begin{itemize}
    \item Sprint Introduction
    \item Stand Up meetings
    \item Skill Group Meetings
    \item Release Preparation
    \item Release Party
    \item Sprint Retrospective
\end{itemize}

\subsection{Sprint Planning} \label{subsec:sprint-planning}
Initially the GIRAF project used sprint planning but later changed to use sprint introduction instead.
Sprint introduction is described after this section.
\\
\\
Sprint planning was the first meeting in a new sprint where each group should have made a prioritized list of which user stories they would want to work with.
Prior to the meeting, the PO group had created user stories based on communication with the customers.
The user stories had prototypes, a definition of what was needed based on the view of a user and a technical description of what was expected to be coded.
\\
\\
The PO group also made relevant and realistic goals for the oncoming sprint, which should translate into a new release of the GIRAF application.
It was important that the goals were reachable to give the participating groups a sense of accomplishment.
This had been an issue in earlier years of the GIRAF project, where groups did not feel that there was a clear improvement in the application which drastically reduced morale.
\\
\\
The meeting started with the PO group presenting, or refreshing, the goals they had set for the whole semester and then more specifically the goals they wanted to be fulfilled in the oncoming sprint.
Afterwards, the groups took a look at the user stories that were in the the backlog and asked clarifying questions if needed.
\\
\\
Then the user stories were distributed among the groups as fairly as possible such that each group had one user story.
Each group then tried to approximate the time required to solve the user story and whether or not it should be split into smaller tasks.


\subsection{Sprint Introduction} \label{subsec:sprint-introduction}
Sprint introduction was the first meeting in a new sprint where all groups and their members were expected to show. 
The meeting started with a presentation from the PO group where they presented the project vision, the sprint vision and the user stories the PO group wanted to be implemented throughout the sprint.
The project vision is an overall vision of what the PO group wants to achieve at the end of the semester.
The sprint vision was what the PO group wanted to achieve throughout the coming sprint.
The user stories that were presented were chosen based on feedback from the customer and what the PO groups deemed was most essential for the application at the current time.
\\
\\
After the meeting, the groups chose a user story from the presented user stories in a first come, first served manner.
Before the groups could begin working, they had to get their choice approved by the PO group.
When the groups  choices were approved by the PO group they could finally start working on their user story.

\subsection{Stand Up}
During a normal sprint week there would be one or two Stand Up meetings.
All groups should send one person to these meetings, but if necessary more could attend, though the goal was to send just the one.
A \textit{Stand Up} meeting took at most 15 minutes.
When 15 minutes had passed, the meeting ended no matter what.
\\
\\
During the meeting each group should present what they had worked on, were working on, and what they would be working on until the next meeting.
They should also notify the others of what problems they faced or were facing, and if they were about to introduce something new that could affect other groups.
Each group representative presented in turns, if there was time afterwards people could ask questions to the other groups.
If there was no time, they had to talk after the meeting.


\subsection{Skill Group Meetings}
During previous years the responsibility of the project was spread out across the groups.
Most groups had one specific area of responsibility.
These were frontend, backend and server.
One group also had the responsibility of being PO and one had scrum masters beside their normal area of responsibility.
This spread of responsibility seemed to be inefficient and caused the groups to feel less responsibility for the project as a whole.
\newline
\newline
In our semester we had full stack groups instead, where all groups where responsible for frontend, backend and server. Every group then had one specialist in each of the areas of responsibility.
This, in theory, should give all groups a sense of responsibility for the project as a whole.\\
All the specialist in the same area then became members of skill groups, so that decisions and experience could be shared across all the regular groups in the project.
The skill groups decided for themselves when they chose to have meetings, but usually they met once a week or more if there were certain difficulties that needed to be addressed.
The skill groups were ultimately responsible for the decision making regarding their area, such as which framework to use or how the code should be standardized.
There were three skill groups which were frontend, backend and server.

\subsection{Release Preparation}
As the sprint came to an end the sprint goals should hopefully have been completed.
This meant a lot of new features had been made which needed to be added to the release version of the application.
Two working days before the sprint ended the PO group opened a release branch.
This branch was for all the features which the PO group had accepted as finished.
All groups were assigned to review user stories that had been implemented throughout the sprint and creating issues if they found any bugs or anything that was lackluster.
The group that implemented the user story originally was then responsible for fixing these issues before the release preparation was over.

\subsection{Release Party}
The release party was an optional social gathering celebrating the new release. 
The state of the application was presented so that everyone could see what had been accomplished throughout the sprint.
This was a chance for all the groups to socialize with each other and feel a sense of pride due to the new release.

\subsection{Sprint Retrospective}
The sprint retrospective was a big meeting usually held the first working day after the release party.
Every members of all of the groups should participate in this meeting.
The goal of the sprint retrospective was to have a discussion about the process of the sprint, what was good, what was bad, new ideas etc.
All participants were split into discussion groups where all groups on the project should have at least one representative.
\newline
\newline
Each discussion group then talked about their experiences during the sprint and gave feedback and new ideas that could be implemented into the process.
All the topics that were being discussed in the discussion groups were noted, and a list of proposals for things that should change for the next sprint was compiled.
Afterwards the process group created a questionnaire with all of the feedback and everyone in the GIRAF project then had the chance to answer on how much they agreed with the points that were raised in the discussion groups.

