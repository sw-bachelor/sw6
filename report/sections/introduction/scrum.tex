\section{Scrum of Scrums}\label{scrum-of-scrums}
Scrum is a framework that is used extensively in software projects.
It is an agile approach to working with complex and changing problems where a normal waterfall model does not work optimally.
In this project all groups used Scrum of Scrums (SoS) to structure the groups across the whole GIRAF team.
SoS is a modification of Scrum made to scale it better for bigger teams.
Many of the activities are similar to normal Scrum.
The sprint process of SoS works in the following way:

\begin{itemize}
    \item Sprint Planning/Sprint Introduction
    \item SoS Stand Up
    \item Skill Group Meetings
    \item Release Preparation
    \item Sprint Review
    \item Sprint Retrospective
    \item Release Party
\end{itemize}

\subsection{Sprint Planning} \label{subsec:SoS-sprint-planning}
Initially the GIRAF project used sprint planning but later changed to use sprint introduction instead.
Sprint introduction is described after this section.
\\
\\
Sprint planning is the first meeting in a new sprint where each group should have made a prioritized list of which user stories they would want to work with.
Prior to the meeting, the PO group has made user stories based on communication with the customers.
The user stories will have prototypes, a definition of what is needed based on the view of a user and a technical description of what is expected to be coded.
\\
\\
The PO group has also made relevant and realistic goals for the oncoming sprint, which should translate into a new release of the GIRAF software.
It is important that the goals are reachable to give the participating groups a sense of accomplishment.
This has been an issue in earlier years of the GIRAF project, where groups did not feel that there was a clear improvement in the software which drastically reduced morale.
\\
\\
The meeting starts with the PO group presenting, or refreshing, the goals they chose for the whole semester and then more specifically the goals they want fulfilled in the oncoming sprint.
Afterwards, the groups will look at the user stories that are in the the backlog and ask clarifying questions if needed.
\\
\\
Then the user stories are distributed among the groups as fair as possible such that each group has one user story.
Each group then tries to approximate the time required to solve the user story and whether or not it should be split into smaller tasks.

\subsection{Sprint Introduction} \label{subsec:SoS-sprint-introduction}
Sprint introduction is a meeting on the first day of a new sprint where all groups and their members are expected to show. 
The start of the meeting takes the same form as the start of the sprint planning described in \autoref{subsec:SoS-sprint-planning}.
\\
\\
Afterwards, the groups then choose a user story from the highest prioritized user stories in a first come, first served manner.
Before the groups can begin working, they have to get their choice approved by the PO group.
When all the groups have had their choices approved, the sprint introduction is over and development can begin.

\subsection{SoS Stand Up}
During a normal sprint week there will be at least one SoS Stand Up meeting.
All groups should send one person to these meetings, but if necessary more can attend, though the goal should be to send just the one.
A \textit{Stand Up} meeting takes at most 15 minutes.
When 15 minutes have passed, the meeting ends no matter what.
\newline
\newline
During the meeting each group should present what they have worked on, are working on, and what they will work on until the next meeting.
They should also notify the others of what problems they faced or are facing, and if they are about to introduce something new that could affect other groups.
Each group representative presents in turns, if there is time afterwards people may ask questions to the other groups.
If there is no time, they have to talk after the meeting.


\subsection{Skill Group Meetings}
During previous years the responsibility of the project was spread out across the groups.
Most groups had one area of responsibility.
These were frontend, backend and server.
One group also worked as PO and one as scrum masters beside their normal area of responsibility.
This spread of responsibility seemed to have been inefficient and caused the groups to feel less responsibility for the project as a whole.
\newline
\newline
Therefore, the groups were advised to split into skill groups, where each group has at least one specialist in each of the areas of responsibility.
This, in theory, should give all groups a sense of responsibility for the project as a whole.
\newline
\newline
All the specialist in the same area then become members of skill groups, so that decisions and experience can be shared across all the regular groups in the project.
The skill groups decide for themselves when they choose to have meetings, but usually they meet once a week or more if there are certain difficulties that need to be fixed.
The skill groups are ultimately responsible for the decision making regarding their area, such as which framework to use or how the code should be standardized.

\subsection{Release Preparation}
As the sprint comes to an end the sprint goals should hopefully have been completed.
This means a lot of new features have been made which need to be added to the release version of the program.
Four days before the sprint ends the PO group opens a release branch.
This branch is for all the features which the PO group have accepted as finished and all the groups are responsible for their own code functioning properly with the rest of the newly added features.


\subsection{Sprint Review} \label{subsec:SoS-sprint-review}
The sprint review is a meeting at the end of a sprint. 
Each group should be represented by one of the group members. 
It can be an advantage to have multiple members of the PO group present if questions about some of the features that have been developed should arise.
Each group presents the current state of their work, gives feedback on the quality of the product, as well as discuss new issues that may have arisen and could be included in the next sprint.
After all the groups have presented there is an open discussion about the projects state and the work of the different groups.

\subsection{Sprint Retrospective}
The sprint retrospective is a big meeting usually held right after the sprint review.
All members of all groups should participate in this meeting.
The goal of the sprint retrospective is to have a discussion about the process of the sprint, what was good, what was bad, new ideas etc.
All participants are split into discussion groups where all groups on the project should have at least one representative.
\newline
\newline
Each discussion group then talks about their experiences during the sprint and comes up with feedback and new ideas that could be implemented into SoS.
All the topics that are being discussed in the discussion groups are noted, and a list of proposals for things that should change for the next sprint is compiled.
Each attendee has three votes that can be used to vote for what they think are good proposals.
Afterwards the process group takes the most voted for proposals and decides if some version of these should be incorporated into the new sprint.

\subsection{Release Party}
The release party is an optional social gathering celebrating the new release. 
This is a chance for all the groups to socialize with each other and feel a sense of pride due to the new release.

