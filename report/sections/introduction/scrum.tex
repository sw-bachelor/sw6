\section{Scrum of Scrums}
Scrum is a framework that is used extensively in software projects.
Its an agile approach to working with complex and changing problems where a normal waterfall model does not work optimally.
In this project we used Scrum of Scrums (SoS) to structure the groups across the whole GIRAF team.
SoS is a modification of Scrum made to scale it better for bigger teams.
Many of the activities are similar to normal Scrum.
The sprint process of SoS works in the following way:

\begin{itemize}
    \item Sprint Planning
    \item SoS Stand Up
    \item Skill Group Meetings
    \item Release Preparation
    \item Sprint Review
    \item Sprint Retrospective
    \item Release Party
\end{itemize}

\subsection{Sprint Planning}
Sprint planning is a meeting on the first day of a new sprint where all groups are expected to show.

Prior to the meeting the PO-group has made user stories based on communication with the customers.
The user stories will have prototypes, a definition of what is needed based of the view of a user and a technical description of what is expected to be coded.

The PO-group has also made relevant and realistic goals for the oncoming sprint, which should translate into a new release of the GIRAF software for the app store.
It is important that the goals are reachable to give the groups a sense of accomplishment.
This has been an issue in earlier years were groups did not feel that there was a clear improvement in the software which drastically reduced moral.

The meeting starts by the PO-group presenting, or refreshing, the goals they chose for the whole semester and then more specifically the goals they want fulfilled in the oncoming sprint.
Afterwards the groups will look at the user stories that is in the the backlog and ask clarifying question if needed.

The groups then choose a user story from the highest prioritized user stories.
Before the groups can begin working they have to get their choice approved by the PO-group.
When all the groups have been approved the sprint planning is over and development can begin.

\subsection{SoS Stand Up}
During a normal sprint week there will be at least one SoS Stand Up meeting.
All groups should send at least one person to these meetings but if necessary more can attend, though the goal should be to send as few as possible.
A Stand Up meeting takes at most 15 minutes, when 15 minutes has past the meeting ends no matter what.

During the meeting each group should present what they have, and are, working on and what they will work on until the next meeting.
They should also notify the others of what problems they had faced or are facing and if they are about to introduce something new that could affect other groups.
Each group representative takes turn to present, if there is time afterwards people may ask questions to the other groups else they have to talk after the meeting.


\subsection{Skill Group Meetings}
Based on advise from last years groups we chose to implement skill groups.
Compared to previous years where a whole group had a role such as frontend, backend or server, these responsibilities has now been spread out across groups so that each group has at least one person responsible for frontend, backend or server.



\subsection{Release Preparation}

\subsection{Sprint Review}

\subsection{Sprint Retrospective}

\subsection{Release Party}


