\section{Technologies and Tools}
This section describes the technologies and tools that are used in this project. 
Some of them are used to facilitate the collaboration between all the groups in the GIRAF project while others are used internally in our group.
\\\\
\textbf{Jira}\\
Jira is a software development tool developed by Atlassian and is used for agile software development.
The software facilitates the creation of a backlog of user stories that can then be assigned to a sprint.
The team can assign story points to each assignment and assign a user to the user story, to distribute the workload properly over the coming sprint.
Jira also includes multiple tools for managing and monitoring sprints and their progress, to help with retrospectives and to ensure the sprint is proceeding as planned.
We used it for our weekly sprints that were run internally in the group.
\\\\
\textbf{Adobe XD}\\
Adobe XD is a program for prototype creation, that is easy to pick up and create simple designs in.
Adobe XD makes it easy to reuse components in multiple design projects and to collaborate with others.
It also lets you assign functionality to the prototypes, meaning they can be used for usability testing with the users to demonstrate the functionality.
\\\\
\textbf{GitHub}\\
GitHub is a development platform that makes it possible for multiple people to collaborate on a project. 
All of the code in the GIRAF project is hosted on GitHub.
The issue and project features are used to create and assign user stories to the different groups that are working on the GIRAF project and to manage the sprints. 
The GIRAF wiki is also hosted on GitHub.
\\\\
\textbf{Slack}\\
Slack is a collaboration hub where users can create a workspace that they can invite their collaborators to.
It is possible to create multiple channels with independent communication. 
The collaborators can then choose which channels they want to join.
Slack has been used for all communication across the participating groups of the GIRAF project.
