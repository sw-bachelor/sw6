\section{Technologies and Tools}
This section describes the technologies and tools that are used in this project. 
Some of them are used to facilitate the collaboration between all the groups in the GIRAF project while others are just used internally in our group.

\paragraph{Jira}~\\
Jira is a software development tool that is developed by Atlassian and is used for agile software development.
It is possible in Jira to create a backlog of user stories that can then be assigned to a sprint. The user can assign story points to each assignment and assign a user to the user story.
Jira also includes multiple tools for managing and monitoring sprints and their progress.
We used it for our weekly sprints that we had in our group.

\paragraph{Adobe XD}~\\
Adobe XD is a program developed by Adobe that used to create prototypes.
It is a simple program that is easy to pick up and create simple designs in.
Adobe XD makes it easy to reuse components in multiple design projects and to collaborate with others.

\paragraph{GitHub}~\\
GitHub is a development platform that makes it possible for multiple people to collaborate on a project. 
All of the code in the GIRAF project is hosted on GitHub.
The issue and project features are used to create and assign user stories to the different groups that are working on the GIRAF project and to manage the sprints. 
The GIRAF wiki is also hosted on GitHub.

\paragraph{Slack}~\\
Slack is a Collaboration hub where users can create a workspace that they can invite their collaborators to.
It is then possible to create multiple channels. The collaborators can then choose which channel they want to join.
Slack has been used for all communication across the GIRAF project.
