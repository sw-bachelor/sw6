\section{Technologies and Tools}\label{technologies-and-tools}
This section describes the technologies and tools that were used in this project. 
Some of them were used to facilitate the collaboration between all the groups in the GIRAF project while others were used internally in our group.
\\\\
\textbf{Jira}\\
Jira is a software development tool developed by Atlassian.
The software facilitates the creation of a backlog of user stories that can then be assigned to a sprint.
The team can assign story points to each assignment and assign a user to the user story.
Jira also includes multiple tools for managing and monitoring sprints and their progress, to help with retrospectives and to ensure the sprint is proceeding as planned.
We used Jira for weekly sprints internally in our group, but this was not used by the GIRAF project.
\\\\
\textbf{Adobe XD}\\
Adobe XD is a program for prototype creation.
The program enables reuse of components in multiple design projects and collaboration with others using cloud functionality.
It also lets you assign functionality to the prototypes, meaning they can be used for usability testing.
We used this program for all the prototypes that were created during this semester.
\\\\
\textbf{GitHub}\\
GitHub is a development platform that makes it possible for multiple people to collaborate on a project. 
All of the code in the GIRAF project is hosted on GitHub.
The issue and project features are used to create and assign user stories to the different groups that are working on the GIRAF project and to manage the sprints. 
\\\\
\textbf{Slack}\\
Slack is a collaboration hub where users can create a workspace they can invite collaborators to.
It is possible to create multiple channels with independent communication. 
The collaborators can choose which channels they want to join.
Slack has been used for all communication across the participating groups of the GIRAF project.
