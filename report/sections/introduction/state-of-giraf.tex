\section{State of GIRAF - February 2019}\label{sec:stateOfGirafFeb2019}
The purpose of this section is to describe the current state of GIRAF as it was delivered to us.
This was done to ensure that we had a basic understanding of what our starting point was, but also to learn and understand the work that was done on the project previously.
\\\\
In the previous years a lot of the back end was rewritten, which meant that several of the applications no longer worked correctly.
This meant that last year's focus was to get the weekplanner application working again as a minimum viable product.
\\\\
When we first looked at the weekplanner it was not very stable or responsive, and was still missing some convenient functionalities.
We tested the login functionality on a tablet several times and it did not seem to function properly.
Instead of accepting the login information, the application loaded for a long time until the user eventually received a message that a problem \textit{might} have occurred.
Therefore, functionality could not be tested further on a tablet.
Likewise, it was not very clear to the user that the application was processing the login information, as there was no spinner or loading bar to indicate this.
This resulted in the application being seemingly frozen when it was loading.
On a phone we were able to successfully login and choose a citizen from the \texttt{choose citizen} page after several attempts.
\\\\
The weekplanner application had the following functionalities:
\begin{itemize}
    \item \textbf{Choosing a week plan:} After choosing a citizen the guardian could choose an already existing week plan for that citizen.
    The guardian could also choose to create a new week plan.
    \item \textbf{Week plan overview:} After choosing an existing week plan or choosing to create a new week plan,the user was redirected to the week plan overview.
    It was possible to see all of the days of the week and which activities were planned for these days.
    There were switches that allow the guardian to delete a weekday plan, and buttons that allowed the guardian to add activities to a weekday.
    A slidebar functionality was added when there were too many activities on a weekday to be able to show on the week plan.
    \item \textbf{Creating a new week plan:} When the guardian chose to create a new week plan, the application redirected to an input page.
    Here the guardian could enter a name for the week plan, choose the year and week and also choose a pictogram that represented the week plan.
    Finally, the guardian could choose to create an empty week plan or use an already existing template to build it.
    \item \textbf{Creating a new template:} A template can be used when creating a new week plan.
    The guardian could choose to create a new template, and this worked the same way as creating a week plan.
    \item \textbf{Deleting a weekday:} The guardian could choose to delete one of the weekdays when viewing a week plan.
    Each weekday had a switch to allow the guardian to delete it.
    When this switch was pressed, a window popped up to ask the guardian if they were sure they wanted to delete the week plan.
    The pop up asking if the guardian was sure they want to delete the weekday was a bit misleading, because the functionality worked more like a hide functionality that hides the weekday from view.
    The weekday was not being deleted, because the guardian could press the switch for the weekday again to make it visible.
    \item \textbf{Saving a week plan:} The guardian had the ability to save changes made to a week plan by clicking a button.
    Another functionality that had been implemented was that it alerted the user if they left the week plan overview with unsaved changed.
    The user would get a last opportunity to save the changes they made before they were lost.
    \item \textbf{Switching from guardian to citizen:} In the top bar there was an icon that allowed the user to switch between guardian and citizen.
    When this icon was pressed it was not very responsive, but eventually it switched the user to a different mode.
    \item \textbf{Switching from citizen to guardian:} The application sometimes did not respond well when clicking the icon again to switch back to guardian mode.
    Eventually the user was redirected to the login page to login as a guardian.
    \item \textbf{Adding an activity:} A guardian could add activities to the weekdays in a week plan.
    This was done by pressing an \textit{Add} button at the bottom of the weekday.
    This action would redirect the guardian to a page where they could search for a pictogram that resembled the activity they wished to add.
    The search functionality on this page sometimes gave weird results.
    One problem on this page was that the guardian was not able to see pictograms during the search, but instead the whole screen was covered by the keyboard.
    \item \textbf{Interacting with activities from the week plan overview:}  The user was able to interact with an activity after adding it to a weekday.
    First of all, activities could be dragged up and down to change the order of the activities for the day.
    Secondly, each activity could be interacted with by pressing it which takes the user to the activity's page.
    Here the user could delete the activity or mark the activity as completed.
    The user could then save the change to the activity which returned them to the week plan overview.
\end{itemize}
Overall, the weekplanner had a lot of functionalities implemented, but it still had many flaws which needed to be improved upon.
