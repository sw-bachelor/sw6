\section{State of Giraf}
The purpose of this section is to describe the current state of Giraf as it was delivered to us. 
This is done so that we have an idea of what our starting point will be,
but also to learn about what work was done on the project last year and what work was left undone. 
\\
Previously much of the backend had been rewritten which meant that many of the apps were no longer working correctly. 
This meant that last year's focus was to atleast get the weekplanner working again as a minimal viable product.
\\
The overall state of the weekplanner is a somewhat working application, but it is not very stable and still missing some convenient functionalities.
\\\\
At the moment the weekplanner is not very stable or responsive. The login does not work when trying to login from a tablet. 
Instead the app is loading for a long time until the user eventually gets a message that a problem has occurred. 
Therefore functionality can not be tested further on the tablet. 
On the phone we are able to succesfully login and then choose a citizen from the 'choose citizen' page.
\\\\
Functionality in the weekplanner:
\begin{itemize}
    \item \textbf{Choosing a weekplan:} After choosing a citizen there is funcitonality that allows the guardian to choose an already existing weekplan for that citizen
    \\
    \item \textbf{Weekplan overview:} After choosing an existing weekplan or choosing to create a new weekplan, the user is redirected to the weekplan overview. 
    Here it is possible to see all of the days of the week and what activities that are planned for these days. There are switches that allows the guardian to delete a weekday plan, and buttons that allow the guardian to add activities to a weekday. 
    A slidebar functionality is added when there are too many activies on a weekday to be able to show on the weekplan.
    \\
    \item \textbf{Creating a new weekplan:} Choose a name for weekplan, choose year and week for weekplan, choose pictogram from the week, choose between making a fresh weekplan or making one from a template. Seems to work
    \item \textbf{Creating a new template:} 
    \item \textbf{Deleting a weekday:} Is more like a hide functionality
    \item \textbf{Saving a weekplan:} Works. Asks to save unsaved changes when pressing back button or switching usermode.
    \item \textbf{Switching from guardian to citizen:} In the top bar there is an icon that allows the user to switch between guardian and citizen. 
    When this icon is pressed it is not very responsive but eventually it does switch the user to a different mode. 
    When it is switched to citizen, the weekplan view changes to view the current day's activities.
    \\
    \item \textbf{Switching from citizen to guardian:} When clicking the icon again to switch back to guardian mode, the app sometimes does not respond very well. 
    Eventually the user is redirected to the login page to login as a guardian again.
    \\
    \item \textbf{Adding an activity:} 

    \item \textbf{Interacting with activities from the weekplan overview:} '
    \\
\end{itemize}