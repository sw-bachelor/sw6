\section{State of GIRAF - February 2019}\label{sec:stateOfGirafFeb2019}
The purpose of this section is to describe the current state of GIRAF as it was delivered to us.
This is done to ensure that we have a basic understanding of what our starting point will be, but also to learn and understand the work that was done on the project previously and what work was left undone.
\\\\
In the previous years a lot of the backend was rewritten, which meant that several of the apps are no longer working correctly.
This meant that last year's focus was to, at least, get the weekplanner working again as a minimum viable product.
\\\\
When we first looked at the weekplanner it was not very stable or responsive, and was still missing some convenient functionalities.
We tested the login functionality on a tablet several times and it did not seem to function properly.
Instead of accepting the login information, the application loaded for a long time until the user eventually receives a message that a problem \textit{might} have occurred.
Therefore, functionality can not be tested further on the tablet at the moment.
Likewise, it was not very clear to the user that the application was processing the login information, as there was no spinner or loading bar to indicate this.
This resulted in the application being seemingly frozen when it was loading.
On the phone we were able to successfully login and then choose a citizen from the \texttt{choose citizen} page after several attempts.
\\\\
The weekplanner application currently has the following functionalities:
\begin{itemize}
    \item \textbf{Choosing a week plan:} After choosing a citizen there is functionality that allows the guardian to choose an already existing week plan for that citizen.
    On this page the guardian can also choose to create a new week plan.
    \\
    \item \textbf{Week plan overview:} After choosing an existing week plan or choosing to create a new week plan, the user is redirected to the week plan overview.
    Here it is possible to see all of the days of the week and which activities are planned for these days.
    There are switches that allow the guardian to delete a weekday plan, and buttons that allow the guardian to add activities to a weekday.
    A slidebar functionality is added when there are too many activities on a weekday to be able to show on the week plan.
    \\
    \item \textbf{Creating a new week plan:} When the guardian chooses to create a new week plan, the application redirects to an input page.
    Here the guardian can enter a name for this specific week plan, choose the year and week for the week plan and also choose a pictogram to represent the week plan.
    Finally, the guardian can choose to create a fresh week plan or use an already existing template to build it. The functionality used for creating a week plan seems to be working correctly.
    \\
    \item \textbf{Creating a new template:} As mentioned before, a template can be used when creating a new week plan.
    On this page the guardian can choose to create a new template, and this basically works the same way as creating a week plan.
    Creating a template and then saving it with a given name seemed to be working fine.
    \\
    \item \textbf{Deleting a weekday:} The guardian can choose to delete one of the weekdays when viewing a week plan.
    Each weekday has a switch to allow the guardian to delete it.
    When this switch is pressed, a window pops up to ask the guardian if they are sure they want to delete the week plan to which the guardian can answer yes or no.
    The pop up asking if the guardian is sure they want to delete the weekday is a bit misleading, because the functionality works more like a hide functionality that hides the weekday from view.
    The weekday is not being deleted as the application says, because the guardian can press the switch for the weekday again to make it visible again.
    \\
    \item \textbf{Saving a week plan:} The guardian has the ability to save changes made to a week plan.
    A change can be, for example, adding a new activity to a weekday.
    A button can then be pressed to save the changes made to the week plan, and this functionality seems to be working fine.
    Another functionality that has been implemented is that it alerts the user if they leave the week plan overview with unsaved changed.
    The user then gets a last opportunity to save the changes they made before they are lost.
    \\
    \item \textbf{Switching from guardian to citizen:} In the top bar there is an icon that allows the user to switch between guardian and citizen.
    When this icon is pressed it is not very responsive, but eventually it does switch the user to a different mode.
    The week plan view changes to view the current day's activities when it is switched to citizen.
    \\
    \item \textbf{Switching from citizen to guardian:} The application sometimes does not respond very well when clicking the icon again to switch back to guardian mode.
    Eventually the user is redirected to the login page to login as a guardian again.
    \\
    \item \textbf{Adding an activity:} A guardian can add activities to the weekdays in a week plan.
    This is done by pressing an \textit{Add} button at the bottom of the weekday the guardian wants to add an activity to.
    This action will redirect the guardian to a page where they can search for a pictogram that resembles the activity they wish to add.
    The search functionality on this page sometimes gives weird results, but overall it works okay.
    One problem on this page is that the guardian is not able to see pictograms during the search, but instead the whole screen is taken up by the keyboard.
    Other than that, the functionality to add activities to a weekday works fine.
    \\
    \item \textbf{Interacting with activities from the week plan overview:}  The user is able to interact with an activity after adding it to a weekday.
    First of all, activities can be dragged up and down to change the order of the activities for the day.
    Secondly, each activity can be interacted with by pressing it which takes the user to the activity's page.
    Here the user can delete the activity or mark the activity as done.
    The user can then save the change to the activity which returns them to the week plan overview.
    \\
\end{itemize}
Overall, the weekplanner had a lot of functionalities implemented, but it still had many flaws which needed to be improved upon.
The PO group from last year had compiled a list of bugs that needed to be fixed, and they had suggested a few important user stories which included functionalities that the customer would like to see implemented in the application.
