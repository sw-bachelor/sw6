\section{The process in our group}

The structure of the work process in our own group was inspired by the Scrum process.
We only chose certain aspects of Scrum to reduce the amount of meetings we needed to have internally in our group. 
Since we all met everyday in the group room and we had a lot of meetings related to the GIRAF project we found that this solution worked the best for us.
From Scrum we chose these aspect to use in our process.

 \begin{itemize}
    \item Product Backlog
    \item Sprint Backlog
    \item Product Increment
    \item Sprints
    \item Sprint planning
    \item Planning poker
    \item Product owner
    \item Retrospectives
 \end{itemize}

 To organize all our internal user stories we used the product backlog.
 In there each member of the group could add user stories if there was something that we needed to do.
 This differs from Scrum where only the Product Owner would handle the Product Backlog.
 The reason for this is that we decided that each sprint there would be appointed a new Product Owner which responsibility was solely to decide which user stories should be added to the new Sprint Backlog and in what priority.
 This was because we internally did not have a customer to contact, instead the person appointed Product Owner also worked as a regular developer with the job as Product owner as an user story.
 \\
 Before a new sprint would start the previously Product Owner would have chosen and prioritized user stories for the new Sprint Backlog.
 The whole group would then partake in planning poker where we would estimate how much each user story would take to complete.
 Afterwards each member would pick some user stories until all user stories was taken.
 We would always try to distribute the user stories evenly so that each member had the same amount of story points.
 \\
 A sprint 
 \\
 At the end of a sprint we would have a retrospective meeting where each member would tell how far they got with their user stories and what could have been better so that they would have been more productive.
 We later decided to stop holding retrospective meetings because the feedback was always the same and eventually no one had anything worthy of an extra meeting to add to the process.


