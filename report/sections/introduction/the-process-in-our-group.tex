\section{The process in our group}

The structure of the work process in our own group was inspired by the Scrum process.
We only chose certain aspects of Scrum to reduce the amount of meetings we needed to have internally in our group. 
Since we all met everyday in the group room and we had a lot of meeting related to the GIRAF project we found that this solution worked the best for us.
From Scrum we chose these aspect to use in our process.

 \begin{itemize}
    \item Product Backlog
    \item Sprint Backlog
    \item Product Increment
    \item Sprints
    \item Sprint planning
    \item Planning poker
    \item Product owner
    \item Retrospectives
 \end{itemize}

 To organize all our internal tasks we used the product backlog, which was handled by Jira.
 In here, each member of the group would add tasks or user stories if they felt something was missing.
 This differs from regular Scrum where only the product owner would handle the product backlog.
 This is because we decided that each sprint there would be appointed a new product owner in the group, whose responsibility was to decide which tasks should be added to the sprint backlog and prioritize them.
 Since we did not have a customer internally in the group, who could be contacted, the role of the product owner was reduced to this, while also working as a regular developer.
 \\
 Before a new sprint would start the previously product owner would have chosen and prioritized task for the sprint backlog.
 The whole group would then partake in planning poker where we would estimate how much each task would take to complete.
 \\
 At the end of the sprint we would 

