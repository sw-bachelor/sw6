\section{The process in our group}
The structure of the work process in our own group was inspired by the scrum process.
We only chose certain aspects of scrum to reduce the amount of meetings we needed to have internally in our group. 
Since we all met everyday in the group room and had a lot of meetings related to the GIRAF project we found that this solution worked the best for us.
From scrum we chose these aspects to use in our process:

 \begin{itemize}
    \item Product backlog
    \item Sprint backlog
    \item Product increment
    \item Sprints
    \item Sprint planning
    \item Planning poker
    \item Product owner
    \item Retrospectives
 \end{itemize}
To organize all our internal tasks we used the product backlog.
Each member of the group could add tasks.
This differs from scrum where only the product owner would handle the product backlog.
The reason for us allowing this is that we decided that with each new sprint a new product owner would be appointed whose responsibility was solely to decide which tasks should be added to our new sprint backlog and their priority.
The person appointed product owner also worked as a regular developer.
\\\\
Before a new sprint would start the previous product owner would have chosen and prioritized tasks for the new sprint backlog.
The whole group would then partake in planning poker where we would estimate how long each task would take to complete, expressed through points.
Afterwards, each member would pick some tasks until all were delegated.
We would always try to distribute the tasks evenly, such that each member had the same amount of points.
\\\\
Our internal sprints were one week long.
At the end of a sprint we would have a retrospective meeting where each member would describe how far they got with their tasks, and what could have been better so that they would have been more productive.
We later decided to stop holding retrospective meetings because the feedback tended to become uniform,and eventually this extra meeting became superfluous.


