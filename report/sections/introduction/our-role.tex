\section{Our role}
Our role this semester is to function as product owners of the GIRAF project and to be in charge of customers contact.
As product owners, we strive to have interviews with customers at the end of every sprint to get feedback and conduct usability tests on the product.
The primary goal for our group is to maximize the value for the customers. 
This will be done by prioritising the user stories so that the customers will get as much value as early as possible \autocite{TheScrumGuide}.\\
\\
There are some well defined tasks for the product owner group, which are:
\begin{itemize}
    \item Interview customers
    \item Create user stories    
    \item Refine the backlog regularly and prioritise user stories
    \item Create prototypes
    \item Create a sprint vision and sprint goals
    \item Ensure that the development teams understand the user stories
    \item Approve or decline features made by the development teams
    \item Conduct usability tests
\end{itemize}
\noindent
The interviews are the foundation for understanding the requirements of the program and are used to create user stories. 
Prototypes are created from the user stories.
These prototypes are used to create an initial design for the new features that the user stories might include. 
This is done so that we can have a visual presentation of the user stories which makes it easier for us to communicate with the customer and make sure that we understand each other. 
This way we can get a confirmation that the design ideas we have been working on safisfies the needs of the customer.
When the prototypes are approved by the customer they are added to the user stories in the backlog, so that the development teams can use them as a frame of reference when implementing them.
\\\\
At the start of every sprint we make a vision for the sprint, define sprint goals and update the backlog to deliver value. 
The vision and sprint goal are made to ensure that every team knows what they are working towards, and to motivate them to deliver a good product.
When a sprint ends, a usability test is conducted on the newest release to test new features and to get feedback in order to determine what should be developed next, and whether or not the new functionality is acceptable.
