\section{Product vision}
Based on state of the GIRAF project and our communication with the customers we defined some goals for the project that we wanted to realize this semester.
\\\\
The customers' biggest wish was to simply get something the application to a point where they can actually use it.
Some of the customers have been participating in the project for 7 years now without getting a proper working version.
It was important for them that they would get something stable and useful, and as such extra functionality was not necessary for them.
Stability is really important as people who have autism does not react well to unforeseen changes, so a program crash can cause a massive problem for the citizen, as well as the guardians using the software.
Based on this we chose to solely focus on the weekplanner application.
Our main goal is to get that application as stable as possible and rework the design so that it is more intuitive for for the users.
\\\\
Another goal is to make the application available for iOS devices, as many of the citizens that would be using it are have access to iPads.
Furthermore, we would like the application to still function without an internet connection, to ensure that it can be used even if the internet should go down at the users institution.
Last, but not least, we would like to make it easier and more accessible for the next year students to get started with the GIRAF project.
Ideally this would drastically reduce the time and effort they need to start making contributions to the project.