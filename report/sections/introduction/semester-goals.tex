\section{Product vision}\label{sec:introduction-product-vision}
Based on the state of the GIRAF project and our communication with the customers we defined some goals for the project that we wanted to realize this semester.
\\\\
The customers' biggest wish was to simply get the application to a point where they could actually use it.
Some of the customers have been participating in the project for 7 years now without getting a proper working version.
It was important for them that they would get something stable and useful, and as such extra functionality was not necessary for them.
Stability is really important for people with autism as they do not react well to unforeseen changes, so an application crash could cause a massive problem for the citizens, as well as the guardians.
Based on this we chose to solely focus on the weekplanner application.
Our main goal was to get the application as stable as possible and redesign it so that it was more intuitive for the users.
\\\\
Another goal was to make the application available for iOS devices, as many of the citizens that would be using it used iPads.
Furthermore, we would like the application to still function without an internet connection, to ensure that it can be used even if the internet should go down at the users institution.
Last, but not least, we wanted to make it easier and more accessible for the next year's students to get started with the GIRAF project.
This was done through thorough documentation and by moving all the documentation from Phabricator and all the repositories from GitLab to GitHub to make it more centralized.
Ideally this would drastically reduce the time and effort they need to start making contributions to the project.
