\section{About GIRAF}
GIRAF (Graphical Interface Resource for Autistic Folk) is an ongoing project developed by 6th semester software engineering students at Aalborg University. 
The project has been continuously developed on since 2011 with Ulrik Mathias Nyman as project coordinator, with the new students assuming responsibility and learning to cooperate in a bigger environment with an existing codebase. 
\\\\
GIRAF is an application that serves the purpose of helping people with autism, with the primary user group being children.
The primary goal of the application is to provide visual representation of the daily or weekly schedule for the users.
During the lifetime of the project, different types of games and communication tools to help with education have been implemented, but most of these functionalities do not work after the API rework of 2017. The current focus of the GIRAF project is to make the weekplanner stable and fit for use, before resuming work on the other parts of the project. 
\\\\
A special aspect of the project, is the direct interaction with customers, who are essential for the project.
The customers serve to define requirements of the application and facilitate the familiarization of students with industry processes.
\\
\noindent
Currently the institutions that are represented are: 
\begin{itemize}
    \item Mette and Emil, Egebakken (School)
    \item Kristine and Susanne, Birken (Kindergarten)
    \item Flemming, Center for Autism
    \item Niels, IT manager in the elderly and disability administration.
\end{itemize}
