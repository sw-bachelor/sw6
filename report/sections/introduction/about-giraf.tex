\section{About GIRAF}
GIRAF (Graphical Interface Resource for Autistic Folk) is an ongoing project developed by 6th semester students studying software engineering at Aalborg University. 
The project has been passed on every year since 2011 with Ulrik Mathias Nyman as project coordinator, with the new students assuming responsibility and learning to cooperate in a bigger environment with an existing codebase. 
GIRAF is a program developed for the purpose of helping people with autism, mainly students at schools that specialize in the subject.
Its primary use is to assist by providing a visual representation of the daily and weekly schedule of these people.
Different types of games and communication tools to help with education were implemented previously, but these functionalities have since been removed to focus on creating the ability to represent schedules properly.
\\
GIRAF is a project with specific customers whose needs must be adhered to. 
These customers are essential for the project, as they define the requirements of the program and facilitate the familiarization of students with industry processes.
The customers and the institutions they represent are: 
\begin{itemize}
    \item Mette and Emil, Egebakken (School)
    \item Kristine and Susanne, Birken (Kindergarten)
    \item Flemming, Center for Autism
    \item Niels, IT manager in the elderly and disability administration.
\end{itemize}

\subsection{State of Giraf - February 2019}
