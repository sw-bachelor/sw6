\section{Semester roles}\label{sec:semesterRoles}
The groups for this semester acted as full stack teams, which was different from previous semesters. 
This meant that all of the groups had responsibilities related to frontend, backend and servers.
Previous years saw groups specializing in certain areas, so there was a frontend -, backend -, server -, scrum master - and product owner group. 
According to the previous years, this had not worked optimally, and they recommended that the groups became full stack teams to ensure all groups had knowledge in all areas.
Even though we decided to be full stack teams, we also decided that there should still be a dedicated process group and a product owner group. 
Every full stack group received user stories to implement in every sprint.

\subsection{Process Group}
The process group was responsible for facilitating the work process across all the other groups.
They arranged meetings and tried to optimize the meetings to avoid wasting time.
\\
There were some well defined tasks for the process group which were:
\begin{itemize}
    \item Decide process changes during the semester
    \item Host the sprint planning meetings
    \item Facilitate stand up meetings for all groups
    \item Plan the first skill group meetings
    \item Host the sprint retrospective meetings
\end{itemize}

\subsection{Product Owner Group}
Our role this semester was to function as product owners of the GIRAF project and to be responsible for customer contact. 
As product owners, we strove to have interviews with the customers at the end of every sprint to get feedback and conduct usability tests on the product.
The primary goal for our group was to maximize the value of the application for the customers. 
This was done by prioritizing user stories so that the customers got as much value as possible \autocite{TheScrumGuide}.\\
\\
There were some well defined tasks for the product owner group, which were:
\begin{itemize}
    \item Interview customers
    \item Create user stories    
    \item Refine the backlog regularly and prioritize user stories
    \item Create prototypes
    \item Create a sprint vision and sprint goals
    \item Ensure that the development teams understand the user stories
    \item Approve or decline features made by the development teams
    \item Conduct usability tests
\end{itemize}
