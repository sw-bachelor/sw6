\section{Semester roles}
The groups for this semester act as full-stack teams, which is different from previous semesters. 
This means that all groups have a responsibility in frontend, backend and servers.
Previous years saw groups specialising in certain areas, so there was a frontend -, backend -, server -, scrum master - and product owner group. 
According to the previous years, this had not worked optimally, and they recommended that the groups became full-stack teams to ensure all groups had knowledge in all areas.
Even though there are full-stack teams, it was decided that there should still be dedicated scrum master and product owner groups. 
The full-stack groups will all receive user stories to implement in every sprint.

\subsection{Scrum master}
Scrum master are the group responsible for facilitating the scrum process across all the other groups.
They try to make meetings run as efficiently as possible to avoid to wasted time.
The scrum master group also defined the processes to be followed.
\\
There are some well defined tasks for the scrum master group which are:
\begin{itemize}
    \item Decide processes during semester
    \item Host sprint planning
    \item Facilitate SOS stand up meetings
    \item Plan first skill group meeting
    \item Host sprint retrospective
\end{itemize}
\noindent
The scrum master group decided the initial processes for this semester based on feedback and discussions with all GIRAF members.
During the semester they evaluate the process models to investigate if the processes work as optimally as possible.
\\
In SOS stand up meetings 15 minutes are allocated to the meeting, and it is the scrum groups responsibility that it does not go over time.

\subsection{Product owner}
Our role this semester is to function as product owners of the GIRAF project and to be in charge of customers contact. 
As product owners, we strive to have interviews with customers at the end of every sprint to get feedback and conduct usability tests on the product.
The primary goal for our group is to maximize the value for the customers. 
This will be done by prioritising the user stories so that the customers will get as much value as early as possible \autocite{TheScrumGuide}.\\
\\
There are some well defined tasks for the product owner group, which are:
\begin{itemize}
    \item Interview customers
    \item Create user stories    
    \item Refine the backlog regularly and prioritise user stories
    \item Create prototypes
    \item Create a sprint vision and sprint goals
    \item Ensure that the development teams understand the user stories
    \item Approve or decline features made by the development teams
    \item Conduct usability tests
\end{itemize}
\noindent
The interviews are the foundation for understanding the requirements of the program and are used to create user stories. 
Prototypes are created from the user stories.
These prototypes are used to create an initial design for the new features that the user stories might include. 
This is done so that we can have a visual presentation of the user stories which makes it easier for us to communicate with the customer and make sure that we understand each other. 
This way we can get a confirmation that the design ideas we have been working on satisfies the needs of the customer.
When the prototypes are approved by the customer they are added to the user stories in the backlog, so that the development teams can use them as a frame of reference when implementing them.
\\\\
At the start of every sprint we make a vision for the sprint, define sprint goals and update the backlog to deliver value. 
The vision and sprint goal are made to ensure that every team knows what they are working towards, and to motivate them to deliver a good product.
When a sprint ends, a usability test is conducted on the newest release to test new features and to get feedback in order to determine what should be developed next, and whether or not the new functionality is acceptable.

\subsection{Development teams}
At the beginning of a sprint the development teams choose one or more user stories they will work on in the next sprint.
During the sprint planning the development team splits the user stories into tasks in cooperation with a product owner.
This gives the development team an opportunity to ask questions about the user stories.

\subsection{Skill groups}
In every group there are people with different responsibilities to facilitate the full-stack development style.
There is an at least one accountable person for frontend, backend and server per group.
The people with the responsibility for one of these areas have meetings with those accountable from the other groups to share knowledge and discuss problems that are encountered.
These groups are called skill groups.
Skill groups are suggested to have at least one weekly meeting.
