\section{Semester roles}
The groups for this semester acted as full stack teams, which was different from previous semesters. 
This meant that all of the groups had responsibilities related to frontend, backend and servers.
Previous years saw groups specializing in certain areas, so there was a frontend -, backend -, server -, scrum master - and product owner group. 
According to the previous years, this had not worked optimally, and they recommended that the groups became full stack teams to ensure all groups had knowledge in all areas.
Even though we decided to be full stack teams, we also decided that there should still be a dedicated process group and a product owner group. 
Every full stack group received user stories to implement in every sprint.

\subsection{Process Group}
The process group was responsible for facilitating the work process across all the other groups.
They arranged meetings and tried to optimize the meetings to avoid wasting time.
\\
There were some well defined tasks for the process group which were:
\begin{itemize}
    \item Deciding process changes during the semester
    \item Host the sprint planning meetings
    \item Facilitate stand up meetings for all groups
    \item Plan the first skill group meetings
    \item Host the sprint retrospective meetings
\end{itemize}
\noindent
The process group decided the initial processes for this semester based on feedback and discussions with all GIRAF members.
During the semester they evaluated the process model to investigate if the process worked as optimally as possible.
\\
15 minutes were allocated for stand up meetings, and it was the responsibility of the process group that they did not go over time.

\subsection{Product Owner Group}
Our role this semester was to function as product owners of the GIRAF project and to be in charge of customer contact. 
As product owners, we strove to have interviews with the customers at the end of every sprint to get feedback and conduct usability tests on the product.
The primary goal for our group was to maximize the value of the application for the customers. 
This was done by prioritizing user stories so that the customers got as much value as possible \autocite{TheScrumGuide}.\\
\\
There were some well defined tasks for the product owner group, which were:
\begin{itemize}
    \item Interview customers
    \item Create user stories    
    \item Refine the backlog regularly and prioritize user stories
    \item Create prototypes
    \item Create a sprint vision and sprint goals
    \item Ensure that the development teams understand the user stories
    \item Approve or decline features made by the development teams
    \item Conduct usability tests
\end{itemize}
\noindent
The interviews were the foundation for understanding the requirements of the application and were used to create user stories. 
Prototypes were created for the user stories.
These prototypes were used to create an initial design for the new features that the user stories might include. 
This was done so that we had a visual presentation of the user stories which made it easier for us to communicate with the customers and make sure that we understood each other. 
This way we could get a confirmation that the design ideas we had been working on satisfied the needs of the customer.
When the prototypes were approved by the customer they were added to the user stories in the backlog, so that the development teams could use them as a frame of reference when implementing them.
\\\\
At the start of every sprint we defined a vision for the sprint, defined sprint goals and updated the backlog. 
The vision and sprint goals were made to ensure that every team knew what they were working towards, and to motivate them to deliver a good product.
When a sprint ended, a usability test was conducted on the newest release to test new features and to get feedback in order to determine what should be developed next, and whether or not the new functionality was acceptable.


