\chapter{Introduction}
Autism spectrum disorder (ASD) is a condition that is characterized by a broad range of challenges within different areas such as social skills, speech and nonverbal communication, or by causing repetitive behavior.
In 2014 there were 16.8 occurrences of the ASD diagnosis per 1,000 children, and approximately 1\% of all Danes have an ASD diagnosis\autocite{cdcdata}. 
As ASD is a spectrum disorder, each person diagnosed with it has different strengths and challenges.
This results in people with ASD learning, thinking and solving problems very differently, with ranges from highly skilled and functional to severely challenged. 
Some may require support in their daily lives while others on the spectrum can live entirely independently\autocite{autismspeaks}.

\section{About GIRAF}
GIRAF (Graphical Interface Resource for Autistic Folk) is an ongoing project developed by 6th semester software engineering students at Aalborg University. 
The project has been continuously developed on since 2011 with Ulrik Mathias Nyman as project coordinator, with the new students assuming responsibility and learning to cooperate in a bigger environment with an existing codebase. 
\\
GIRAF is a program that serves the purpose of helping people with autism, with the primary user group being children.
The primary goal of the system is to provide visual representation of the daily or weekly schedule for the users.
During the lifetime of the project, different types of games and communication tools to help with education have been implemented, but most of these functionalities do not work after the API rework of 2017. The current focus of the GIRAF project is to make the weekplanner stable and fit for use, before resuming work on the other parts of the project. 
\\\\
A special aspect of the project, in comparison to previous projects, is the direct interaction with real customers, who are essential for the project.
The customers serve to define requirements of the program and facilitate the familiarization of students with industry processes.
\\
\noindent
Currently the institutions that are represented are: 
\begin{itemize}
    \item Mette and Emil, Egebakken (School)
    \item Kristine and Susanne, Birken (Kindergarten)
    \item Flemming, Center for Autism
    \item Niels, IT manager in the elderly and disability administration.
\end{itemize}

\section{State of Giraf}
The purpose of this section is to describe the current state of Giraf as it is delivered to us. 
This is done so that we have an idea of what our starting point will be,
but also to learn about what work was done on the project last year and what work was left undone. 
\\\\
Previously a lot of the backend had been rewritten which meant that many of the apps are no longer working correctly. 
This meant that last year's focus was to atleast get the weekplanner working again as a minimal viable product.
\\\\
At the moment the weekplanner is not very stable or responsive, and it is still missing some convenient functionalities. 
The login functionality was tested on a tablet a couple of times and it did not seem to work. This could have something to do with the version of the tablet.
Instead of accepting the login information, the app is loading for a long time until the user eventually gets a message that a problem has occurred. 
Therefore functionality can not be tested further on the tablet at the moment. Also it is not very clear to the user that the app is processing the login information. 
There is not implemented any loading spinner to indicate that the user must wait. Instead it seems more like the app has frozen when it is loading.
On the phone we are able to succesfully login and then choose a citizen from the 'choose citizen' page.
\\\\
Functionality in the weekplanner:
\begin{itemize}
    \item \textbf{Choosing a weekplan:} After choosing a citizen there is funcitonality that allows the guardian to choose an already existing weekplan for that citizen. 
    On this page the guardian can also choose to create a new weekplan.
    \\
    \item \textbf{Weekplan overview:} After choosing an existing weekplan or choosing to create a new weekplan, the user is redirected to the weekplan overview. 
    Here it is possible to see all of the days of the week and what activities that are planned for these days. There are switches that allows the guardian to delete a weekday plan, and buttons that allow the guardian to add activities to a weekday. 
    A slidebar functionality is added when there are too many activies on a weekday to be able to show on the weekplan.
    \\
    \item \textbf{Creating a new weekplan:} When the guardian chooses to create a new weekplan, the app redirects to an input page. 
    Here the guardian can enter a name for this specific weekplan, choose the year and week for the weekplan and also choose a pictogram to represent the weekplan. 
    Finally the guardian can choose to create a fresh weekplan or use an already existing template to build it. The functionality used for creating a weekplan seems working correctly.
    \\
    \item \textbf{Creating a new template:} As mentioned before a template can be used when creating a new weekplan. 
    On this page the guardian can choose create a new template, and this basically works the same way as creating a weekplan. 
    Creating a template and then saving it with a given name seemed to be working fine.
    \\
    \item \textbf{Deleting a weekday:} When viewing a weekplan, the guardian can choose to delete one of the weekdays. Each weekday has a switch to allow the guardian to delete it.
    When this switch is pressed, a window pops up to ask the guardian if they are sure they want to delete the weekplan to which the guardian can answer yes or no. 
    The pop up asking if the guardian is sure they want to delete the weekday is a bit misleading, because the functionality works more like a hide functionality that hides the weekday from the view. 
    The weekday is being deleted as the app says, because the guardian can press the switch for the weekday again to make it visible again.
    \\
    \item \textbf{Saving a weekplan:} The guardian has the ability to save changes made to a weekplan. A change can be for example adding a new activity to a weekday.
    A button can then be pressed to save the changes made to the weekplan, and this functionality seems to be working fine.
    Another good functionality that has been implemented is that it alerts the user if they leave the weekplan overview with unsaved changed. The user then gets a last opportunity to save the changes they made.
    \\
    \item \textbf{Switching from guardian to citizen:} In the top bar there is an icon that allows the user to switch between guardian and citizen. 
    When this icon is pressed it is not very responsive but eventually it does switch the user to a different mode. 
    When it is switched to citizen, the weekplan view changes to view the current day's activities.
    \\
    \item \textbf{Switching from citizen to guardian:} When clicking the icon again to switch back to guardian mode, the app sometimes does not respond very well. 
    Eventually the user is redirected to the login page to login as a guardian again.
    \\
    \item \textbf{Adding an activity:} A guardian can add activities to the weekdays in a weekplan. This is done by pressing an "Add" button at the bottom of a weekday the guardian wants to add an activity to.
    This action will redirect the guardian to a page where they can search for a pictogram that resembles the activity they wish to add. The searches on this page sometimes give weird results, but overall it works okay.
    One problem on this page is that the guardian is not able to see pictograms during the search, but instead the whole screen is taken up by the keyboard. 
    Other than that, the functionality to add activities to a weekday works fine.
    \\
    \item \textbf{Interacting with activities from the weekplan overview:} After adding an activity to a weekday the user is able to interact with this activity. 
    First of all, activities can be dragged up an down to change the order of the activities for the day. 
    Secondly, the each activity can be interacted with by pressing it which takes the user to the activity's page. 
    Here the user can delete the activity or mark the activity as done. The user can the save the change to the activity which then returns them to the weekplan overview.
    \\
\end{itemize}

\section{Scrum of Scrums}
Scrum is a framework that is used extensively in software projects.
Its an agile approach to working with complex and changing problems where a normal waterfall model does not work optimally.
In this project we used Scrum of Scrums (SoS) to structure the groups across the whole GIRAF team.
SoS is a modification of Scrum made to scale it better for bigger teams.
Many of the activities are similar to normal Scrum.
The sprint process of SoS works in the following way:

\begin{itemize}
    \item Sprint Planning
    \item SoS Stand Up
    \item Skill Group Meetings
    \item Release Preparation
    \item Sprint Review
    \item Sprint Retrospective
    \item Release Party
\end{itemize}

\subsection{Sprint Planning}
Sprint planning is a meeting on the first day of a new sprint where all groups are expected to show.

Prior to the meeting the PO-group has made user stories based on communication with the customers.
The user stories will have prototypes, a definition of what is needed based of the view of a user and a technical description of what is expected to be coded.

The PO-group has also made relevant and realistic goals for the oncoming sprint, which should translate into a new release of the GIRAF software for the app store.
It is important that the goals are reachable to give the groups a sense of accomplishment.
This has been an issue in earlier years were groups did not feel that there was a clear improvement in the software which drastically reduced moral.

The meeting starts by the PO-group presenting, or refreshing, the goals they chose for the whole semester and then more specifically the goals they want fulfilled in the oncoming sprint.
Afterwards the groups will look at the user stories that is in the the backlog and ask clarifying question if needed.

The groups then choose a user story from the highest prioritized user stories.
Before the groups can begin working they have to get their choice approved by the PO-group.
When all the groups have been approved the sprint planning is over and development can begin.

\subsection{SoS Stand Up}
During a normal sprint week there will be at least one SoS Stand Up meeting.
All groups should send at least one person to these meetings but if necessary more can attend, though the goal should be to send as few as possible.
A Stand Up meeting takes at most 15 minutes, when 15 minutes has past the meeting ends no matter what.

During the meeting each group should present what they have, and are, working on and what they will work on until the next meeting.
They should also notify the others of what problems they had faced or are facing and if they are about to introduce something new that could affect other groups.
Each group representative takes turn to present, if there is time afterwards people may ask questions to the other groups else they have to talk after the meeting.


\subsection{Skill Group Meetings}
Based on advise from last years groups we chose to implement skill groups.
Compared to previous years where a whole group had a role such as frontend, backend or server, these responsibilities has now been spread out across groups so that each group has at least one person responsible for frontend, backend or server.



\subsection{Release Preparation}

\subsection{Sprint Review}

\subsection{Sprint Retrospective}

\subsection{Release Party}



\section{Technologies and Tools}
This section describes the technologies and tools that are used in this project. 
Some of them are used to facilitate the collaboration between all the groups in the GIRAF project while others are used internally in our group.

\textbf{Jira}\\
Jira is a software development tool developed by Atlassian and is used for agile software development.
The software facilitates the creation of a backlog of user stories that can then be assigned to a sprint.
The team can assign story points to each assignment and assign a user to the user story, to distribute the workload properly over the coming sprint.
Jira also includes multiple tools for managing and monitoring sprints and their progress, to help with retrospectives and to ensure the sprint is proceeding as planned.
We used it for our weekly sprints that were run internally in the group.
\\\\
\textbf{Adobe XD}\\
Adobe XD is a program for prototype creation, that is easy to pick up and create simple designs in.
Adobe XD makes it easy to reuse components in multiple design projects and to collaborate with others.
It also lets you assign functionality to the prototypes, meaning they can be used for usability testing with the users to demonstrate the functionality.
\\\\
\textbf{GitHub}\\
GitHub is a development platform that makes it possible for multiple people to collaborate on a project. 
All of the code in the GIRAF project is hosted on GitHub.
The issue and project features are used to create and assign user stories to the different groups that are working on the GIRAF project and to manage the sprints. 
The GIRAF wiki is also hosted on GitHub.
\\\\
\textbf{Slack}\\
Slack is a collaboration hub where users can create a workspace that they can invite their collaborators to.
It is possible to create multiple channels with independent communication. 
The collaborators can then choose which channels they want to join.
Slack has been used for all communication across the participating groups of the GIRAF project.

This section describes the technologies and tools that are used in this project. 
Some of them are used to facilitate the collaboration between all the groups in the GIRAF project while others are just used internally in our group.

\paragraph{Jira}~\\
Jira is a software development tool that is developed by Atlassian and is used for agile software development.
It is possible in Jira to create a backlog of user stories that can then be assigned to a sprint. The user can assign story points to each assignment and assign a user to the user story.
Jira also includes multiple tools for managing and monitoring sprints and their progress.
We used it for our weekly sprints that we had in our group.

\paragraph{Adobe XD}~\\
Adobe XD is a program developed by Adobe that used to create prototypes.
It is a simple program that is easy to pick up and create simple designs in.
Adobe XD makes it easy to reuse components in multiple design projects and to collaborate with others.

\paragraph{GitHub}~\\
GitHub is a development platform that makes it possible for multiple people to collaborate on a project. 
All of the code in the GIRAF project is hosted on GitHub.
The issue and project features are used to create and assign user stories to the different groups that are working on the GIRAF project and to manage the sprints. 
The GIRAF wiki is also hosted on GitHub.

\paragraph{Slack}~\\
Slack is a Collaboration hub where users can create a workspace that they can invite their collaborators to.
It is then possible to create multiple channels. The collaborators can then choose which channel they want to join.
Slack has been used for all communication across the GIRAF project.

\section{Prepared work from previous years}\label{prepared-work-from-previous-years}
Before the first semester wide sprint could start, our group had some work to do as preparation hereof.
First of all, during the readthrough of the reports from previous years, it was discovered that the PO group of 2018 had left us a suggestion for content in the first sprint:

\subsection{Sprint 1}
The first sprint that the previous PO group suggests includes two user stories:
The first user story presents the need for a guardian to be able to mark multiple activities and perform actions on these activities at the same time.
\\\\
The second user story is about a user being able to change the way that an activity is marked as being complete.
This could, for example, be represented by a checkmark, by hiding the activity or by moving the activity a bit to the right on the schedule for the day.
These user stories are suggested for the first sprint because it should be easy for the developers during the start of the project, as they are not familiar with the codebase yet.

\subsection{Sprint 2}
For the second sprint, the previous product owners suggest two user stories, where the first one is that a user should be able to time activities with a timer.
This is needed so the citizens know how much time their activities take.
The guardian should be able to add this timer with a specific time and connect it to the activity, after which the citizen should then be the one that starts the timer.
\\\\
The second user story concerns a feature for guardians that allows them to choose between a set of visual representations for the timers that they can add for the activities.
This is needed because the citizens have different preferences when it comes to representation of the time.
\\\\
Just as the first suggested sprint, this suggested sprint is meant to be light in workload to allow the developers to get familiar with the codebase, and to make sure that all groups can finish their tasks for the sprint.

\subsection{Design guide}
A design guide was available, made by the previous years.
However, this guide seemed to be last updated in 2015 and seemed to never have been properly used for implementation.
So in order to ensure that the guide is up to date, we decided to start working on a renewed version of the design guide, which should be available in the github wiki instead of as a separate pdf document.
The changes that are being imagined for the new design guide are, first of all, a set of rules for ensuring the user experience of the application by updating the icons, and to make the application seem less like a special needs tool, as this has been requested in the interview with Emil.

\subsection{Producing Prototypes in Adobe XD}
In addition to updating the design guide, we decided to update the prototypes to a more suitable program type.
The currently available prototypes are made by putting images into a PowerPoint presentation and making clickable areas to navigate through them.
This has resulted in prototype consisting of 122 slides, which could not be edited by other means than replacing a given element in every single slide.
By changing this to Adobe XD, it is possible to mark a part of the prototype as a symbol, and by changing this symbol in one place it will be replaced in all aspects of the prototype.
This allows for easier updates of the design in comparison to the prototypes made in PowerPoint.

