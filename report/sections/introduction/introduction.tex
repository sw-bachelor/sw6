\chapter{Introduction}
Autism spectrum disorder (ASD) is a condition that is characterized by a broad range of challenges within different areas such as social skills, speech and nonverbal communication, or by causing repetitive behavior.
In 2014 there were 16.8 occurrences of the ASD diagnosis per 1,000 children, and approximately 1\% of all Danes have an ASD diagnosis\autocite{cdcdata}. 
As ASD is a spectrum disorder, each person diagnosed with it has different strengths and challenges.
This results in people with ASD learning, thinking and solving problems very differently, with ranges from highly skilled and functional to severely challenged. 
Some may require support in their daily lives while others on the spectrum can live entirely independently\autocite{autismspeaks}.

\section{About GIRAF}
GIRAF (Graphical Interface Resource for Autistic Folk) is an ongoing project developed by 6th semester software engineering students at Aalborg University. 
The project has been continuously developed on since 2011 with Ulrik Mathias Nyman as project coordinator, with the new students assuming responsibility and learning to cooperate in a bigger environment with an existing codebase. 
GIRAF is a program that serves the purpose of helping people with autism, with the primary user group being children.
Its primary goal of the system is to provide visual representation of the daily or weekly schedule for the users.
During the lifetime of the project, different types of games and communication tools to help with education have been implemented, but most of these functionalities do not work after the API rework of 2017. The current focus of the GIRAF project is to make the weekplanner stable and fit for use, before resuming work on the other parts of the project. 
\\
A special aspect of the project, in comparison to previous projects, is the direct interaction with real customers, who are essential for the project.
The customers serve to define requirements of the program and facilitate the familiarization of students with industry processes.

Currently the institutions that are represented are: 
\begin{itemize}
    \item Mette and Emil, Egebakken (School)
    \item Kristine and Susanne, Birken (Kindergarten)
    \item Flemming, Center for Autism
    \item Niels, IT manager in the elderly and disability administration.
\end{itemize}

\subsection{State of Giraf - February 2019}

\section{Scrum}
This semester scrum is used as the project management process. 
The scrum master group created the \textit{GIRAF Process Manual} and this section describes the most important things in this manual.

\todo{Hvordan henviser vi til GIRAF Process Manualen?}

\subsection{Scrum of Scrums}
The previous semesters suggest that Scrum of Scrums is used as a project management process. 
Scrum of scrums is used to scale scrum to be used for bigger teams.

\subsection{Sprints}
This subsection focuses on how the initial process planning of sprints are conducted. 

The stucture of a sprint is:
\begin{enumerate}
    \item Sprint planning
    \item Scrum of Scrums Stand ups
    \item Skill group meetings
    \item Release Preparation
    \item Sprint Review
    \item Sprint Retrospective
    \item Release Party
\end{enumerate}

\subsubsection{Sprint planning}
\subsubsection{Scrum of scrums stand up}
\subsubsection{Skill group meetings}
\subsubsection{Release preparation}
\subsubsection{Sprint review}
\subsubsection{Sprint retrospective}
\subsubsection{Release party}

\subsection{Sprint scheduling}
These are 4 intial sprints this semester:
\begin{itemize}
    \item Sprint 1: 25/2 - 18/3, 3 weeks
    \item Sprint 2: 18/3 - 8/4, 3 weeks
    \item Sprint 3: 8/4 - 29/4, 3 weeks
    \item Sprint 4: 29/4 - 13/5, 2 weeks
\end{itemize}
\noindent
The last 4 days before a sprint ends is used to prepare a working release build. 
The release then have to be approved by product owners.
The remaining time before project delivery will be used to write on the report and to prepare information and material to future semesters. 

\subsection{Semester roles}
Different from previous semesters, this semester the groups are fullstack teams. 
This means that all groups have responsibility in frontend, backend and servers.
Previous semesters there had been a frontend -, backend -, server -, scrum master - and product owner group. 
According to the previous semesters, this had not worked optimally, and they recommended that the groups became fullstack teams.
Even though that they are fullstack teams, it has been decided that there will still be a scrum master - and product owner group. 
The fullstack groups will get user stories to implement at every sprint.

\subsection{Skill groups}
In every fullstack group there are people with different responsibilities.
There are an accountable person for frontend, backend and server.
The people with the responsibility with one of these areas will have meetings with the accountable from the other groups to share knowledge and discuss problems that have been encountered.
These groups are called skill groups.
Skill groups are suggested to have at least one weekly meeting.

\subsection{Our role}
This semester our role is to function as product owner and be in contact with the customers.
As product owners we strive to have interviews with customers at every sprint to get feedback on the product.
\\
There are some well defined tasks for the product owner group, which are:
\begin{itemize}
    \item Interview customers
    \item Create user stories    
    \item Refine backlog regularly and prioritise user stories
    \item Create prototypes
    \item Create sprint vision and sprint goals
    \item Ensure that the fullstack teams understand the user stories
    \item Conduct usability tests
\end{itemize}
\noindent
The interviews are used to create user stories. 
From user stories prototypes are created, which are later shown to the customers.
When the prototypes are approved they are added to the user stories in the backlog.
\\
At every sprint we make a vision for the sprint, sprint goals and update the backlog. 
The vision and sprint goal is made to ensure that every team knows what they are working towards.
When a sprints ends a usability test is conducted to test new features and get feedback on it.

\section{Technologies and Tools}
This section describes the technologies and tools that are used in this project. 
Some of them are used to facilitate the collaboration between all the groups in the GIRAF project while others are used internally in our group.
\\\\
\textbf{Jira}\\
Jira is a software development tool developed by Atlassian and is used for agile software development.
The software facilitates the creation of a backlog of user stories that can then be assigned to a sprint.
The team can assign story points to each assignment and assign a user to the user story, to distribute the workload properly over the coming sprint.
Jira also includes multiple tools for managing and monitoring sprints and their progress, to help with retrospectives and to ensure the sprint is proceeding as planned.
We used it for our weekly sprints that were run internally in the group.
\\\\
\textbf{Adobe XD}\\
Adobe XD is a program for prototype creation, that is easy to pick up and create simple designs in.
Adobe XD makes it easy to reuse components in multiple design projects and to collaborate with others.
It also lets you assign functionality to the prototypes, meaning they can be used for usability testing with the users to demonstrate the functionality.
\\\\
\textbf{GitHub}\\
GitHub is a development platform that makes it possible for multiple people to collaborate on a project. 
All of the code in the GIRAF project is hosted on GitHub.
The issue and project features are used to create and assign user stories to the different groups that are working on the GIRAF project and to manage the sprints. 
The GIRAF wiki is also hosted on GitHub.
\\\\
\textbf{Slack}\\
Slack is a collaboration hub where users can create a workspace that they can invite their collaborators to.
It is possible to create multiple channels with independent communication. 
The collaborators can then choose which channels they want to join.
Slack has been used for all communication across the participating groups of the GIRAF project.

This section describes the technologies and tools that are used in this project. 
Some of them are used to facilitate the collaboration between all the groups in the GIRAF project while others are just used internally in our group.

\paragraph{Jira}~\\
Jira is a software development tool that is developed by Atlassian and is used for agile software development.
It is possible in Jira to create a backlog of user stories that can then be assigned to a sprint. The user can assign story points to each assignment and assign a user to the user story.
Jira also includes multiple tools for managing and monitoring sprints and their progress.
We used it for our weekly sprints that we had in our group.

\paragraph{Adobe XD}~\\
Adobe XD is a program developed by Adobe that used to create prototypes.
It is a simple program that is easy to pick up and create simple designs in.
Adobe XD makes it easy to reuse components in multiple design projects and to collaborate with others.

\paragraph{GitHub}~\\
GitHub is a development platform that makes it possible for multiple people to collaborate on a project. 
All of the code in the GIRAF project is hosted on GitHub.
The issue and project features are used to create and assign user stories to the different groups that are working on the GIRAF project and to manage the sprints. 
The GIRAF wiki is also hosted on GitHub.

\paragraph{Slack}~\\
Slack is a Collaboration hub where users can create a workspace that they can invite their collaborators to.
It is then possible to create multiple channels. The collaborators can then choose which channel they want to join.
Slack has been used for all communication across the GIRAF project.

\section{Prepared work from previous years}\label{prepared-work-from-previous-years}
Before the first semester wide sprint could start, our group had some work to do as preparation hereof.
First of all, during the readthrough of the reports from previous years, it was discovered that the PO group of 2018 had left us a suggestion for content in the first sprint.
They had prepared prototypes to illustrate the vision of the project.
A design guide detailing the foundation of design for GIRAF was also available. 

\subsubsection{Suggestion for sprint 1}
The first sprint that the previous PO group suggests includes two user stories.
The first user story presents the need for a guardian to be able to mark multiple activities and perform actions on these activities at the same time.
\\\\
The second user story is about a user being able to change the way that an activity is marked as being complete.
This could, for example, be represented by a checkmark, by hiding the activity or by moving the activity a bit to the right on the schedule for the day.
These user stories are suggested for the first sprint because it should be easy for the developers during the start of the project, as they are not familiar with the codebase yet.

\subsubsection{Suggestion for sprint 2}
For the second sprint, the previous product owners suggest another two user stories.
The first one is that a user should be able to time activities with a timer.
This is needed so the citizens know how much time their activities take.
The guardian should be able to add this timer with a specific duration and connect it to the activity, after which the citizen should then be the one that starts the timer.
\\\\
The second user story concerns a feature for guardians that allows them to choose between a set of visual representations for the timers that they can add to the activities.
This is needed because the citizens have different preferences when it comes to the representation of the time.
\\\\
Just as the first suggested sprint, this suggested sprint is meant to be light in workload to allow the developers to get familiar with the codebase, and to make sure that all groups can finish their tasks for the sprint.

\subsubsection{Design guide}
A design guide was available, made by an earlier GIRAF group.
However, this guide seemed to be last updated in 2015 and seemed to never have been properly used for implementation.
This design guide was located on a decentralized location, meaning not many of the developers were even aware of its existence.
It was also around 80 pages long, meaning it could be quite intimidating to search through and to follow.

\subsubsection{Prototypes}
A set of prototypes was constructed by the previous year.
The available prototypes were made by putting images into a PowerPoint presentation and making clickable areas to navigate through them.
This resulted in a single set of prototypes consisting of 122 slides, which could not be edited by other means than replacing a given element in every single slide, even for the slides with very little variation.


\subsubsection{Conclusion on the prepared work}
Two user stories for each sprint is not enough work for 7 groups. 
As previous years had dedicated backend, frontend and server groups, a user story often had to be split into tasks for multiple groups.
The user stories are not large enough to be given to multiple groups.
This year we decided to work in full stack groups, and because of this a group is able to take one or more user stories for each sprint and then only work on the assigned user stories.
GIRAF in 2019 will also be mostly focused on frontend development which the previous year was not.
The user stories that were prepared are still useful and they have been put into the backlog along with others, but they will not be the foundation of the starting sprint.
The design guide is not entirely up to date, not completely finished and too long to effectively use.
As such, it is not all that useful for the GIRAF project in 2019.
The prototypes made by the previous PO group are functional, but their design and usability could be improved. 



