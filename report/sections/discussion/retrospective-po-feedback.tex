\section{Feedback from final retrospective}\label{sec:retrospective-discussion}
As mentioned in \autoref{sec:sprint-4-retrospective}, a large part of the comments from the retrospective were PO related, and as such we will discuss them in this section.
There were more comments in the retrospective, but we only reflect on the ones that were PO related.
\\\\
\textit{"Next years PO group should do stories in a story format instead of just a title. Sometimes it was hard to agree how a user story should be made."}
\\
18 agree, 7 do not care, 5 disagree.
\\
This suggestion is a good idea. 
This should however be an addition to the user story title and an addition to the prototypes. 
The story format would be describing how a user would complete the user story through text. 
Doing this would eliminate some of the ambiguity that can occur in user stories.
\\\\
\textit{"The tasks were difficult to understand directly from the description"}
\\
7 agree, 6 do not care, 17 disagree.
\\
There does not seem to be a lot of developers who had difficulties understanding the tasks directly from the description. 
Including a story format in the user stories would make the tasks easier to understand.
However, as we were always available in the group rooms, we had always recommended that they could approach us for clarification if something was ambiguous or difficult to understand. 
\\\\
\textit{"It would be nice if there were smaller task that you could take without contacting PO"}
\\
10 agree, 11 do not care, 9 disagree.
\\
This is something that we did not recommend implementing. 
It was required for us to understand what was being implemented. 
If people assigned themselves to tasks, we would have a more difficult time understanding how far we were in the process.
There is also the possibility that someone could forget to assign themselves to the tasks, and then the story might be delegated to another group, meaning multiple groups could end up working on the same task.
Currently, if someone forgot to assign themselves, we had the possibility to discover the error before another person was assigned to the task.
\\\\
\textit{"Task delegation was fine"}
\\
20 agree, 9 do not care, 1 disagrees.
\\\\
\textit{"Good understanding of sprint 4, which made the sprint good. The final process was good. Delegation of user stories was good to give it purpose."}
\\
21 agree, 7 do not care, 2 disagree.
\\
It was surprising for us that so many people were satisfied being delegated tasks. 
At the start of the project, many groups expressed concern that they would end up implementing user stories they did not want to, which led to the process of groups deciding for themselves from a selection of user stories.
We chose to delegate the tasks with the purpose of trying to achieve the minimal viable product, and because of the lack of a negative reaction we proposed to the PO group of next year to do the same for the final sprint in our PO advice documentation.
\\\\
\textit{"PO decides how the product should be, but the developers decide how it should be implemented"}
\\
27 agree, 3 do not care, 0 disagree.
\\
This is something that most developers agreed with, but this is also how the process worked this year so the purpose of this suggestion is unclear.
If the suggestion actually refers to the user interface of the application, wanting developers to have free reign, we disagree heavily.
The user interface was based on our prototypes, which were all shown to the customer for feedback. 
If the developers did not have these prototypes to work off of, it would likely mean that some functionality would be implemented incorrectly, or not implemented at all.
\\\\
\textit{"It is irrelevant to mention which files you are working on in the standup meetings. The user stories should be planned better to avoid merge conflicts."}
\\
9 agree, 10 do not care, 11 disagree.
\\
This partly conflicts with the previous comment. 
If we were to plan the user stories and see which files were needed for each user story, then we would need to decide where and how the developer should be implementing their user story. 
This would be very time consuming for the PO group, but also result in a worse product, as the developers would likely be more knowledgeable about how to implement the user stories.
We think that it was much better to communicate which files you were working on in the stand up meetings to avoid merge conflicts, and for the groups to discover dependencies.
Additionally, the stand up meetings were rather short, so this suggestion was not based on a lack of time to communicate during these.
Finally, for GIRAF 2019, 7 groups worked together on the weekplanner application, meaning some overlap would likely be unavoidable.

