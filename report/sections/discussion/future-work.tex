\section{Suggested upcoming features}
In this section we propose suggestions for future work. 
The complete list of user stories, tasks and bug reports can be found in \autoref{appendix:future-work-user-stories}.
In the following sections we go through the user stories prioritized as highest and high and explain why they are prioritized so highly, and what the customers gain from these user stories.
Addtionally, the suggested first sprint is explained in this section.
User stories of medium, low and lowest prioriy will not be described in this section.

\subsection{Highest priority tasks}\label{highest-priority-tasks}
These are the tasks of highest priority for the GIRAF project that would add the most value to the customers.

\begin{itemize}
    \item Weekplanner\#66 As a guardian I would like to be able to add a new citizen, so that the system is easily set up when a new citizen starts at our institution
    \item Weekplanner\#200 As a guardian I would like for the week plans to be sorted consistently every time I look at the week plans for a specific citizen
    \item Weekplanner\#220 As a guardian I would like to be able to create a choice board so that I can create an activity where the citizen can make a choice between multiple activities 
    \item Weekplanner\#233 Week names split into multiple lines if the display resolution is not high enough 
\end{itemize}

\noindent
Weekplanner\#66 is a key feature to have implemented. 
As of now it is not possible to add new citizens, which makes it difficult to use the system. 
\\\\
For Weekplanner\#200 the purpose is to save time when guardians are operating the system. 
The list of week plans is currently not sorted, so whenever they load the screen for choosing a citizens week plan, the different weeks show up in a random order. 
When this application has been in use for some time and a lot of week plans have been added, it will become impossible to navigate for the guardians without spending a long time finding the correct week plan.
Currently you can solve the problem by deleting week plans, but this should not be necessary to use the program.
\\\\
For Weekplanner\#220 the purpose is to support the teaching of decision making to the children.
According to Emil, the children often have problems with making decisions and can get stuck on very simple ones. 
An example could be that a child might not be able to decide if they want to do a puzzle or paint.
This feature would be really helpful in helping the children make decisions.
\\\\
Weekplanner\#233 is a bug that is important to fix.
The citizens are often not able to read "Mandag" if it is written on two lines. 
This is due to them recognizing the entire word and not recognizing the individual letters. 
If the word is split onto two lines, then it can be incomprehensible for them.
This is something that Emil pointed out, pointing out that this bug is important to fix.

\subsection{High prioritized tasks}
These user stories are prioritized as high. 
This means that these user stories are very important, but do not give the same amount of value to the customer as the highest prioritized user stories.
\begin{itemize}
    \item Weekplanner\#10 As a citizen I would like to disable colors in the app so that it does not overstimulate me
    \item Weekplanner\#15 As a guardian I would like to be able to choose how many days a citizen can see at a time on their weekplanner, so that it fits their personal preference 
    \item Weekplanner\#62 As a guardian I would like to be able to remove a timer from an activity so that the activity is no longer time specific
    \item Weekplanner\#162 As a guardian I would like to be able to setup a template week so that I don't have to duplicate the same weekplan  
    \item Weekplanner\#177 As a guardian I would like the week plans on "vælg ugeplan" screen to show the week and year of each plan so that I can easily find the right plan 
    \item Weekplanner\#221 As a guardian I would like to be able to lock a citizens timer so that they can not pause or stop them once they are started
    \item Weekplanner\#227 As a guardian, I would like the search for pictograms to be ordered by how popular a pictogram is, so that I can find the most commonly used pictogram quickly
\end{itemize}
\noindent
Weekplanner\#10 gives some citzens a better experience when looking at the weekplanner. 
Some citizens will get overwelmed by the amount of colours in the application and can get confused by them. 
This function is however only needed for a few citizens, but the application would be less useful for them if they are unable to understand the pictograms due to having too many colors, or avoid them being overwhelmed by the different colors used for the specific days.
\\\\
Weekplanner\#15 is linked very closely with Weekplanner\#10. 
It is the same reasoning behind this user story, as some citizens will be overwelmed by too many days with a lot of pictograms. 
They will be unable to relate to the pictograms and understand which day it is.
As the citizens grow older and get used to the system, the guardians would usually give them more days to look at.
\\\\
For Weekplanner\#62 it is important to be able to remove timers from activities. 
Not all activities need a timer, and especially activities such as going to the bathroom or eating should not be timed, as it is necessary that they complete the tasks, and the amount of time that they spend on it does not matter.
\\\\
The purpose of Weekplanner\#162 is to save time for the guardians. 
Most days have some things in common for the citizens, and therefore it is a waste of time if the guardians have to make the same week plan over and over again.
Solving this would give the guardians more time to engage with the citizens.
\\\\
Weekplanner\#177 has the same purpose as Weekplanner\#200 in \autoref{highest-priority-tasks}. 
The reason why Weekplanner\#117 is still needed is if the guardians forget whether or not they have created a new week plan.
If so, it would be easier for them to see if the week plan has been created.
Addtionally, it is easier for them to find a specific week plan from a previous week.
\\\\
Weekplanner\#221 is something that Emil requested. 
For some activities it makes sense for the citizens to be able to pause an activity, but for some citizens it would be overwelming.
Time is also something that the citizens commonly have problems with. 
Giving them the possibily to pause the timer would be very confusing for some of them, so the option to not let them should be present.
\\\\
For Weekplanner\#227 it is important for the guardians to be consistent with the pictograms.
Emil said that they usually do not use more than 50 pictograms for all acitivities. 
To make the search function better, he suggested that it could be sorted by the popularity of the pictograms, so that the guardians tend to pick the same pictograms.
The reason why it is important to use the same pictograms is because the citizens have a specific understanding of the pictograms.

\subsection{Sprint 2020S1}
For sprint 1 in 2020 we suggest that the following user stories are being worked on:

\begin{itemize}
    \item Weekplanner\#15 As a guardian I would like to be able to choose how many days a citizen can see at a time on their weekplanner, so that it fits their personal preference 
    \item Weekplanner\#62 As a guardian I would like to be able to remove a timer from an activity so that the activity is no longer time specific
    \item Weekplanner\#177 As a guardian I would like the week plans on "vælg ugeplan" screen to show the week and year of each plan so that I can easily find the right plan 
    \item Weekplanner\#227 As a guardian, I would like the search for pictograms to be ordered by how popular a pictogram is, so that I can find the most commonly used pictogram quickly
    \item Weekplanner\#220 As a guardian I would like to be able to create a choice board so that I can create an activity where the citizen can make a choice between multiple activities 
    \item Weekplanner\#233 Week names split into multiple lines if the display resolution is not high enough 
\end{itemize}
\noindent
All of the mentioned user stories are either prioritized as highest or high. 
They are considered for the first sprint both because they are important features to have implemented, but also because they are the easier tasks to get started on.
These tasks require an understanding of the giraf project, but are good first issues to get started with the code base.
