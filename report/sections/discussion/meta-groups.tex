\section{GIRAF and meta groups}
GIRAF 2019 differed from previous years in that it was the first to employ meta groups.
Meta groups were introduced to facilitate full stack development, meaning all development groups should have knowledge in every area of the program.
A person from each group would be appointed as the one primarily responsible for an area of expertise, and one would be appointed as an alternate in case of sickness.
There was a meta group dedicated to each of the following areas:
\begin{itemize}
    \item Front end
    \item Back end
    \item Server
\end{itemize} 
\noindent
Initially, all groups seemed to work well. 
Meetings were organized regularly, and they were important to keep up with.
However, some of the meta groups lost their importance as the project progressed.
After the decision was made to change to Flutter, and the design guide had been discussed, the front end group had trouble remaining relevant.
This was most likely due to the purpose of the group being poorly defined.
For the semester, the priority of the front end group had been purely design.
This proved to make the group irrelevant in the long term.
In order to better take advantage of the front end group, a better purpose should be defined.
Instead of simply making decisions relating to design, the front end group should have focused more on implementation details in Flutter as well.
This would serve to keep the group relevant, and provide a good source of knowledge for developers.
\\\\
The server meta group encountered a different problem.
As the project progressed, it became more insular with fewer developers feeling that they could contribute.
The reason for this was hard to define.
It could be caused by the GIRAF project lacking developers with an interest in the subject.
It could also be caused by a smaller, more passionate group of developers wanting to focus on this aspect, and the rest being content with them leading the group.
In order to ensure that the knowledge is spread more evenly, an effort should be made within the group to catch everyone up.
\\\\
The back end group was the one that fulfilled its purpose the best.
It continually held meetings throughout the duration of the project, knowledge was spread fairly evenly between participants, and it did not become largely irrelevant.
The back end group also organized a hackathon during development, in which the group got together for a full day to complete tasks related to the back end.
This worked well, and should be considered for other groups if this concept was to be used again.
\\\\
Even though our usage of meta groups might have had some issues, we still believed it to be superior to the alternative of strictly delegating areas of expertise to different groups.
When a group is capable of implementing a full user story without it needing to be divided into parts, it can lead to a more cohesive application.
At the same time, being familiar with all areas helps to create a feeling of responsibility and ownership for the application in all developers, leading to a better product.
