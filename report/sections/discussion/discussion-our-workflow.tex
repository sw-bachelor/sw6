\section{Our process}
In the start of the semester we discussed how the process should work in our group.
We ended up doing it the way we described in \autoref{intro:the-process-in-our-group}.
We wanted to use something based on Scrum, as we had used it in the last couple of semesters and because the GIRAF project was going to use the GIRAF version of Scrum as described in \autoref{the-giraf-process}.
\\\\
One of our first decisions was that the length of our sprints should only be one week compared to the sprints of the GIRAF project which were two to four weeks long.
One of the counter arguments was that it would be easier to keep track of them if we aligned our sprints with the GIRAF project, but unlike the GIRAF project our sprints were not planned to such an extent that we could have enough tasks in the backlog for a four week sprint.
Often, we would finish the planned content of a sprint before the deadline.
When this happened, we would include new user stories from the backlog in the sprint backlog. 
This worked well for us, as we generally tried to start with small sprint backlogs, and gradually increased them, so that we would still be sure that we could finish the sprint before the deadline.
We also had weekly meetings with our supervisor, therefore scheduling the sprints around this worked well.
Our sprints would end the day we had to send our report to our supervisor and a new sprint would start.
The new sprint always included a task to correct feedback from the supervisor, and that someone had to act as internal product owner.
\\\\
One thing that we would have liked to change was that not everyone got the opportunity to implement user stories.
What initially was planned was that at every sprint two people would implement user stories.
As we did not expect to implement user stories in sprint 1, we expected everyone to have the possibility of implementing user stories in the later sprints.
When we reached sprint 4 we realized that we did not have enough time to set two people without much Flutter experience aside to implement user stories, if we wanted to create comprehensive documentation for the next generation.
If we could, we would have changed the process of assigning development tasks within our group.
We could have assigned more people to implement the same user stories, so that more people were knowledgeable about how our user stories were implemented. 
Multiple people would also have enough knowledge to answer questions from the other groups about the implementations.
\\\\
We decided to not have a daily stand up meeting, as everyone would meet in our group room everyday and progress would naturally be discussed amongst our group members.
Therefore, everyone would know how far along the other group members were with their tasks, and if anything was blocked it could quickly be resolved without arranging an extra meeting.
It would be interesting to attempt to have formal daily stand up meetings in the future, to see how it affects the productivity of the group, as we suspect that knowing that the other people are working on something may inspire them to also get started on their tasks.
\\\\
The product owner role rotated every week between group members which worked better for our situation. 
It gave everyone a chance to decide how they wanted our project to proceed and decide the priority of tasks for the sprint.
Likewise, allowing everyone to add new tasks, we ensured that everyone had a say in what should be included in the project and report.
