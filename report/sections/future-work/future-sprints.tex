\section{Suggested upcoming features}\label{SuggestedUpcomingFeatures}
In this section we propose suggestions for future work. 
The complete list of user stories, tasks and bug reports can be found in \autoref{appendix:future-work-user-stories}.
In the following sections we go through the user stories prioritized as highest and high and explain why they are prioritized so highly, and what the customers gain from these user stories.

\subsection{Highest priority tasks}\label{highest-priority-tasks}
These are the tasks of highest priority for the GIRAF project that would add the most value for the customers.

\begin{itemize}
    \item Weekplanner\#66 As a guardian I would like to be able to add a new citizen, so that the system is easily set up when a new citizen starts at our institution
    \item Weekplanner\#200 As a guardian I would like for the week plans to be sorted consistently every time I look at the week plans for a specific citizen
    \item Weekplanner\#220 As a guardian I would like to be able to create a choice board so that I can create an activity where the citizen can make a choice between multiple activities 
    \item Weekplanner\#233 Week names split into multiple lines if the display resolution is not high enough 
    \item Weekplanner\#245 Show error on pictogram upload failed
\end{itemize}

\noindent
\textit{Weekplanner\#66} is a key feature to have implemented. 
As of now it is not possible to add new citizens through the application, which makes it difficult to use the application. 
\\\\
For \textit{Weekplanner\#200} the purpose is to save time when guardians are using the application. 
The list of week plans is currently not sorted, so whenever they load the screen for choosing a citizens week plan, the different weeks show up in a random order. 
When this application has been in use for some time and a lot of week plans have been added, it will become impossible to navigate for the guardians without spending a long time finding the correct week plan.
\\\\
For \textit{Weekplanner\#220} the purpose is to support the teaching of decision making to the children.
According to Emil, the children often have problems with making decisions and can get stuck on very simple ones. 
An example could be that a child might not be able to decide if they want to do a puzzle or paint.
This feature would be really helpful in helping the children make decisions.
\\\\
\textit{Weekplanner\#233} is a bug that is important to fix, as pointed out by Emil.
Some citizens might not be able to read "Mandag" if it is written on two lines. 
This is due to them recognizing the entire word and not the individual letters.
\\\\
\textit{Weekplanner\#245} has been included as it is important to give feedback when errors occur. 
The user will not know what happened if an error occurs if no message was given, and that will lead to frustration and will give the user a feeling that the application is unstable.

\subsection{High prioritized tasks}
These user stories are prioritized as high. 
This means that these user stories are very important, but do not give the same amount of value to the customer as the highest prioritized user stories.
\begin{itemize}
    \item Weekplanner\#9 As a guardian, I would like the application to be fully available offline so that I can still use it if the internet is down
    \item Weekplanner\#15 As a guardian I would like to be able to choose how many days a citizen can see at a time on their weekplanner, so that it fits their personal preference 
    \item Weekplanner\#162 As a guardian I would like to be able to setup a template week so that I don't have to duplicate the same week plan  
    \item Weekplanner\#177 As a guardian I would like the week plans on "vælg ugeplan" screen to show the week and year of each plan so that I can easily find the right plan 
    \item Weekplanner\#221 As a guardian I would like to be able to lock a citizens timer so that they can not pause or stop them once they are started
    \item Weekplanner\#227 As a guardian, I would like the search for pictograms to be ordered by how popular a pictogram is, so that I can find the most commonly used pictogram quickly
    \item Weekplanner\#264 As a citizen I would like a ding sound to play when my timer is completed so that I can hear that I am done with my activity
    \item Weekplanner\#265 As a guardian I would like an error message to be shown if I try to log in while I am offline so that I know that I am offline
\end{itemize}
\noindent
The purpose of \textit{Weekplanner\#9} is to make the application useful even when they are not connected to the internet.
The citizens can get frustrated when the internet is not usable, and this will increase their frustration if they are not able to see their schedule.
\\\\
For \textit{Weekplanner\#15} it is important to show fewer days to not overwhelm and confuse some of the citizens.
They might be unable to relate to the pictograms and understand which day it is because there could be too much visual stimulation.
As the citizens grow older and get used to the system, the guardians would usually give them more days to look at.
\\\\
The purpose of \textit{Weekplanner\#162} is to save time for the guardians. 
Most days have things in common for the citizens, and therefore it is a waste of time if the guardians have to make the same week plan over and over again.
Solving this would give the guardians more time to engage with the citizens.
\\\\
\textit{Weekplanner\#177} has the same purpose as \textit{Weekplanner\#200} in \autoref{highest-priority-tasks}. 
The reason why \textit{Weekplanner\#177} is still needed is if the guardians forget whether or not they have created a new week plan.
If so, it would be easier for them to see if the week plan has been created.
Additionally, it is easier for them to find a specific week plan from a previous week.
\\\\
\textit{Weekplanner\#221} is something that Emil requested. 
For some activities it makes sense for the citizens to be able to pause an activity, but for some citizens it would be overwhelming.
Time is also something that the citizens commonly have problems with. 
Giving them the possibility to pause the timer would be confusing for some of them, so the option to not let them should be present.
\\\\
For \textit{Weekplanner\#227} it is important for the guardians to be consistent with the pictograms.
During the usability test described in \autoref{usability-test-sprint-3}, Emil said that they usually do not use more than 50 pictograms for all activities.
To make the search function better he suggested that it could be sorted by their popularity.
The reason why it is important to use the same pictograms is because that the citizens have a specific understanding of the pictograms.
\\\\
The purpose of \textit{Weekplanner\#264} is to enhance the timer functionality.
The citizens will not always pay attention to the timer, so for them to realize that the time is up, they need to hear a ding sound.
\\\\
\textit{Weekplanner\#265} is a requirement we discovered in \autoref{usability-test-14-05}. 
Currently, the user will get an error message that either the username or password is wrong. 
This will get the user to try logging in again instead of solving the problem with the internet connection.
This will lead to frustration and give the user the experience that the program is unstable.

