\section{Future work for PO}
There are multiple tasks for the PO group to get started with.
The new PO group should expect to have a heavy workload in the beginning of the semester because they have to gain an understanding of the status of the project, and what the customers require.
The PO specific tasks we suggest for the first sprint are:

\begin{itemize}
    \item Create prototypes for the following user stories on GitHub:
    \begin{itemize}
        \item Weekplanner\#13 As a guardian I would like guides available for the system so that it is easy to look up the features that I don't fully understand how to use
        \item Weekplanner\#265 As a guardian I would like an error message to be shown if I try to log in while I'm offline so that I know that I'm offline
        \item Weekplanner\#266 As a guardian I would like to be able to add a text to a pictogram in a week plan so that other guardians know what I mean by it
        \item Weekplanner\#267 As a citizen I would like to have text shown under my pictograms that describes what it is supposed to be so that I can learn to combine words with pictures 
        \item Weekplanner\#270 As a citizen I would like an indicator on my week plan to show that an activity has a timer so that I know it beforehand
        \item Weekplanner\#273 As a guardian I would like to be able to delete pictograms I've uploaded so that I can remove wrongly uploaded / outdated pictograms 
    \end{itemize}
    \item Create icons for:
    \begin{itemize}
        \item Log In
        \item Not synced
        \item Synced
    \end{itemize}
    \item Schedule meetings with customers for interviews, usability tests and feedback on prototypes
    \item Read PO advice for 2020. This can be found in \autoref{appendix:PO-advice}
\end{itemize}
\noindent
We strongly suggest to the next PO group that they create the missing prototypes in Adobe XD.
Our prototype files can be found in the \href{https://github.com/aau-giraf/wiki/tree/master/design_guide/prototypes}{GIRAF wiki}. 
Remember that once the prototypes are created, they should be presented to the customers and approved before they are attached to the user stories.
This is a good idea so that the developers will not start working on something only for the customers to wish for something completely different.
\\\\
For the icons there are only three missing. The current icons can be found in the \href{https://github.com/aau-giraf/wiki/blob/master/design_guide/icons.md}{GIRAF design guide}. 
The currently used icons are generally from \href{https://fontawesome.com/}{Font Awesome}, but some are custom made, such as switching between guardian and citizen icons.
These icons were made using Adobe Illustrator to make them available as vector (svg) format.
\\\\
It is also required of the PO group to schedule meetings with the customers.
From our experience, it is recommended that you both call the customers and send emails. 
If you only send emails, it is very likely that you will not get a response from many of the customers. 
If you call them and they are not available, they will usually call back to schedule a meeting.
