\section{Sprint 2020S1}
For the first sprint in 2020, we suggest focusing on the following user stories:

\begin{itemize}
    \item Weekplanner\#15 - As a guardian I would like to be able to choose how many days a citizen can see at a time on their weekplanner, so that it fits their personal preference 
    \item Weekplanner\#177 - As a guardian I would like the week plans on "vælg ugeplan" screen to show the week and year of each plan so that I can easily find the right plan 
    \item Weekplanner\#200 - As a guardian I would like for the week plans to be sorted consistently every time I look at the week plans for a specific citizen
    \item Weekplanner\#202 - As a guardian I would like to be able to edit week plan details (name, picture) so that I can fix mistakes I made
    \item Weekplanner\#233 - Week names split into multiple lines if the display resolution is not high enough 
    \item Weekplanner\#264 - As a citizen I would like a ding sound to play when my timer is completed so that I can hear that I'm done with my activity
\end{itemize}
\noindent
All of the mentioned user stories are either prioritized as high or highest. 
They are considered for the first sprint both because they are important features to have implemented, but also because they are the easier tasks to get started with.
These tasks require an understanding of the GIRAF project, but are good first issues to get familiar with the code base.
