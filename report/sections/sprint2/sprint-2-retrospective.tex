\section{Sprint retrospective}
The sprint retrospective for sprint 2 followed the same procedure as the one for sprint 1, detailed in \autoref{sprint-1-retrospective}

\todo{Im not convinced that this should be here, but rather be a summarized form.}
\subsubsection{The retrospective for the GIRAF project}
The ideas presented at the retrospective have been compiled into the following list:
\begin{itemize}
    \item PO should review the designs for all pull requests
    \item More sessions in which the reviewer sits down with the author of a pull request when reviewing
    \item Keep information about meetings in the calendar
    \item Have a testing workshops
    \item Increase maximum number of characters in a line of code in the lint analysis
    \item Review checklist should be included in each pull request, and be followed by reviewers
    \item The one who merges should ensure their screen works with the retrospective
    \item Too many scrum of scrums meetings
    \item Remember to delete branches after merges
    \item Enforce a time limit on retrospectives and make it more structured
    \item Create rules regarding who should create commits on a pull request and merge or delete them
    \item When a pull request is created it should always include pictures of the user interface
    \item Deadlines on pull request reviews
    \item Better agenda for meetings
    \item A channel on the GIRAF Slack to document changes in the process
    \item More social arrangements
    \item Add start of release to calendar
    \item A better method of determining a groups progress with their user stories
    \item Change lint analysis to give errors rather than warnings when public members are undocumented
    \item Meetings should not be spread out over a single day
    \item Present user stories at the release party
    \item Convert release preparation to a single day where everyone works together in a single room
\end{itemize}
\noindent
All of the ideas were voted on by all GIRAF developers.
This was done to determine which ideas resonated the most with as many developers as possible to determine how to change the process for the next sprint.
The most voted for suggestion was that the release preparation should be converted into a single day.
This was especially popular as the last release preparation was the first one held on the GIRAF project, meaning there was room for improvement.
Many developers felt that the process used for this preparation was a bit messy.
As such, the idea to focus on a single day where all developers gathered to prepare the release was chosen to be implemented in sprint 3.
The second most popular idea was the idea that pull requests should always contain a picture showing the graphical result of the changes.
This would make code reviews more effective.
\\\\
In sprint 2, our group specifically would have to examine the graphical changes made by the different groups to ensure they adhered to the prototypes.
This could take unnecessarily long, so adding pictures to the pull request would aid us in performing these checks.
The third idea to be implemented was to modify the lint analysis of pull requests to give errors rather than a warning when the documentation of public members was not acceptable.
This was implemented to ensure that the code reached a higher quality, and to make it easier for new students to work with in the future.

\subsubsection{A retrospective for our process}
Our process seemed to work well.
As such, we will not be making any changes for the next sprint of the GIRAF project.
