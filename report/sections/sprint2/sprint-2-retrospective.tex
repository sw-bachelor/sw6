\section{Sprint retrospective}\label{sec:sprint-2-retrospective}
The sprint retrospective for sprint 2 followed the same procedure as the one for sprint 1, detailed in \autoref{sprint-1-retrospective}

\subsubsection{The retrospective for the GIRAF project}
Like in the first sprint, multiple ideas were formulated by the GIRAF teams, two of which related to our work:
\begin{itemize}
    \item PO should review the designs for all pull requests
    \item When a pull request is created it should always include pictures of the user interface
\end{itemize}
\noindent
Of 22 ideas suggested, one was the idea that pull requests should always contain a picture showing the graphical result of the changes, as it would make code reviews more effective.
\\\\
In sprint 2, we would compile the application to examine the graphical changes made by the different groups to ensure they adhered to the prototypes.
This could take unnecessarily long, so adding pictures to the pull request would aid us in performing these checks.
\\\\
In the upcoming sprints, there will always be assigned a member of our group to pull requests on the weekplanner repository, so that we can see the design and ensure that it follows the prototypes and seems intuitive for the customers.

\subsubsection{A retrospective for our process}
We experienced that it was hard for us to keep track of exactly what the groups were working on, as some of them would come in with requests to fix minor bugs, that were not reported as issues on GitHub.
To fix this, we used a whiteboard in the group room to write down what each group was working on, with a system where we would circle the issue as soon as it entered the review process, and then remove it from the board when it got through the review.
This helped us a lot with seeing if a group was getting behind on their work, and might need some assistance.
Likewise, we could see if a group were just waiting for their work to go through a review, and could start a new user story.
