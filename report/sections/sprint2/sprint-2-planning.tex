\section{Sprint introduction}
Sprint 2 began on the 18th of March, and stretched until the 8th of April.
As with the first sprint planning conducted for the GIRAF project described in \autoref{sprint-1-planning}, all participating teams gathered to get an understanding of what would happen in the sprint.
This sprint planning differed quiet a bit, however.
Because of suggestions from the developers, the process group had decided that this meeting should primarily function as an introduction where the goals of the sprint and user stories should be presented.

\subsection{The goals and visions for the GIRAF project}
Based on the recent switch of framework for frontend development, we were left with a system that had less functionality than when we started the previous sprint.
Based on this, the following goals were developed:
\begin{itemize}
    \item Achieve a foundation of functionality to make the system usable again
    \item Improve the graphical user interface in the week planner as specified through the prototypes in accordance with the new framework, Flutter.
\end{itemize}
Some development groups were still working on user stories from sprint 1, as there was not enough time left to complete them after the change of framework.
This served as a key point for creation of the sprint backlog.
The most important user stories were those the developers were already working on, but had not completed. 
\\\\
User story selection would differ from the method used during the first sprint, as it was not carried out during the actual planning meeting.
As this meeting was mainly supposed to be an introduction, different teams were allowed to pick an open, highly prioritised user story after discussing them once the meeting had concluded.
This meant that the PO group would not be able to send a representative to each group during planning, but the different groups should approach the PO group for discussion and confirmation of stories when ready.
However, as long as a group had an unfinished story, that story should be the one they focused on.
This meant that the following stories relating to basic functionality of the week planner were carried into sprint 2:
\begin{itemize}
    \item Login functionality
    \item Week plan view
    \item Navigation
    \item Search pictograms
\end{itemize}
When a new story were to be selected, the groups should initially choose one from the following highly prioritised stories:
\begin{itemize}
    \item Update status of activities
    \item Swap between citizen and guardian mode and prompt the user for a password when needed
    \item Reorganizing the order of activities
    \item Show details of an activity
\end{itemize}
When a group finishes implementing their user story during the sprint, they should discuss if they have time to implement a new user story before the sprint is ended. They should then present their chosen user story to the PO group and get a confirmation.

\subsubsection{The goals for the PO group}
One of the goals for the PO group in sprint 2 is to implement the user stories assigned to our group.
The two user stories are weekplanner \# 44 and weekplaner \#43. These are:
\begin{itemize}
 \item "As a citizen I would like to be able to view my week plan so that I know what is going to happen" 
 \item "As a guardian I would like to be able to view a given citizen's week plan so that I can get an overview of what is happening this week for them"
\end{itemize}
We also need to prepare a release, where we choose which user stories should be in the release, based on whether or not they fulfil their requirements in an acceptable fashion.
This release will have a usability test with the customers.
The preparations for sprint 3 include preparing user stories, prioritizing them and making sure that the prototypes are up to date.
\begin{table}[H]
    \centering
    \begin{tabular}{|l|l|}
    \hline
    Goals:                                   \\ \hline
    Implement weekplanner \#44               \\ \hline
    Implement weekplanner \#43               \\ \hline
    Prepare release                          \\ \hline
    Have a usability test                   \\ \hline
    Prepare sprint 3                         \\ \hline
    \end{tabular}
    \caption{Goals for the PO group in sprint 2}
    \label{PO-goal-sprint-2}
\end{table}
