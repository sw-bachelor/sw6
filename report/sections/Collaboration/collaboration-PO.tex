\section{Collaboration between PO and developer groups}
As the PO group we were responsible for making and maintaining user stories for the developer groups.
Whenever a group needed a new task they would have to come to us and ask permission before assigning themselves to a user story.
This gave us a good overview of what groups were working on and where merge conflicts might arise.
Merge conflicts were very resource intensive as it could alter the way a group had gone about solving a user story, requiring mayor code rewriting.
Therefor it was important for us to have as much overview as possible of the developer groups so that we could warn them if merge conflicts might be a problem and advise them to talk to the corresponding group who were working in the same set of files.
\\
\\
Doing the project we often noticed that some user stories were not completed in time despite us never having heard of any troubles from the groups responsible for the user stories.
We therefore started visiting the other groups and ask them how far they had come with their user stories.
This gave us a much better understanding of the problems that the developer groups were experiencing. 
Based on our knowledge on what all the groups had been working on priorly, we were able to recruit people from different groups to come work as consultants to help groups that were stuck.
In extreme situations we would decide to reallocate a user story to a different group if the original group did not have enough time to finish it before the sprint ended.
This greatly increased the productivity as some user stories could be stuck for weeks and then be fixed in a matter of days.
\\
\\
To further help the developer groups we made prototypes for each user story to better convey what the customers wish to have implemented and how it should look.
Developer groups would often visit us for clarification of why the prototypes looked as they did and if they were allowed to change certain aspects or how the interaction between prototypes worked such as if it should be a simple button press or a hold, should there be a pop up window or a whole new screen etc.
The prototypes they were handed were approved by customers so in most cases direct changes there to were denied, but this also helped the developer groups getting a bit of inside into the choices that we, the PO group, took.
\\
\\
When a sprint was at its end we would choose the features that was finished that we wanted to include in a new release.
Together with the Process group we would then create a table where each group would get responsibility for checking a number of features functionality and that there has not been introduced new bugs since merging all the different new features.
To better organize this the we create a release branch together with the process group.
From this branch all the release fixes are branched out and merged into when done.
When working on the release branch all groups were gathered in one room where everyone would be working on getting the application ready for release.
This meant that if there was any question about how a feature worked the group who created it would be present and ready to help if needed.