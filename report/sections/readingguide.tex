\chapter*{Reading guide}
If you are not planning on reading the entire report, these are the sections we recommend you take a look at:

\begin{itemize}
    \item State of GIRAF - February 2019 (\autoref{sec:stateOfGirafFeb2019})
    \item Semester roles (\autoref{sec:semesterRoles})
    \item Technologies and tools (\autoref{technologies-and-tools})
    \item Suggested upcoming features (\autoref{SuggestedUpcomingFeatures})
    \item PO advice for next year's PO group (\autoref{appendix:PO-advice})
    \item User stories for future work (\autoref{appendix:future-work-user-stories})
    \item State of GIRAF - May 2019 (\autoref{appendix:finals-state-of-giraf-2019})
\end{itemize}
\noindent
In addition to these chapters, we suggest that you take a look at the \href{https://github.com/aau-giraf/wiki}{GIRAF wiki}.
This repository contains most of the available information about the GIRAF project, including guides for the new PO and process groups.
\\\\
In the wiki, you will also find a lot of information that is not available in this report, such as the design guide, which explains the design choices that previous PO groups have made in collaboration with the customers.
In addition to this, you will also find information about the weekplanner application, server administration and an overview of the repositories in the GIRAF project.
\\
If you are part of the 2020 PO group, we suggest that you have a look at the GitHub milestone for the suggested first sprint, which can be found \href{https://github.com/issues?utf8=%E2%9C%93&q=milestone%3A%222020S1%22+user%3Aaau-giraf+is%3Aopen+}{here}.
\\\\
Good luck, and enjoy your time working on the GIRAF project!
