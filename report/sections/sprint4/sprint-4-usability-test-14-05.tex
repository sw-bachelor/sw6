\section{Usability test 14/05}\label{usability-test-14-05}
We had planned this usability test so that there were two customers from Egebakken and one customer from Birken who could participate.
Unfortunately, one of Susannes colleagues called in sick, meaning Susanne had to cancel the usability test.
We had created two different usability tests. 
One was specifically for Emil, which was aimed at testing the new features we had completed since the last usability test.
The second one was for Susanne and Mette, and was aimed at testing the new features as well as the features that Emil previously tested in \autoref{usability-test-sprint-3}. 
The purpose of this was to detect if the other customers also found the implemented features intuitive.

The new features we were testing were:

\begin{itemize}
  \item Create a timer in guardian mode
  \item Start and pause a timer in citizen mode
  \item Copy multiple activities to multiple days
  \item Cancel an activity
  \item Delete multiple activities on a weekplan
  \item Delete multiple week plans
\end{itemize}


\subsection{Mette}
Mette did not have many problems using the application.
The tasks for Mette can be found in \autoref{usability-test-14-05-mette}.
The two things she had problems with were mark mode and logging in.
When trying to login she was prompted that she either had a wrong username or password. 
She tried it again and was still not able to login.
We realized that this was due to the tablet disconnecting from the internet.
After setting up a local hotspot and connecting it to the internet, she was able to login.
This indicated we lacked an appropriate error message for this issue.
\\\\
She did not have any problems navigating to the correct citizen's weekplan and adding an activity.
She then had an assignment to move the newly added activity between two other activities, which proved rather difficult. 
This was also one of the problems that Emil had at the last usability test.
She also had some issues with creating a new week plan.
She filled out the form, but forgot to add a pictogram. 
When she pressed the button to create the new week plan, there was no feedback that she was missing a pictogram.
After that, she pressed the back button to find it since she thought it was created and had to start over with the assignment. 
She needed help to realize that she needed to add a pictogram before pressing the button to create a new pictogram.
This issue was also present at the last usability test, described in \autoref{usability-test-sprint-3}.
\\\\
After this, her assignment was to create a normal weekday and copy it to the other weekdays.
She had no problems creating a normal weekday, but she had issues copying to other days.
She tried multiple times by dragging and dropping, but found that this was not the solution. She looked for an icon to copy but was unable to find it. 
We guided her to the mark mode button, and after clicking it she figured out how to copy to other days.
\\\\
After this she had to delete the last activity on every weekday, and realized after a bit of time that she would be able to mark them all, and then delete them all at once.
The assignments involving the timer posed no problems, but when she had to mark an activity as done, there was a bug that did not update the pictogram with the check mark properly. 
We intervened again to tell her how it was supposed to react and confirm that she had completed the task.
When she had to delete week plans she initially tried to longpress, but then remembered how the mark mode button worked. 
Once employing mark mode, she had no problems solving the exercise.

\subsection{Emil}
Emil generally did not have many issues using the application. 
His tasks can be found in \autoref{usability-test-14-05-emil}.
He was accustomed to the application since he also participated in the usability test conducted on May 2nd. 
Because of this we prepared tasks for Emil that were mainly focused on new additions to the application in sprint 4 specifically.
As was the case with the last usability test he easily added pictograms to the weekplanner, but felt that he was unsure whether they were saved or not, making him hesitant to go back to the previous screen where a week plan is selected. 
Some sort of feedback that the state of the application was saved could be considered for the next semester.
\\\\
Initially Emil struggled a bit with copying activities.
He tried to longpress rather than use the mark mode functionality, but eventually used the proper functionality.
He tried to copy without marking any activities, and reacted positively to the error message that ensued, telling him to actually mark activities.
Deleting was easy once he understood the mark mode functionality.
Instead of marking an activity as cancelled, he ended up mistakenly deleting it.
This was due to him thinking the cancel button was used to stop mark mode rather than its actual purpose, cancelling activites.
All the functionality related to adding a timer was intuitive. 
Emil performed the timer tasks without any issues.
\\\\
Emil thought the application was easy to use overall considering the functionalities it had, even when he compared it to similar applications he had worked with in the past.
He once again pointed out that he would like some kind of sorting for pictograms.
The cancel activity button was not as intuitive as we had hoped, and he requested that we add some text to the button to further explain its functionality.
He thought that the terminology used for cancellation within mark mode was a bit ambiguous, and requested we change it.
Finally, he asked if we could add the name of the citizen to the top bar as a way to keep track.

\subsection{Discussion of new prototypes}
We showed them our initial thoughts related to the implementation of a choice board.
This functionality would let the citizen choose one activity from a selection of activities.
We had built a design based around four activities being the maximum amount, and both Emil and Mette confirmed that this was the proper amount, and that they would generally not exceed it. 
However, they made us aware of a complication we had not considered - what if the citizen chose the wrong activity?
This would require some sort of reset button, either on the week plan itself or for the specific choice board.
\\\\
During the last usability test, described in \autoref{usability-test-sprint-3}, Emil had expressed a need for the ability to lock a timer.
The way this was prototyped was accepted, but we were unsure what should happen once the timer had finished.
They explained that adding a “ding” sound would most likely be beneficial, as a lot of the citizens liked that sort of feedback.
When showing the prototypes for showing different amounts of days of a week plan at a time, they remarked that the prototype we showed for showing a single day would likely be confusing for the citizens.
On both edges of the screen we had a small part of the previous and next days shown, to indicate that they would actually occur.
They wanted this removed, so that only one color would be visible.
When discussing the functionality to add a new pictogram through the gallery of the device, we discussed whether or not they would like the functionality to make a pictogram private for a specific citizen.
This was something they had not thought about in relation to the application, but they decided that this was actually important, and requested that it would be added.

 	          
