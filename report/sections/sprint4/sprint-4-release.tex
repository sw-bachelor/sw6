\section{Release 2019S4R1}
As mentioned in the introduction \autoref{sec:sprint-4-introduction}, we wanted to focus on having a minimal viable product ready for this release, but also on having documented the necessary information that we could hand over to the next students that will be working on GIRAF in 2020.
\\\\
For this release we used the same setup as in the previous release preparation in \autoref{sec:sprint-3-release}. 
We had two days to prepare the release, so the deadline for the different user stories we had assigned to the groups were two days before the sprint ended. When release preparation started, we created the release branch that the groups would be branching out from and merging into during release. Each group would then review the implementation of one of the other group's user story to check if it had been correctly implemented and if any bugs had been introduced. We then had them report all the bugs that they found. We had them use the same checklist that we had developed for sprint 3 release. The rest of the release preparation was spent with all groups sitting together and fixing the most critical bugs. Instead of us, the PO group, assigning groups to the reported issues, we let groups assign themselves to the issues.
\\\\
The students that had worked on GIRAF the previous year had implemented an admin panel for web browsers. This had been broken earlier by some of the changes we had made to the backend/server. During this sprint we got it working again, which meant that we once again could add guardians and citizens to the system. 
Furthermore, each group finished their user stories for this release, which meant that we achieved the minimal viable product that we had set as our goal. 
