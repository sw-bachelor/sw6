\section{Release 2019S4R1}
As mentioned in the introduction \autoref{sec:sprint-4-introduction}, we wanted to focus on having a minimal viable product ready for this release, but also on having documented the necessary information that we could hand over to the next students that will be working on GIRAF in 2020.
\\\\
For this release we used the same setup as in the previous release preparation in \autoref{sec:sprint-3-release}. 
We had two days to prepare the release, so the deadline for the different user stories we had assigned to the groups were two days before the sprint ended.
When release preparation started, we created the release branch that the groups would be branching out from and merging into during release.
Each group would then review the implementation of another group's user story to check if it had been correctly implemented and if any bugs had been introduced.
We then had them report all the bugs that they found by using the same checklist that we had developed for sprint 3 release.
The rest of the release preparation was spent with all groups sitting together and fixing the most critical bugs.
Instead of us assigning groups to the reported issues, we let groups assign themselves to the issues.
\\\\
The students who had worked on GIRAF the previous year had implemented an admin panel for web browsers, which had been broken earlier by some of the changes that were made to the server.
During this sprint group sw612f19 got it working again, which meant that we once again could add guardians and citizens to the system.
\\\\
The user story assigned to group sw612f1 required removing the back button from the \texttt{GirafAppBar} when the weekplan screen was in citizen mode, and had not been completed in time for the release preparation.
Therefore, we took over the implementation of this user story, since we had created the current implementation of the \texttt{GirafAppBar}.
The implementation details are described below.
\begin{lstlisting}[caption={Removing back button from the citizens screen},label={lst:removeBackButton},language={[Sharp]C}]
    GirafAppBar({Key key, this.title, this.appBarIcons, this.isGuardian = true})
\end{lstlisting}
The constructor of the \texttt{GirafAppBar} can be seen on \autoref{lst:removeBackButton}. 
The way that we implemented it was by adding a flag called \texttt{isGuardian} to the \texttt{GirafAppBar}. 
The \texttt{isGuardian} flag defaults to true if it is not set when instantiating the \texttt{GirafAppBar}.
The reason for this is that more often than not the application needs a back button.
Therefore, to make it easier for the programmer, the \texttt{isGuardian} flag only needs to be set when the back button should not be shown.
\\\\
Besides this user story, each group finished their delegated user stories for this release.
All the completed user stories were featured in the release, which meant that we achieved the minimal viable product that we had set as our goal. 
