This chapter presents the fourth and final sprint of the GIRAF project 2019.
This was the shortest sprint, and the procedures used to carry it out differed a bit from previous sprints.
To ensure a usable product by the end of the semester, we defined a minimum viable product and prioritized user stories according to this.
Extensive documentation had to be created to facilitate a smooth introduction to the project for the next generation of GIRAF developers.
What we focused on documenting will be detailed in the chapter. 
After the release had been completed, we carried out two different usability tests.
We attempted to invite representatives from the different cooperating institutions, but the representatives from Birken had to cancel.
To get as much feedback as possible, we scheduled an extra usability test for the Birken representatives.
Both usability tests will be detailed in this chapter. 
Finally, a retrospective was carried out to reflect on the GIRAF project as a whole.
