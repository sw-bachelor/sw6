\section{Sprint retrospective}
This sprint retrospective was like the sprint 3 retrospective that is described in \autoref{sprint-3-retrospective}.
The only difference on this retrospective from the last was that the process group did not choose areas where we needed to focus.
They wrote the amount of people who agree, do not care and disagree followed by their comments to it.
There were more comments in the retrospective, but we only reflect on the once that are PO related.
\\\\
\textit{"Next years PO group should do stories in a story format instead of just a title. Some times it was hard to agree on how a user story should be made."}
\\
18 agree, 7 do not care, 5 disagree.
\\
This suggestion is a good idea. 
This should however be an addition to the user story title and an addition to the prototypes. 
The story format would be describing in text, how a user would complete the user story. 
Doing this would eliminate some of the ambiguity that can be in some user stories.
\\\\
\textit{"The tasks were difficult to understand directly from the description"}
\\
7 agree, 6 do not care, 17 disagree.
\\
There doesn't seem to be a lot who had difficulties understanding the tasks directly from the description. 
Including a story format in the user stories would make the tasks easier to understand.
However, as we are always available in the group rooms, we have always recommended that if something is ambiguous or difficult to understand, then they could just come in and ask us. 
\\\\
\textit{"It would be nice if there were smaller task that you could take without contacting PO"}
\\
10 agree, 11 do not care, 9 disagree.
\\
This is something that we do not recommend to implement. 
It is required for us to understand what is being worked on, and what is not being worked on. 
If people assign themselves to tasks, we will have a more difficult time to understand how far we are in the process.
There is also the possibility, that someone forgets to assign themselves to the tasks hereby multiple people work on the same task.
Currently if someone forgets to assign themselves to tasks, then we have the possibility to discover the error, before another person is assigned to the task.
\\\\
\textit{"Task delegation was fine"}
\\
20 agree, 9 do not care, 1 disagrees.
\\
It was surprising for us that so many people are satisfied with being delegated tasks. 
We choose to delegate the tasks with the purpose of trying to achieve the minimal viable product.
\\\\
\textit{"Good understanding of sprint 4, which made the sprint good. The final process was good. Delegation of user stories was good to give it purpose."}
\\
21 agree, 7 do not care, 2 disagree.
\\
Fine.
\todo{Hvad skriver man lige til det her når jeg har skrevet det tidligere? Skal denne her slettes eller den anden}
\\\\
\textit{"PO decides how the product should be, but the developers decide how it should be implemented"}
\\
27 agree, 3 do not care, 0 disagree.
\\
This is something that there is a big agreement on, but this is also how the process worked this year.
\\\\
\textit{"It is irrelevant to say which files you are working on in the standup meetings. The user stories should be planned better to avoid merge conflicts."}
\\
9 agree, 10 do not care, 11 disagree.
\\
This partly conflicts with the previous comment. 
If we are to plan the user stories and see what files are needed for each user story, then we need to decide where and how the developer should be implementing their user story. 
This would be very time consuming for the PO groups, but also give a worse product, as the developers will have better knowledge on how to implement the user stories.
We think that it is much better to communicate which files you are working on in the stand up meetings to avoid merge conflicts, and additionally for the groups to see dependencies.
