\section{Documentation for handover}
Our primary focus for the final sprint of the project was to ensure that the system was properly documented, and easy to get started with for next year.
To facilitate this, we working with the process group on formulating some tasks for documentation.

This included ensuring that all important parts of the system was described, such as screens, blocs and endpoints.
In addition to this, documents were made for the next process group and PO group, with our advice on how they get started, and what their role in the GIRAF project is.

\subsubsection{Our documentations}
Most of our work on documentation was spent on the \href{https://github.com/aau-giraf/wiki/blob/master/advice_for_future_giraf/PO_advice.md}{GIRAF PO handbook}, which details our role in the GIRAF project:
This can also be found in \autoref{appendix:PO-advice}.

\begin{itemize}
    \item Customer contact
    \item Creating user stories 
    \item Creating prototypes
    \item Distribution of user stories across groups
    \item Communication with other groups
    \item Cooperation with the process group
    \item Approval of GUI related design
    \item Release preparation
    \item Usability testing
    \item Keeping the issues updated
    \item Sprint planning
    \item Internal sharing of knowledge
\end{itemize}

These are the things we wish we had known, when we first started working on the GIRAF project, so we spent a great deal of time ensuring that this would be available for next year.
\\\\
We also documented parts of the system:

\begin{itemize}
    \item Described new endpoints and new controller in the API
    \item Documented the activity\_bloc, which is responsible for handling all the functionality related to activities
    \item Documented the auth\_bloc, which is responsible for handling authentication, such as logging in and out
    \item Documented all the repositories in the GIRAF project, including the archived ones from previous years, in collaboration with sw611f19
\end{itemize}

All of this documentation can be found in the \href{https://github.com/aau-giraf/wiki}{wiki repository}.
