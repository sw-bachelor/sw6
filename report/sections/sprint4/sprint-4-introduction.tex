\section{Sprint introduction}
Sprint 4 began on the 29th of April, and lasted until the 13th of May.
Unlike the first sprints, this sprint lasted only two weeks, to allow the groups to have some time to work on their individual reports before the hand in on the 28th of May.

\subsection{The goals and vision for the GIRAF project}
The goals for the fourth sprint were focused more on the transmission of information for the students of 2020, and ensuring that the product was at a state that could be classified as a minimum viable product for the customers.

\begin{itemize}
    \item Finish user stories from previous sprints
    \item Finish the following functionalities: 
    \begin{itemize}
        \item Delete week plan
        \item Timer functionality
        \item Mark activities as cancelled
        \item Expansion of the mark mode
    \end{itemize}
    \item Minor bug fixes
    \item Documentation
\end{itemize}

Unlike the previous sprints where each group got to decide what they wanted to work on, we made an assessment on which user stories needed to be finished in order to have a minimal viable product. 
We assigned each story to a group based on their previous experiences, such that they would mostly work on expanding their own previously written code, to ensure the stories would be implemented in time.

This lead to the following focuses for each group:
\begin{itemize}
    \item Group 8 would work on marking an activity as cancelled
    \item Group 9 would work on adding a timer to an activity
    \item Group 11 would work on fixing a bug where the password for changing from citizen to guardian would be saved
    \item Group 12 would work on hiding the "back" button when using citizen mode
    \item Group 13 would work on expanding mark mode so that it is possible to delete week plans and mark several activities as cancelled or delete them at once
\end{itemize}

While the development groups focused on implementing their user stories, the process and PO groups decided to focus on writing documentation to transmit our knowledge to the next semester of developers.
This meant that we did not have any user stories to implement this sprint, but rather focused on figuring out what information we needed to pass on, write it down and figure out how to make it easily accessible for next year.

Meanwhile, we also focused on finishing the last prototypes, so that all user stories had prototypes available for next year, to minimize the amount of work needed by the next PO group.
