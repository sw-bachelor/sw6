PO:
Okay så, vi har ligesom fået delt spørgsmålene ind i sådan tre kategorier. 
De første de bliver sådan lidt med for at forstå hvordan i lige nu benytter IT i hverdagen, og så kommer der sådan noget specifikt til GIRAF projektet, og så har vi lige nogle sådan rent proces relaterede - vores gruppe og jer som kunde imellem.
Det første vi har tænkt på, det er, sådan i store træk, hvordan benytter i IT i hverdagen lige nu?   

Emil: 
Ja, og det er jo, der skal vi lige sådan snævre det lidt ind. 
Jeg fortalte jo også hvor bredt vi egentligt benytter IT på skolen, ikke også, indenfor alle områder. 
Det gennemsyrer jo en arbejdsplads som en specialskole, men er det med særlig fokus på det her omkring struktur og kommunikation, eller?

PO: 
Ja, lige præcis.
Som støttemiddel til eleverne.

Emil:
Ja.
Hvordan bruger vi det der?
Jamen vi bruger det ikke sådan, hvad kan man sige, ensrettet.
Det er meget fra klasse til klasse, med det personale der tilfældigvis er i klasserne hvad de har med sig af erfaring og hvad det er de har for nogle konkrete elever og hvad det ligesom kan løse af udfordringer i hverdagen.
Det er med det perspektiv hvor det er.
Så det er på den måde er det meget forskelligartet hvordan det bliver brugt.
Vi har nogle klasser hvor at de kører for eksempel, dagstruktur bliver kørt sådan fælles på en tavle, så alle elever får det samme, ikke også, som sådan et fælles opmærksomhedspunkt.
Og andre har hver deres system hvor måske nogle af dem har det sådan elektronisk. 
Så, ja.

PO:
Lige for at følge op til det der - er det sådan at hver klasse, er det kun delt ind i sådan årgang og sådan nogle ting der eller er det også hvordan de er som personer?

Emil:
Det er et godt spørgsmål.
Selvfølgelig så har, så går dem der går til afgangsprøve jo ikke med nogle som ikke har talesprog, det er jo ikke sjov undervisning.
Så, vi har på skolen delt dem op i tre afdelinger.
En afdeling som der faktisk ikke engang ligger i Hammer Bakker, men som ligger ude i Sulsted der hedder Agernhuset for vores højt begavede, normal begavede unge som der følger almindelige skolematerialer.
De går ude i den afdeling, og så har vi sådan en hovedafdeling hvor der er en mellemgruppe, både for dem der er så, ja, dem der ikke lige passer direkte i nogen af dem, og så er der dem der næsten ingen talesprog har.
Mellemgruppen den er rimelig bred. 
Der er nogle som hvor deres sprog det er sådan rimeligt upåfaldende, men de har måske nogle massive autismevanskeligheder omkring det sociale, for eksempel, og ja.
Der kan være mange ting i det, men så hver afdeling har så tre trin.
En udskolingsgruppe, og en indskolingsgruppe og et mellemtrin. 
Så man finder ud af sådan hvor passer de henne sådan kognitivt og hvad for nogle skolematerialer kan de følge, ikke også, og så placerer man dem i afdelinger efter det, og så placerer man dem i klasse efter alder.
Det skal være en af de tre trin. 
Så der er rimelig stor spredning i vores, det er klart når der er nogle der går i både børnehaveklasse med nogle der også går i trejde.
Så, ja, men sådan er det.
Det er jo små grupper til gengæld, så vi differentierer jo hæftigt i hvad de får af undervisning.

PO:
Ja okay.
Har de sådan, alle eleverne, har de iPad eller tablet til rådighed?

Emil:
Ja.
Alle elever har iPads til rådighed. 
Det er helt fast, og nogle har endda mere end en iPad til rådighed.

PO:
Er det noget de sådan er rimelig kompetente i at bruge?

Emil: 
Alle er 100\% kompetente i iPads, ja.
Jeg tror faktisk der er en enkelt klasse der så ikke har iPads ude ved Agernhuset, men de har fået PCer fordi de skal bruge det i forhold til deres afgangsprøve, og man kan kun få en enhed.
Generelt så er det iPads der, for 85\% af alle vores elever. 
Og dem kan alle finde ud af at bruge, selv de aller dårligste kan finde ud af at trykke på en iPad.
Ja.
Det er noget af et frmeskridt i forhold til da det var på PC med mus, den der hvor du skal styre noget og klikke op på en skærm.
Der var det ikke alle der kunne bruge computer, men alle kan bruge en iPad.

PO: 
Er det sådan nogle specille områder du sådan i hverdagen har tænkt at, det kunne være smart hvis det her det var digtaliseret?

Emil:
Ja. 
Godt spørgsmål.
Altså, vi er jo sådan en arbejdsplads og et område med rigtig mange traditioner også, hvor at ,am jo gør meget man plejer og har erfaring for at fungere. 
Så det, man kan jo sige at specialområdet er jo generelt, selvom de også er med på at udvikle mange tinge, så er de også meget traditionsbundne. 
Så det kan gøre det lidt svært at rykke.
Altså, jeg synes egentligt, altså det er nogle af de områder jeg har været inde på omkring kommunikation og struktur hvor jeg tænker at der er, det er mest meningsfult at bruge det i nogle særlige situationer.
Vil jeg sige.
Det er, der kan man sige, de programmer der kan altid gøres bedre end det de er. 
Der er ulemper ved dem alle sammen. 
De har hver deres fordele og hver deres ulemper.
så vi er hele tiden sådan på afsøgning af, hvad er det, hvor er det vi får mest for pengene.
Hvad, ja, hvad rammer dem bedst.
Og det skifter også hele tiden.
Det gør det.
Jeg tror egentligt umiddelbart at vores største udfordring, det er personalet som brugere, umiddelbart.
Så jo mere nemt det er at gå til for personalet, jo større vil impact kan du få på det.

PO:
Må jeg lige høre, nu når du siger det der, er det sådan at lærer personalet op i hvordan man bruger de der apps der?

Emil:
Ja.
Det er forskelligt også, nogle, der er jo nogle de kan jo gå til det med det samme ikke? 
Og så har vi også nogle der har rigtigt svært ved det.
Det er bare hvis du skal bruge noget, hvis du skal støtte en elev på noget der er så vigtigt som at de trives i deres hverdag og kan have en forudsiglighed og struktur, ikke også, så er det jo vigtigt at alle voksne omkring barnet kan finde ud af at tilgå det her og støtte barnet med det.
Det nytter ikke noget det kun er halvdelen.
Så er det bedre at bruge noget andet.
Så derfor, så kan man sige, er det meget brugervenligt så er der stor sandsynlighed for at alle vil kunne gå ind og gøre det.

PO:
Er det noget, sådan de apps som i har nu for eksempel, de gør noget for at få, eller sådan gør det nemmere for jer?
Der er Selvfølgelig brugervenlighed i appsne, men er det også sådan guides inde i appsne eller sådan noget?

Emil: 
Faktisk så vil jeg sige, at den der som vi nu lige forsøger at tilkøbe, det irriterer mig enormt meget at jeg ikke kan huske det.
Hvad er det den hedder?
Det er ikke Showmyday.
Nå, den vandt faktisk, vi havde, vi har haft en større afsøgning siden efteråret hvor vi så, hvad er der egentligt overhovedet af muligheder?
Og så skriver vi det ind.
Vi havde to tilbage til sidst. 
Dene ene den var billig, og den anden var sådan en mellemklasse app.
Vi valgte den vi valgte primært fordi at der er så ekstremt mange guides på YouTube til hvordan man skal bruge det.
Så det er mega nemt for personalet at gå ind, den er simpelthen, når du åbner den så er den utrolig intuitiv og nem at gå til for de fleste, og har man brug for støtte er støtten rigtig nem at få.
Det vil også være i forhold til forældre og sådan noget, ikke også, at de kan, man kan henvise til har du set der er YouTube klip nummer 35, der kan du se lige præcis det der du spørger om.

8.40

