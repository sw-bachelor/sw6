PO:
Okay så, vi har ligesom fået delt spørgsmålene ind i sådan tre kategorier. 
De første de bliver sådan lidt med for at forstå hvordan i lige nu benytter IT i hverdagen, og så kommer der sådan noget specifikt til GIRAF projektet, og så har vi lige nogle sådan rent proces relaterede - vores gruppe og jer som kunde imellem.
Det første vi har tænkt på, det er, sådan i store træk, hvordan benytter i IT i hverdagen lige nu?   

Emil: 
Ja, og det er jo, der skal vi lige sådan snævre det lidt ind. 
Jeg fortalte jo også hvor bredt vi egentligt benytter IT på skolen, ikke også, indenfor alle områder. 
Det gennemsyrer jo en arbejdsplads som en specialskole, men er det med særlig fokus på det her omkring struktur og kommunikation, eller?

PO: 
Ja, lige præcis.
Som støttemiddel til eleverne.

Emil:
Ja.
Hvordan bruger vi det der?
Jamen vi bruger det ikke sådan, hvad kan man sige, ensrettet.
Det er meget fra klasse til klasse, med det personale der tilfældigvis er i klasserne hvad de har med sig af erfaring og hvad det er de har for nogle konkrete elever og hvad det ligesom kan løse af udfordringer i hverdagen.
Det er med det perspektiv hvor det er.
Så det er på den måde er det meget forskelligartet hvordan det bliver brugt.
Vi har nogle klasser hvor at de kører for eksempel, dagstruktur bliver kørt sådan fælles på en tavle, så alle elever får det samme, ikke også, som sådan et fælles opmærksomhedspunkt.
Og andre har hver deres system hvor måske nogle af dem har det sådan elektronisk. 
Så, ja.

PO:
Lige for at følge op til det der - er det sådan at hver klasse, er det kun delt ind i sådan årgang og sådan nogle ting der eller er det også hvordan de er som personer?

Emil:
Det er et godt spørgsmål.
Selvfølgelig så har, så går dem der går til afgangsprøve jo ikke med nogle som ikke har talesprog, det er jo ikke sjov undervisning.
Så, vi har på skolen delt dem op i tre afdelinger.
En afdeling som der faktisk ikke engang ligger i Hammer Bakker, men som ligger ude i Sulsted der hedder Agernhuset for vores højt begavede, normal begavede unge som der følger almindelige skolematerialer.
De går ude i den afdeling, og så har vi sådan en hovedafdeling hvor der er en mellemgruppe, både for dem der er så, ja, dem der ikke lige passer direkte i nogen af dem, og så er der dem der næsten ingen talesprog har.
Mellemgruppen den er rimelig bred. 
Der er nogle som hvor deres sprog det er sådan rimeligt upåfaldende, men de har måske nogle massive autismevanskeligheder omkring det sociale, for eksempel, og ja.
Der kan være mange ting i det, men så hver afdeling har så tre trin.
En udskolingsgruppe, og en indskolingsgruppe og et mellemtrin. 
Så man finder ud af sådan hvor passer de henne sådan kognitivt og hvad for nogle skolematerialer kan de følge, ikke også, og så placerer man dem i afdelinger efter det, og så placerer man dem i klasse efter alder.
Det skal være en af de tre trin. 
Så der er rimelig stor spredning i vores, det er klart når der er nogle der går i både børnehaveklasse med nogle der også går i trejde.
Så, ja, men sådan er det.
Det er jo små grupper til gengæld, så vi differentierer jo hæftigt i hvad de får af undervisning.

PO:
Ja okay.
Har de sådan, alle eleverne, har de iPad eller tablet til rådighed?

Emil:
Ja.
Alle elever har iPads til rådighed. 
Det er helt fast, og nogle har endda mere end en iPad til rådighed.

PO:
Er det noget de sådan er rimelig kompetente i at bruge?

Emil: 
Alle er 100\% kompetente i iPads, ja.
Jeg tror faktisk der er en enkelt klasse der så ikke har iPads ude ved Agernhuset, men de har fået PCer fordi de skal bruge det i forhold til deres afgangsprøve, og man kan kun få en enhed.
Generelt så er det iPads der, for 85\% af alle vores elever. 
Og dem kan alle finde ud af at bruge, selv de aller dårligste kan finde ud af at trykke på en iPad.
Ja.
Det er noget af et frmeskridt i forhold til da det var på PC med mus, den der hvor du skal styre noget og klikke op på en skærm.
Der var det ikke alle der kunne bruge computer, men alle kan bruge en iPad.

PO: 
Er det sådan nogle specille områder du sådan i hverdagen har tænkt at, det kunne være smart hvis det her det var digtaliseret?

Emil:
Ja. 
Godt spørgsmål.
Altså, vi er jo sådan en arbejdsplads og et område med rigtig mange traditioner også, hvor at ,am jo gør meget man plejer og har erfaring for at fungere. 
Så det, man kan jo sige at specialområdet er jo generelt, selvom de også er med på at udvikle mange tinge, så er de også meget traditionsbundne. 
Så det kan gøre det lidt svært at rykke.
Altså, jeg synes egentligt, altså det er nogle af de områder jeg har været inde på omkring kommunikation og struktur hvor jeg tænker at der er, det er mest meningsfult at bruge det i nogle særlige situationer.
Vil jeg sige.
Det er, der kan man sige, de programmer der kan altid gøres bedre end det de er. 
Der er ulemper ved dem alle sammen. 
De har hver deres fordele og hver deres ulemper.
så vi er hele tiden sådan på afsøgning af, hvad er det, hvor er det vi får mest for pengene.
Hvad, ja, hvad rammer dem bedst.
Og det skifter også hele tiden.
Det gør det.
Jeg tror egentligt umiddelbart at vores største udfordring, det er personalet som brugere, umiddelbart.
Så jo mere nemt det er at gå til for personalet, jo større vil impact kan du få på det.

PO:
Må jeg lige høre, nu når du siger det der, er det sådan at lærer personalet op i hvordan man bruger de der apps der?

Emil:
Ja.
Det er forskelligt også, nogle, der er jo nogle de kan jo gå til det med det samme ikke? 
Og så har vi også nogle der har rigtigt svært ved det.
Det er bare hvis du skal bruge noget, hvis du skal støtte en elev på noget der er så vigtigt som at de trives i deres hverdag og kan have en forudsiglighed og struktur, ikke også, så er det jo vigtigt at alle voksne omkring barnet kan finde ud af at tilgå det her og støtte barnet med det.
Det nytter ikke noget det kun er halvdelen.
Så er det bedre at bruge noget andet.
Så derfor, så kan man sige, er det meget brugervenligt så er der stor sandsynlighed for at alle vil kunne gå ind og gøre det.

PO:
Er det noget, sådan de apps som i har nu for eksempel, de gør noget for at få, eller sådan gør det nemmere for jer?
Der er Selvfølgelig brugervenlighed i appsne, men er det også sådan guides inde i appsne eller sådan noget?

Emil: 
Faktisk så vil jeg sige, at den der som vi nu lige forsøger at tilkøbe, det irriterer mig enormt meget at jeg ikke kan huske det.
Hvad er det den hedder?
Det er ikke Showmyday.
Nå, den vandt faktisk, vi havde, vi har haft en større afsøgning siden efteråret hvor vi så, hvad er der egentligt overhovedet af muligheder?
Og så skriver vi det ind.
Vi havde to tilbage til sidst. 
Dene ene den var billig, og den anden var sådan en mellemklasse app.
Vi valgte den vi valgte primært fordi at der er så ekstremt mange guides på YouTube til hvordan man skal bruge det.
Så det er mega nemt for personalet at gå ind, den er simpelthen, når du åbner den så er den utrolig intuitiv og nem at gå til for de fleste, og har man brug for støtte er støtten rigtig nem at få.
Det vil også være i forhold til forældre og sådan noget, ikke også, at de kan, man kan henvise til har du set der er YouTube klip nummer 35, der kan du se lige præcis det der du spørger om.
Så det gør det nemt at få det, få hele barnets familie med omkring det, så det ikke er en person i teamet der skal vejlede både kollegaer og forældre og altså det bliver en kæmpe arbejdsbelastning.
Så det var faktisk det den vandt på. 
Den anden den tabte især på at den kun kørte på Android. 
Det er også lidt en ulempe. 
Eller også så kørte den kun på Apple, og så var der nogle af vores elever der havde... 
Nej den kørte faktisk kun på iOS, og så var det sådan at vi har nogle elever der ville have rigtig stor gavn af den som har Android telefoner og ikke ville kunne bruge den der.
Hvor den anden vi købte den kører på begge systemer.
Så det er et stort plus at der er den fleksibilitet. 

PO:
Så det er en stor bonus hvis der var noget dokumentation, for eksempel i videoformat, tilgængeligt for GIRAF?

Emil:
Ja, klart.
Noget guide.
Altså man skal tænke at det skal være nemt at gå til og der skal være, du skal kunne hente støtte nogen steder, ja.

PO:
Er det sådan hovedsagligt kun for guardians, eller er det også sådan for citizens? At de skal have det der i videoformat, eller er det simpelthen personalet man helst bare skal?

Emil:
Det kan jo være, det kommer jo an på hvordan man laver det.
Så det behøver det jo ikke være.
Det kan også bare være hjælpefunktioner der er gode og så noget derinde, så det synes jeg er svært at svære entydigt på.
Hjælpen skal bare tænkes ind.
Man skal tænke ind at lærere og pædagoger er ikke nødvendigvis gode til IT.
Overhovedet.
Det er bedre at tænke det modsatte. 
Og så er der nogle der har rigtigt nemt ved det, men sådan er det.
Det skal designes til ens mormor.
Så noget jeg egentligt også synes vi mangler lidt, det er apps der kan, jeg viste det der med timeren i så med uret, det er jo sådan en app, det er jo en købe-app, jeg tror den koster en halvtredser eller sådan noget, og den er god fordi den ligner rigtig meget det ur vi har.
Nogle forskellige repræsentationsformer af tiden der går, det er faktisklidt en mangelvare synes jeg.    
Jeg synes egentligtpgså det er halvdyrt for et program der er meget, meget simpelt at lave. 
Det der det kan satan edme ikke tage ret meget tid at lave det kode til og køre den der ned.
Det synes jeg faktisk at vi har haft tænkt lidt ind i GIRAF, hvordan skulle tiden repræsenteres.
Det er der tænkt lidt for lidt over synes jeg.  
Det er faktisk et ret vigtigt issue for vores elever. 

PO:
Hvordan tænker du så man kunne repræsentere tid? 
Jeg tænker umiddelbartat det var noget der ville kunne være svært.

Emil:
Det kunne jo være alt fra, der er sådan lidt forskellige systemer man bruger, der er nogle der arbejder med sådan noget der hedder quarter-dots hvor det er sådan nogen røde prikker der ligesom forsvinder væk.
Du kender det jo også, det kunne også være sådan en bar som du har når et stykke software det loader som der går ned.    
Så det kan tænkes på mange måder, det er bare, det behøver ikke nødvendigvisvære den der røde cirkel.
 
PO:
Vi har også, jeg ved at der har været snak om lige nøjagtigt den feature med sidste års studerende, hvor de blandt andet har haft noget timeglasrepræsentation eller bare som digital tid.

Emil:
Ja, vi har også nogle der bruger almindelige timeglas, det er der også nogen der er glade for.
Det vil være rimeligtmeneingsfuldt at der var flere måder at gøre det på.
Det ser jeg i hvert fald ikke i ret mange software, at de har det. 

PO:
Er det så sådan for hvert barn at man skal kunne gp ind og vælge hvilken form de gerne vil se?

Emil:
Ja.
Det kunne også være at man havde dem, det er jo igen det der med hvor meget skal det være flettet sammen, og hvor meget skal det være selvstændige apps. 
Hvor man kan sige, for nogle børn ville det, hvis man bare havde en app der hed tidstageren eller hvad fanden man kunne kalde den, så havde du den gennem dagen på din iPad, hvor du kunne se nu skal du lige, nu skal du vænne dig, vi ved bare den her med batteriet den er super god til dig fordi du kender den fra din iPad.
Så den tæller ned eller hvad den gør, ikke?
Ja, eller quarter-dots eller en time timer så man ligesom har en lille pakke der med forskellige tidsrepræsentationer.  
Så kan det også være meningsfuldt selvfølgelig hvis du kan koble det sammen med aktiviteterne, at man kan se det i forbindelse med kalenderen. 
Ja, det er hvor integreret det skal være.   
Men tidsdelen, den synes jeg den mangler.

PO:
Der er et spørgsmål der relaterer sig til farver, hvor vi fik at vide at det var en international standard, er der sådan andre standarder man skal være opmærksom på, du lige kender til?

Emil: 
Nej.
Standarden er jo bare at det skal kunne tilpasses. 
Eller, farverne ja, det er en standard, det vil jeg sige, det nok er det. 
Så er der noget som er hyppigt brugt, som for eksempel ikonerne, hvor du tit bruger et system der hedder Boardmaker, som jo ikke er det eneste der findes på markedet, men som er rigtig, rigtigt udbredt indenfor autismeområdet.
Men altså, der kan tales godt og dårligt om dem, de bliver bare rigtigt meget brugt og der er også rigtigt mange elever der lærer dem at kende.
Det er jo så en anden ting, ikke, at de skal lære og forstå, hvad vil det sige, det der ikon som der viser en der for eksempel holder pause, at det ser sådan ud, og så skal jeg gøre sådan.
Hvor at det der med at forstå symboler det er jo lidt abstrakt og faktisk noget vores elever har svært ved, så der er meget udenadslære i det.
Så derfor er det ikke sådan noget man bare lige kan skifte rundt i mellem, men til gengæld tager du nogle af vores normalt begavede unge så synes de jo de er utroligt barnlige og grimme, og det er de jo også, det vil jeg give dem ret i.
Så jeg gjorde meget det i det tidligere projekt, der søgte jeg bare ikonerne på nettet og fandt nogle der var lidt mere pæne, cleane i deres, ja, den visualisering der var lavet. 
Det var mere spiseligt, vil jeg sige.
Det blev der taget godt imod. 
Så det er jo sådan den ene halv-standard man har, det er det Boardmaker der. 

PO:
I forhold til farver, hvordan, udover der er en standard, er det sådan noget de går op i, eller er det bare sådan at det skal ikke ændre sig?

Emil:
Nogle gør.
Nogle går op i det, nogle går ikke op i det.
Det andet, altså hvor kraftige de er i farven er nok mindre vigtigt.
Nu kan du se dem jeg lige har med der, ikke også, de er sådan meget farvefyldte eller farvemættede eller hvad man kan kalde det.
Men om de er mere nedtonede eller sådan, det tror jeg er mindre vigtigt.  
Så man kan godt arbejde med nuancerne tænker jeg, men farverne de er nu som de er.

16.00








