\section{Interview with Emil}\label{interview-with-emil}
Prior to the first sprint a presentation was scheduled with Emil, a representative from Egebakken, a school focused on children with autism.
An interview was conducted with Emil following his presentation, in which preliminary questions were posed in an effort to establish a direction for the project.

\subsubsection{The key points of the interview}
Technology is used extensively at Egebakken, but the relevant areas in which technology is used is in structure and communication as a supporting tool for the students.

\begin{displayquote}
    ``It varies from class to class, depending on the personnel that happens to be in those classes, what experience they have, what kind of students they have and what challenges it can solve in their everyday'' - Emil
\end{displayquote}
Each class can use technological and pedagogical support in the way they see fit.
Some might have a common daily structure on a blackboard, while others have personal calendars.
The division of students in classes is based on many different factors, but the main factors are their age and social abilities.
All students have at least one iPad at their disposal according to Emil, and they are fairly competent in their use.
\\\\
In terms of which areas of the work at Egebakken that would be most beneficial to digitize, Emil pointed out the areas of communication and structure as prime targets for meaningful introductions of digital support.
\begin{displayquote}
    ``I actually think that our greatest challenge is the personnel as users''  - Emil
\end{displayquote}
The employees of Egebakken are diverse, and some are not familiar with the usage of modern technology.
Because of this, the system must deliver a good user experience.
As an extension of this, Emil thinks it would be beneficial if GIRAF had documentation to support the employees in different use cases.
\\\\
In terms of features, Emil sees a lack of support for representing time passing to the citizens, both in GIRAF and other applications they have tried.
Colors are standardised for each weekday and should not be changed heavily, nuances can be modified a little, however.
Some children prefer grayscale, and this should be an option.
Another preference to keep in mind is how many days are shown at a time.
Some children can be overwhelmed by large amounts of activities, while others need to be able to see all days to know that, for example, they do not go to school on Saturdays.
\\\\
Chaos runs rampant at the school whenever the internet goes offline, meaning that being able to run offline is an important requirement of the system.
A crucial thing to keep in mind for GIRAF is that words should not be the main method of conveying information.
Pictograms should serve as the main tool for communication, and the structure and design of the software should be developed such that large amounts of text can be avoided.
\begin{displayquote}
    ``It has to involve as little language as possible, because language is one of the things they struggle with'' - Emil
\end{displayquote}
Egebakken uses a large amount of pictograms, but they do not have one for every possible situation.
As such, Emil would like functionality to search Google for new pictures to use directly and integrate them into the software.
Instead of focusing on adding different applications to the GIRAF software to support different areas, Emil would rather see the week planner improved.
