\section{Interview with Emil}
Information related to the problem is needed in order to steer the project in the right direction.
This information should be gathered by those who will use the system, as they have extensive knowledge within the field that the developers do not.
Two presentations were schedule prior to the first sprint being started, one of them with Emil, a representative from Egebakken, a school focused on children with autism.
An interview was conducted with Emil after his presentation, in which preliminary questions were posed in an effort to establish a direction.
This interview was structured as a semi-structured interview, in which certain pre-defined questions were asked and discussed to gather information.

\subsection{The interview structure}
The interview was conducted with three main areas of focus:
\begin{itemize}
    \item Introductory questions
    \item Program specific
    \item Practical questions 
\end{itemize}
\noindent
The introductory questions served to give an overview of his work at Egebakken, and to determine how Egebakken currently employs IT.
The program specific ones were constructed to explore their needs for the program, and how Emil would like it to perform as the representative of Egebakken.
Finally, the practical questions were posed as a way to finish the interview, in order to find the best methods of reaching Emil for feedback on the GIRAF project in the future.

\subsection{The key points of the introductory questions}
IT permeates Egebakken, but the relevant areas in which IT is used is in structure and communication as a supporting tool for the students.

\begin{quote}
    It varies from class to class, depending on the personnel that happens to be in those classes, what experience they have, what kind of students they have and what challenges it can solve in their everyday - Emil
\end{quote}
IT is used in many different ways, as evidenced in the quote above. Each class can use technological and pedagogical support in the way they see fit.
Some might have a common daily structure on a blackboard, while others have individualised calendars.
The division of students in classes is based on many different factors, but the main ones are age and ability. 
If the student is of regular ability, they are not in the same class as those that struggle.
All students have at least one iPad at their disposal according to Emil, and they are fairly competent in their use.
\\\\
In terms of which areas of the work at Egebakken that would be most beneficial to digitize, Emil responded that much of their work is bound by traditions and can be difficult to change.
He pointed out the areas of communication and structure as prime targets vfor meaningful introductions of digital support, however. 

\begin{quote}
    I actually think that our greatest challenge is the personnel as users, actually  - Emil
\end{quote}

The employees of Egebakken are varied, and some are not familiar with IT.
If you need to support a student in something as important as making sure they thrive and are happy, it is important that all the adults that surround the child can easily make use of the software and support the child.
Because of this, the system must deliver a good user experience - not just taking into account the different preferences and peculiarities of the children, but also that the adults might not be experienced with software of this kind.
As an extension of this, Emil think it would be a boon if GIRAF had documentation to support the users in different use cases.
\\\\
In terms of features, Emil sees a lack of support for representing time passing to the users, both in GIRAF and other applications they have tried.
Keepign track of the passage of time as activities are performed is an important issue for the software.
It is essential for the users, and it should support different representations as the childrens have different preferences.
The different preferences of the children is also a thing to remember for the colors used in the application.
Colors are standardised for each weekday and should not be changed heavily, nuances can be modified a little, however.
Some children also prefer grayscale, and this should be an option.
\\\\
linje 245
\subsection{The key points of the program specific}