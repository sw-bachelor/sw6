\section{Interview with Emil}\label{interview-with-emil}
Information related to the problem is needed in order to steer the project in the right direction.
This information should be gathered by the developers from those who will use the system, as they have extensive knowledge within the field that the developers do not.
Two presentations were scheduled prior to the first sprint being started, the first one being by Emil, a representative from Egebakken, a school focused on children with autism.
An interview was conducted with Emil following his presentation, in which preliminary questions were posed in an effort to establish a direction for the project.
This interview was structured as a semi-structured interview, in which a series of pre-defined questions were asked and discussed to gather information.

\subsection{The interview structure}
The interview was conducted with three main areas of focus:
\begin{itemize}
    \item Introductory questions
    \item Program specific
    \item Practical questions
\end{itemize}
\noindent
The introductory questions served to give an overview of Emil's work at Egebakken, and to determine how Egebakken currently employs IT.
The program specific ones were constructed to explore their needs for the program, and how Emil would like it to perform as the representative of Egebakken.
Finally, the practical questions were posed as a way to finish the interview, in order to find the best methods of reaching Emil for feedback on the GIRAF project in the future.

\subsection{The key points of the introductory questions}
Technology permeates Egebakken, but the relevant areas in which technology is used is in structure and communication as a supporting tool for the students.

\begin{displayquote}
    ``It varies from class to class, depending on the personnel that happens to be in those classes, what experience they have, what kind of students they have and what challenges it can solve in their everyday'' - Emil
\end{displayquote}
Technology is used in many different ways, as expressed in the quote above.
Each class can use technological and pedagogical support in the way they see fit.
Some might have a common daily structure on a blackboard, while others have personal calendars.
The division of students in classes is based on many different factors, but the main factors are their age and social abilities.
If the student is of regular ability, they will usually not be in the same class as those that are not.
All students have at least one iPad at their disposal according to Emil, and they are fairly competent in their use.
\\\\
In terms of which areas of the work at Egebakken that would be most beneficial to digitize, Emil responded that much of their work is bound by traditions and habits and can be difficult to change.
He pointed out the areas of communication and structure as prime targets for meaningful introductions of digital support, however.

\subsection{The key points of the program specific questions}
Following the introductory questions, the interview took a dive into the specifics of the software in relation to the needs of Egebakken.
\begin{displayquote}
    ``I actually think that our greatest challenge is the personnel as users, actually''  - Emil
\end{displayquote}
The employees of Egebakken are diverse, and some are not familiar with the usage of modern technology.
If you need to support a student in something as important as making sure they thrive and are happy, it is important that all the adults that surround the child can easily make use of the software and support the child.
Because of this, the system must deliver a good user experience - not just taking into account the different preferences and peculiarities of the children, but also that the adults might not be experienced with software of this kind.
As an extension of this, Emil thinks it would be beneficial if GIRAF had documentation to support the employees in different use cases.
\\\\
In terms of features, Emil sees a lack of support for representing time passing to the users, both in GIRAF and other applications they have tried.
Keeping track of the passage of time as activities are performed is an important issue for the software.
It is essential for the users, and GIRAF should support different representations as the children have different preferences.
The different preferences of the children is also a thing to remember for the colours used in the application.
Colours are standardised for each weekday and should not be changed heavily, nuances can be modified a little, however.
Some children also prefer grayscale, and this should be an option.
Another preference to keep in mind is how many days are shown at a time.
Some children can be overwhelmed by large amounts of activities, while others need to be able to see all days to know that, for example, they do not go to school on Saturdays.
\\\\
Chaos runs rampant at the school whenever the internet goes offline, meaning that being able to run offline is an essential requirement of the system.
In terms of languages, previous reports on the GIRAF project indicated that some of the customers were interested in having access to multiple languages.
Emil, however, was not convinced of the importance of this functionality.
At Egebakken the children mainly speak Danish, and the ones that do not speak a wide variety of languages.
As such, it would require many translations to facilitate this.
English was the only language of interest, however, as lots of the children spend time with that particular language on the internet.
A crucial thing to keep in mind for GIRAF is that words should not be the main method of conveying information.
Pictograms should serve as the main tool for communication, and the structure and design of the software should be developed such that large amounts of text can be avoided.
\begin{displayquote}
    ``It has to involve as little language as possible, because language is one of the things they struggle with'' - Emil
\end{displayquote}
Egebakken uses a large amount of pictograms, but they do not have one for every possible situation.
As such, Emil would like functionality to search Google for new pictures to use directly and integrate them into the software.
Instead of focusing on adding different applications to the GIRAF software to support different areas, Emil would rather see the week planner improved.
If another application were to be developed to further support the users, Emil would want it to be the category game, a small game in which users are trained in associating different objects.

\subsection{Summary of the first interview with Emil}
Emil is an employee at Egebakken, who would like to see development resources focused on the week planner functionality, making it more stable and usable.
The most important aspect of the week planner is the ability to change it based on the preferences of the child, and this should be facilitated.
Offline functionality is paramount, and documentation to assist in learning the program would be beneficial.
