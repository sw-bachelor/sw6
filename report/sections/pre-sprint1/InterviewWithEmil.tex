\section{Interview with Emil}
Information related to the problem is needed in order to steer the project in the right direction.
This information should be gathered by those who will use the system, as they have extensive knowledge within the field that the developers do not.
Two presentations were schedule prior to the first sprint being started, one of them with Emil, a representative from Egebakken, a school focused on children with autism.
An interview was conducted with Emil after his presentation, in which preliminary questions were posed in an effort to establish a direction.
This interview was structured as a semi-structured interview, in which certain pre-defined questions were asked and discussed to gather information.

\subsection{The interview structure}
The interview was conducted with three main areas of focus:
\begin{itemize}
    \item Introductory questions
    \item Program specific
    \item Practical questions 
\end{itemize}
\noindent
The introductory questions served to give an overview of his work at Egebakken, and to determine how Egebakken currently employs IT.
The program specific ones were constructed to explore their needs for the program, and how Emil would like it to perform as the representative of Egebakken.
Finally, the practical questions were posed as a way to finish the interview, in order to find the best methods of reaching Emil for feedback on the GIRAF project in the future.

\subsection{The key points of the introductory questions}
IT permeates Egebakken, but the relevant areas in which IT is used is in structure and communication as a supporting tool for the students.

\begin{quote}
    It varies from class to class, depending on the personnel that happens to be in those classes, what experience they have, what kind of students they have and what challenges it can solve in their everyday - Emil
\end{quote}
IT is used in many different ways, as evidenced in the quote above. Each class can use technological and pedagogical support in the way they see fit.
Some might have a common daily structure on a blackboard, while others have individualised calendars.
The division of students in classes is based on many different factors. 

\subsection{The key points of the program specific}