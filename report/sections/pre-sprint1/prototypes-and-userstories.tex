\section{The relationship between prototypes and user stories}
The previous two sections, \autoref{pre-sprint-1-user-stories} and \autoref{prototype-comp}, introduced user stories and prototypes respectively.
These two concepts are intrinsically linked.
Initially, we communicate with the customers to determine what functionality to implement. 
For the GIRAF project, we also inherited user stories from the previous year to give a sense of direction.
Once a direction has been found in collaboration with the customers, user stories can be established.
These define the functionality for developers and motivate their inclusion.
Once a user story is established in the format shown in \autoref{pre-sprint-1-user-stories}, it will generally need more information to be properly implemented to the expectations of the customer.
To provide this additional information, prototypes are used.
\\\\
Once the initial user story is established, we design an initial screen, or multiple screens, illustrating how we envision the feature to be implemented.
For example, when making a prototype for Weekplanner\#4 described in \autoref{pre-sprint-1-user-stories}, multiple prototype screens should be made.
The user story states that a guardian should be able to mark activities. 
As such, a prototype showing the activity detail page with a button to mark an activity as completed should be made.
However, a prototype showing the result of clicking the complete button should also be made.
This should be made as visual feedback should be present to ensure that the guardian knows the button has been properly clicked, and since the button might change to represent undoing the completion marker.
\\\\
Once the necessary prototypes have been made, they can be presented to the customer to show our thoughts and have them confirm whether or not the way they are designed is acceptable.
Upon having the prototypes accepted, they can then be added to the description of the user story.
The same prototype might appear in multiple user stories, and one user story can have multiple prototypes illustrating the different possible versions of the screens needed to implement the feature.
This provides a guideline for the developers, they know the approximate design they need to create and can structure their work to reach it.
\\\\
This process repeats every sprint in the following manner:
\begin{itemize}
    \item We communicate with the customer to figure out the most valuable new additions
    \item We define these needs as user stories
    \item We create prototypes for their design
    \item We present the prototypes to the customer
    \item The prototypes are added to the corresponding user stories to guide developers
\end{itemize}
