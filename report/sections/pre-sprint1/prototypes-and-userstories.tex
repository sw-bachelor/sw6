\section{The relationship between prototypes and user stories}
Sections \ref{pre-sprint-1-user-stories} and \ref{prototype-comp} introduced user stories and prototypes respectively.
These two concepts are intrinsically linked.
Initially, we communicated with the customers to determine what functionality to implement. 
Once a direction had been established in collaboration with the customers, user stories could be created.
These defined the functionality for developers and motivated their inclusion.
Once a user story was established in the format shown in \autoref{pre-sprint-1-user-stories}, it would generally need more information to be properly implemented to the expectations of the customer.
To provide this additional information, prototypes were used.
\\\\
Once the user story was formulated, we designed an initial screen, or multiple screens, illustrating how we envisioned the feature to be implemented.
For example, when making a prototype for Weekplanner\#4 described in \autoref{pre-sprint-1-user-stories}, multiple prototype screens should be made.
The user story states that a guardian should be able to mark activities. 
As such, a prototype showing the activity detail page with a button to mark an activity as completed should be made.
However, a prototype showing the result of clicking the complete button should also be made.
This should be made as visual feedback should be present to ensure that the guardian knows the button has been properly clicked.
\\\\
Once the necessary prototypes had been made, they were presented to the customers to show our thoughts and have them confirm whether or not they were acceptable.
When accepted, they can be added to the description of the user story.
The same prototype might appear in multiple user stories, and one user story could have multiple prototypes illustrating the different possible versions of the screens needed to implement the feature.
This provided a guideline for the developers, they knew the approximate design they needed to create and could structure their work to reach it.
\\\\
This process repeated every sprint in the following manner:
\begin{itemize}
    \item We communicated with the customer to figure out the most valuable new additions
    \item We defined these needs as user stories
    \item We created prototypes for their design
    \item We presented the prototypes to the customer
    \item The prototypes were added to the corresponding user stories to guide developers
\end{itemize}
