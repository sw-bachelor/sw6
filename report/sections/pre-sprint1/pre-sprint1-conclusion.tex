\section{Pre sprint 1 conclusion}
In pre sprint 1 we defined preliminary user stories for the first sprint.
These were based on both the user stories received from the previous PO group, as well as the interview with Emil from Egebakken.
During the interview Emil specified that he would like to see development resources focused on the weekplanner functionality, making it more stable and usable.
One of the most important aspect of the weekplanner, according to Emil, was the ability to change it based on the preferences of the child, and this should be facilitated.
He was also of the opinion that offline functionality was important, and that documentation to assist in learning the application would be beneficial.
The user stories that were prepared can be seen in \autoref{User-stories-for-first-sprint}.
\\\\
An overhaul of the prototypes provided by the previous PO group was also necessary.
As we wanted to focus GIRAF 2019 on the frontend of the application, the design was updated.
At the same time we wanted to make the prototypes more interactive and easier to give to the next generation of GIRAF developers.
These prototypes will be continually updated as more functionality is defined for the project or other issues arise. 