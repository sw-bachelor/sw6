\section{Prepared work from previous years}\label{prepared-work-from-previous-years}
Before the first semester wide sprint could start, our group had some work to do as preparation hereof.
First of all, during the readthrough of the reports from previous years, it was discovered that the PO group of 2018 had left us a suggestion for content in the first sprint:

\subsection{Sprint 1}
The first sprint that the previous PO group suggests includes two user stories:
The first user story presents the need for a guardian to be able to mark multiple activities and perform actions on these activities at the same time.
\\\\
The second user story is about a user being able to change the way that an activity is marked as being complete.
This could, for example, be represented by a checkmark, by hiding the activity or by moving the activity a bit to the right on the schedule for the day.
These user stories are suggested for the first sprint because it should be easy for the developers during the start of the project, as they are not familiar with the codebase yet.

\subsection{Sprint 2}
For the second sprint, the previous product owners suggest two user stories, where the first one is that a user should be able to time activities with a timer.
This is needed so the citizens know how much time their activities take.
The guardian should be able to add this timer with a specific time and connect it to the activity, after which the citizen should then be the one that starts the timer.
\\\\
The second user story concerns a feature for guardians that allows them to choose between a set of visual representations for the timers that they can add for the activities.
This is needed because the citizens have different preferences when it comes to representation of the time.
\\\\
Just as the first suggested sprint, this suggested sprint is meant to be light in workload to allow the developers to get familiar with the codebase, and to make sure that all groups can finish their tasks for the sprint.

\subsection{Design guide}
A design guide was available, made by the previous years.
However, this guide seemed to be last updated in 2015 and seemed to never have been properly used for implementation.
This design guide was located on a decentralized location, meaning not many of the developers were even aware of its existence.
It was also around 80 pages long, meaning it could be quite intimidating to search through and to follow.

\subsection{Prototypes}
A set of prototypes was constructed by the previous year.
The available prototypes were made by putting images into a PowerPoint presentation and making clickable areas to navigate through them.
This has resulted in a single set of prototypes consisting of 122 slides, which could not be edited by other means than replacing a given element in every single slide, even for the slides with very little variation.



