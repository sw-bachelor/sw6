\section{Prepared work from previous years}\label{prepared-work-from-previous-years}
Before the first sprint could start, our group had some work to do as preparation hereof.
First of all, during the readthrough of the reports from previous years, it was discovered that the PO group of 2018 had left us a suggestion for content in the first sprint.
The following resources were available to us upon the start of the project:
\begin{itemize}
    \item User stories for the first sprint
    \item Prototypes for the GIRAF application
    \item A design guide for the interface of the application
\end{itemize}
The previous year produced two releases in each sprint.
As such, user stories were prepared for two releases.
For our development process we only released once per sprint, so these user stories would be incorporated into only the first sprint.
They had also prepared prototypes to illustrate the vision of the project.
A design guide from 2015 detailing the foundation of design for GIRAF was also available. 

\subsubsection{Suggestion for the first release by the previous PO group}
The first release that the previous PO group suggested included two user stories.
The first user story presented the need for a guardian to be able to mark multiple activities and perform actions on these activities.
\\\\
The second user story was about a user being able to change the way that an activity was marked as being completed.
This could, for example, be represented by a checkmark, by hiding the activity or by moving the activity a bit to the right on the schedule for the day.

\subsubsection{Suggestion for the second release by the previous PO group}
For the second release, the previous PO group suggested another two user stories.
The first one was that a user should be able to time activities with a timer.
This would be beneficial for the citizens to know how much time their activities would take.
The guardian should be able to add this timer with a specific duration and connect it to the activity, after which the citizen should then be the one that starts the timer.
\\\\
The second user story concerned a feature for guardians that allowed them to choose between a set of visual representations for the timers that they could add to the activities.
This was needed because the citizens had different preferences when it came to the representation of time.


\subsubsection{Design guide}
A design guide was available, made by an earlier GIRAF group.
However, this guide seemed to be last updated in 2015 and seemed to never have been properly used for implementation.
This design guide was located on a subpage on Phabricator, meaning not many of the developers were even aware of its existence.
It was also around 80 pages long, meaning it could be quite intimidating to search through and to follow.

\subsubsection{Prototypes}
A set of prototypes was constructed by the previous year.
The available prototypes were made by putting images into a PowerPoint presentation and making clickable areas to navigate through them.
This resulted in a single set of prototypes consisting of 122 slides, which could not be edited by other means than replacing a given element in every single slide.


\subsubsection{Conclusion on the prepared work}
Two user stories for each release was not enough work for the 7 groups on GIRAF 2019. 
As previous years had dedicated back end, front end and server groups, a user story was often split into multiple tasks.
These tasks could then be assigned to the dedicated groups.
We worked in full stack groups, and because of this a group was able to take one or more user stories for each sprint without splitting the user story into multiple tasks.
Because of this, we needed more user stories rather than just the user stories that they had prepared for us.
The user stories that were prepared were still useful and were put into the backlog, but they were not the foundation of the starting sprint.
The design guide was not entirely up to date, not completely finished and too long to effectively use.
As such, it was not all that useful for the GIRAF project in 2019.
The prototypes made by the previous PO group were functional, but their design and usability could be improved.
PowerPoint also did not seem like a very efficient tool for the purpose due to its inflexibility. 


