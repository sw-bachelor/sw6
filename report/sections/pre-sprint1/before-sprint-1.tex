\section{Before sprint 1}
Before the first semester wide sprint could start, our group had some work to do as preparation hereof.
First of all, during the readthrough of the reports from previous years, it was discovered that the PO group of 2018 had left us a suggestion for content in the first sprint:

\subsection*{Suggested sprint 1}
The first sprint that the previous PO group suggests includes two user stories:
The first user story presents the need for a guardian to be able to mark multiple activities and perform actions on these activities at the same time.
\\\\
The second user story is about a user being able to change the way that an activity is marked as being complete.
This could, for example, be represented by a checkmark, by hiding the activity or by moving the activity a bit to the right on the schedule for the day.
These user stories are suggested for the first sprint because it should be easy for the developers during the start of the project, as they are not familiar with the codebase yet.

\subsection*{Suggested sprint 2}
For the second sprint, the previous product owners suggest two user stories, where the first one is that a user should be able to time activities with a timer.
This is needed so the citizens know how much time their activities take.
The guardian should be able to add this timer with a specific time and connect it to the activity, after which the citizen should then be the one that starts the timer.
\\\\
The second user story concerns a feature for guardians that allows them to choose between a set of visual representations for the timers that they can add for the activities.
This is needed because the citizens have different preferences when it comes to representation of the time.
\\\\
Just as the first suggested sprint, this suggested sprint is meant to be light in workload to allow the developers to get familiar with the codebase, and to make sure that all groups can finish their tasks for the sprint.

\subsection{User stories}
A major part of the work as a PO group is to define \texttt{user stories} to describe a feature and why it is wanted in the simple format of: As a (user type) I would like (feature) so that (reason).
The development team should then be able to transform these user stories into technical requirements that they can distribute within their own group.
Based on the suggested sprints from the previous PO group and the interview with Emil, we defined a series of user stories with suggested features, which primarily focused on increasing usability and the user experience of the application.
For the first sprint, the following user stories were included, in order of importance:
% TODO: Overvej om alle de her skal med i den endelige rapport eller smides i appendix
\begin{itemize}
    \item As a citizen I want a time timer so that I know how long is left of my current activity
    \item As a user I would like the icons to be updated so that they are modern and easy to understand
    \item As a citizen I would like to be able to choose how many days I see at a time on my weekplanner, so that it fits my personal preference
    \item As a guardian, I would like that the app is fully available offline so that I can still use it if the internet is down
    \item As a guardian I would like the user selection screen to look better so it makes it easier for me to find the correct user
    \item As a guardian, I would like to be able to mark activity(s)
    \item As a guardian I would like to confirm with a password that the system is changing to guardian mode so that a citizen cannot gain access to it
    \item As a guardian I would like to be able to see results as I'm typing the name of a pictogram so that I can see if there are any results instead of just seeing my keyboard
    \item As a guardian, I would like to be able to copy the content of a plan in the weekplanner from one user to another so that I won't have to do it twice, if two citizens have the same schedule
    \item As a guardian, I would like a way to add pictograms directly from google so that I can quickly improvise if the system does not have the activity I want
    \item As a citizen I would like to disable colors in the app so that it does not overstimulate me
    \item As a citizen I would like the ability to choose how my day is represented (horizontally or vertically) so that it fits my personal preference
    \item As a guardian I would like a better login screen so that the application is more appealing
    \item As a citizen I would like different ways to mark an activity as done so that it fits my personal preference
    \item As a guardian I would like guides available for the system so that it is easy to look up the features that I don't fully understand how to use
    \item As a citizen I would like the icons to be consistent throughout the system so that I instinctively know their meaning
\end{itemize}

\subsection{Design guide}
A design guide was available, made by the previous years.
However, this guide seemed to be last updated in 2015 and seemed to never have been properly used for implementation.
So in order to ensure that the guide is up to date, we decided to start working on a renewed version of the design guide, which should be available in the github wiki instead of as a separate pdf document.
The changes that are being imagined for the new design guide are, first of all, a set of rules for ensuring the user experience of the application by updating the icons, and to make the application seem less like a special needs tool, as this has been requested in the interview with Emil.

\subsection{Producing Prototypes in Adobe XD}
In addition to updating the design guide, we decided to update the prototypes to a more suitable program type.
The currently available prototypes are made by putting images into a PowerPoint presentation and making clickable areas to navigate through them.
This has resulted in prototype consisting of 122 slides, which could not be edited by other means than replacing a given element in every single slide.
By changing this to Adobe XD, it is possible to mark a part of the prototype as a symbol, and by changing this symbol in one place it will be replaced in all aspects of the prototype.
This allows for easier updates of the design in comparison to the prototypes made in PowerPoint.
