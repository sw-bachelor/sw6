\section{Prepared work from previous years}\label{prepared-work-from-previous-years}
Before the first semester wide sprint could start, our group had some work to do as preparation hereof.
First of all, during the readthrough of the reports from previous years, it was discovered that the PO group of 2018 had left us a suggestion for content in the first sprint:

\subsection{Sprint 1}
The first sprint that the previous PO group suggests includes two user stories:
The first user story presents the need for a guardian to be able to mark multiple activities and perform actions on these activities at the same time.
\\\\
The second user story is about a user being able to change the way that an activity is marked as being complete.
This could, for example, be represented by a checkmark, by hiding the activity or by moving the activity a bit to the right on the schedule for the day.
These user stories are suggested for the first sprint because it should be easy for the developers during the start of the project, as they are not familiar with the codebase yet.

\subsection{Sprint 2}
For the second sprint, the previous product owners suggest two user stories, where the first one is that a user should be able to time activities with a timer.
This is needed so the citizens know how much time their activities take.
The guardian should be able to add this timer with a specific time and connect it to the activity, after which the citizen should then be the one that starts the timer.
\\\\
The second user story concerns a feature for guardians that allows them to choose between a set of visual representations for the timers that they can add for the activities.
This is needed because the citizens have different preferences when it comes to representation of the time.
\\\\
Just as the first suggested sprint, this suggested sprint is meant to be light in workload to allow the developers to get familiar with the codebase, and to make sure that all groups can finish their tasks for the sprint.

\subsection{Conclusion on the prepared work}
Two user stories for each sprint is not enough work for 7 groups. 
As previous years had dedicated backend, frontend, and server groups, a user story often had to be split into tasks for multiple groups.
The user stories are not large enough to be given to multiple groups.
This year we decided to work in full stack groups, and hereby a group is able to take one or more user stories for each sprint and then only work on the assigned user stories.
However, the user stories that were prepared are still useful and they have been put into the backlog along with others.

\subsection{Design guide}
A design guide was available, made by the previous years.
However, this guide seemed to be last updated in 2015 and seemed to never have been properly used for implementation.
So in order to ensure that the guide is up to date, we decided to start working on a renewed version of the design guide, which should be available in the github wiki instead of as a separate pdf document.
The changes that are being imagined for the new design guide are, first of all, a set of rules for ensuring the user experience of the application by updating the icons, and to make the application seem less like a special needs tool, as this has been requested in the interview with Emil.

\subsection{Producing Prototypes in Adobe XD}
In addition to updating the design guide, we decided to update the prototypes to a more suitable program type.
The currently available prototypes are made by putting images into a PowerPoint presentation and making clickable areas to navigate through them.
This has resulted in prototype consisting of 122 slides, which could not be edited by other means than replacing a given element in every single slide.
By changing this to Adobe XD, it is possible to mark a part of the prototype as a symbol, and by changing this symbol in one place it will be replaced in all aspects of the prototype.
This allows for easier updates of the design in comparison to the prototypes made in PowerPoint.
