\section{Sprint retrospective}
This sprint retrospective was different than the ones conducted for the previous sprints.
As the first thing all GIRAF groups shortly debated how the process for the sprint was for each group.
After this debate everyone split into new groups to discuss their thoughts with people from other groups.
Everything that was discussed was added to a document, and after the meeting the process group created a survey to get everyone's opinion on the subjects.
In the previous retrospectives we used dotstorm, and were only able to vote on 3 things that we agreed on.
This meant that people would only agree upon the most important things, even though there might be smaller less important things that were useful.
\\\\
For each statement or idea there were 3 possible answers which were the following:
\begin{itemize}
    \item It is a good idea / I agree
    \item I do not care
    \item It is not a good idea / I disagree
\end{itemize}
As a result of the survey the process group decided to to focus on updating multiple aspects of our process, some of which affected us:
\begin{itemize}
    \item Stand up meetings should focus more on explaining what the different groups' user stories are about. The purpose of this is to find dependencies between groups.
    \item If you are assigned a user story that is blocked, then comment on GitHub which user story it is blocked by and unassign your group from the user story.
    \item One pull request for each user story unless two user stories overlap.
    \item Contact the PO group if you have nothing to do.
    \item Tell the PO group if your user story is not able to be completed within the following sprint.
\end{itemize}
\noindent
At the retrospective, we got an impression that it was somewhat expected of us to try and assign user stories to people in a way such that they would have as few merge conflicts and dependencies as possible.
However, with the main part of the application being the \texttt{show weekplan} screen, it was often difficult to avoid. 
To help the developers together, it was decided that at each stand up, they would mention which part of the system they are working on, to give them an indicator for whether they were working on the same part of the application as another group.
\\\\
Likewise, we experienced that some groups got stuck because they were working on a part of the application that depended on the work of another group.
This would sometimes cause the group to just sit around and do nothing as long as their story was blocked.
In the upcoming sprints, the groups are encouraged to come in and talk to us if they have nothing to do, since we have a big enough backlog to find them something that is not blocked.
\\\\
Finally, in each sprint we experienced that some groups did not finish their user stories, but we did not know about this until the sprint preparation.
They were encouraged to contact us if they were stuck, so that we could get developers from other team to help them.
\\\\
