\section{Sprint retrospective}
This sprint retrospective was different than the ones conducted for the previous sprints.
As the first thing all giraf groups shortly debated how the process for the sprint was for each group.
After this debate everyone split into new groups to discuss their thoughts with people from other groups.
Everything that was discussed was added to a document, and after the meeting the process group created a survey to get everyone's opinion on the subjects.
In the previous retrospectives we used dotstorm, and were only able to vote on 3 things that we agreed on.
This meant that people would only agree upon the most important things, even though there might be smaller less important things that were useful.
\\\\
For each statement or idea there were 3 possible answers which were the following:
\begin{itemize}
    \item It is a good idea / I agree
    \item I do not care
    \item It is not a good idea / I disagree
\end{itemize}
As a result of the survey the process group decided to change the following aspects: 
\begin{itemize}
    \item Stand up meetings should focus more on explaining what the different groups' user stories are about. The purpose of this is to find dependencies between groups.
    \item If you are assigned a user story that is blocked, then comment on GitHub which user story it is blocked by and unassign your group from the user story.
    \item One pull request for each user story unless two user stories overlap.
    \item Read through the entire pull request every time you review.
    \item You should aim to review the pull request that you are assigned to on the same day that you get assigned. If that is not possible, then contact the pull request owner.
    \item Be more active on Slack. Check it at least once a day.
    \item Contact the PO group if you have nothing to do.
    \item Tell the PO group if your user story is not able to be completed within the following sprint.
\end{itemize}
