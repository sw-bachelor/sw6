\section{Release 2019S3R1}
For the release in this sprint, all groups had a release preparation period for the last two days of the sprint. 
When the release preparation started, we expected that every user story in the sprint would be finished and merged into develop by the end of the preparation period. 
The ones that were not ready by then, would not be included in the release.
The user stories that were not completed it in time for the release can be seen in \autoref{table:unfinished-user-stories-sprint-3}.

\begin{table}[H]
    \small
    \begin{tabular}{|p{3.5cm}|p{9cm}|}
    \hline
    Issue ID        & User story   \\ \hline
    Weekplanner\#57  & As a guardian I would like to be able to remove an activity from the week plan so that cancelled activities can be removed from the citizens plan \\ \hline
    Weekplanner\#155 & As a guardian I would like to see the icons of a citizen on the choose citizen screen so that I can quickly identify them \\ \hline
    Weekplanner\#157 & As a user I want to be prompted an error message when failing to login correctly during a change from guardian to citizen \\ \hline
    \end{tabular}
    \caption{User stories that did not finish in time for the release}\label{table:unfinished-user-stories-sprint-3}
\end{table}

For the preparation, all the groups met up and systematically went through each new feature to see if they worked as expected. We did this as a result of \autoref{sec:sprint-2-retrospective} to make communication a lot smoother during release preparation.
The way it was done was by having each group review a couple of user stories that they had not developed themselves and check for bugs or inconsistent design. 
We created a checklist that the groups could use when testing the features:
\begin{itemize}
    \item Can the screen be reached through navigation in the application?
    \item Can you perform all the functionality defined in the issue?
    \item Can it be used without crashing?
    \item Does it run without bugs?
    \item Does it still look acceptable if you change to a new device or change orientation?
\end{itemize}
Bugs that were found was reported on Github by the groups. 
We, the PO group, then assigned other groups to solve these bugs.
The groups then solved the bugs by branching out from the release branch and merging it back into the release branch when they had solved the bug. 
We still had two code reviewers and a PO reviewer for each pull request during release. 
We tried to be highly available during the release so that if the groups had any questions regarding design or functionality we would be there to answer them.
This was one of the benefits of having a release preparation together in the same room.
\\\\
When the most critical bugs had been fixed, we held a release party where each group had a chance to present the functionality that they had been working on during the sprint. 
