\section{Our assigned user stories}
As described in \autoref{sprint-3-po-goals}, we had the goal of implementing the following three user stories.
\begin{itemize}\label{item:our-stories-sprint-3}
    \item Weekplanner\#133 - "As a guardian I would like to be able to logout when I'm on the choose citizen page so that I can return to the login page"
    \item Weekplanner\#92 - "As a developer I would like the application to use a default font so that I can easily change it"
    \item Weekplanner\#125 - "Exception thrown when rendering WeekPlanScreen, issue \#44"
\end{itemize}

\subsection{Weekplanner\#133}
The initial user story seemed simple to implement.
It just defined that the guardian should have the ability to logout when on the choose citizen page.
This evolved however after an initial pull request was created.
Based on feedback from the development groups and our workload, we decided to extend the issue to be the creation of a easily implementable navigation bar.
This navigation bar should have a default state, and allow for developers to include arguments defining which icons should be on the bar.

\begin{lstlisting}[caption={Building the appbar},label={lst:appbarbuild},language={[Sharp]C}]
  @override
  Widget build(BuildContext context) {
    toolbarBloc.updateIcons(appBarIcons, context);
    return AppBar(
        title: Text(title),
        backgroundColor: const Color(0xAAFF6600),
        actions: <Widget>[
          StreamBuilder<List<IconButton>>(
              initialData: const <IconButton>[],
              key: const Key('streambuilderVisibility'),
              stream: toolbarBloc.visibleButtons,
              builder: (BuildContext context, 
                AsyncSnapshot<List<IconButton>> snapshot) {
                return Row(
                  children: snapshot.data
                );
              }),
        ]);
  }
\end{lstlisting}
\autoref{lst:appbarbuild} shows the way we created the application bar widget.
We cleaned up the initial implementation of it, moving all the code related to instantiating icons to the \texttt{toolbar BLoC}.
\autoref{lst:appbarbuild} receives a context from which it updates the icons through the \texttt{updateIcons} method in the toolbar BLoc, which can be seen in \autoref{lst:updateicons}.
Within the \texttt{AppBar} widget that is returned, the icons are defined through the list of \texttt{IconButtons} on line 8.
\texttt{IconButton} is an enum we created defining all possible icons.
The order in which one inputs the arguments when instantiating the appbar determines the order in which the icons are rendered on the page. 

\begin{lstlisting}[caption={Updating the icons in the appbar},label={lst:updateicons},language={[Sharp]C}]
  void updateIcons(List<AppBarIcon> icons, BuildContext context) {
  List<IconButton> _iconsToAdd;
  _iconsToAdd = <IconButton>[];

  if (icons == null) {
    icons = <AppBarIcon>[];
    icons.add(AppBarIcon.settings);
    icons.add(AppBarIcon.logout);
  }
  for (AppBarIcon icon in icons) {
    _addIconButton(_iconsToAdd, icon, context);
  }

  final BehaviorSubject<List<IconButton>> iconList =
      BehaviorSubject<List<IconButton>>.seeded(<IconButton>[]);
  iconList.add(_iconsToAdd);
  _visibleButtons = iconList;
}
\end{lstlisting}
\autoref{lst:updateicons} shows the method used in the \texttt{toolbarBloc}.
It takes a list of icons and a context as input.
It instantiates a list of the enum \texttt{IconButton}.
If the input list of icons is null, no arguments have been given to the appbar, meaning it should be created with the standard icons, settings and logout.
If any icons were defined, it uses the \texttt{\_addIconButton} method to add it.
Finally a new \texttt{BehaviorSubject} is created and the list of now instantiated buttons are added to it.
This is required for the appbar to update correctly whenever the icons that are supposed to be rendered change.

\begin{lstlisting}[caption={Testing the appbar with certain icons},label={lst:appbaricontest},language={[Sharp]C}]
  testWidgets('Change to citizen button is displayed',
  (WidgetTester tester) async {
  final GirafAppBar girafAppBar = GirafAppBar(
    title: 'Ugeplan',
    appBarIcons: const <AppBarIcon>[AppBarIcon.changeToCitizen]);
  
  await tester.pumpWidget(makeTestableWidget(child: girafAppBar));
  await tester.pump();
  
  expect(find.byTooltip('Skift til borger tilstand'), findsOneWidget);
  });
  
\end{lstlisting}
To ensure the quality of the code the GIRAF project requires tests to be written.
\autoref{lst:appbaricontest} shows an example of one of the test constructed to achieve satisfactory coverage.
This specific test tests that the \textit{Change to citizen} button is displayed correctly on the appbar when it is provided as an argument.
This is done by instantiating a test widget of the appbar with the \textit{Change to citizen} icon on lines 2-7.
Line 7 uses the \texttt{pumpWidget} method to render the UI of the given widget.
Line 8 employs the \texttt{pump} method to rebuild the widget.
If a button is pressed to change state, for example, Flutter will not automatically rebuild the environment to reflect the change.
As such, we use the \texttt{pump} method to ensure that the environment is as expected.
Finally we search for the tooltip of the icon to ensure it exists in the environment.

\subsection{Weekplanner\#92}
This user story was relatively simple.
After discussing the issue of fonts with the customers, we were made aware that they wanted a font as close to text regularly written by humans as possible.
This required a change of font.
To make a new font the default used by the application, we added it to the theme settings when the application is initially run:

\begin{lstlisting}[caption={An excerpt of the run method},label={lst:fontfamily},language={[Sharp]C}]
  void _runApp() {
  runApp(MaterialApp(
      title: 'Weekplanner',
      theme: ThemeData(fontFamily: 'Quicksand'),
\end{lstlisting}
\autoref{lst:fontfamily} shows the beginning of the \texttt{\_runApp} method.
Changing the default font was done by adding the \texttt{fontFamily} to the theme.
The font family was defined in the \texttt{pubspec} file, which is where packages are specified in Flutter.
\begin{lstlisting}[caption={An excerpt of the pubspec file. Multiple assets are in the file for different styles, such as regular font.},label={lst:fontpubspec},language={[Sharp]C}]
  fonts:
  - family: 'Quicksand'
    fonts:
      - asset: assets/fonts/Quicksand-Bold.ttf
\end{lstlisting}
\subsection{Weekplanner\#125}
This issue related to the main screen of the weekplanner application.
Whenever it was opened, an exception would be thrown.
We deduced that it was related to the pictogram images. 
When the application would get these the exception would be thrown.
The issue turned out to be related to a custom widget called \texttt{PictogramImage}.
When we used the \texttt{PictogramImage} widget, we could not define a width or height.
When these were undefined, it would default to 0 and throw the exception.
To change this, we swapped from using the \texttt{PictogramImage} custom widget.

\begin{lstlisting}[caption={An excerpt showing the use of FittedBox instead of PictogramImage},label={lst:exceptionfix},language={[Sharp]C}]
  child: Card(
    child: FittedBox(
      child: Stack(
        alignment: AlignmentDirectional.center,
        children: <Widget>[
          SizedBox(
            width: MediaQuery.of(context).size.width,
            height: MediaQuery.of(context).size.width,
            child: FittedBox(
              child: _getPictogram(activities[index]),
            ),
          ),
          Icon(
            Icons.check,
            key: const Key('IconComplete'),
            size: MediaQuery.of(context).size.width,
            color: Colors.green,
          ),
        ],
      ),
    ),
\end{lstlisting}
As can be seen on line 2 of \autoref{lst:exceptionfix} we swapped to using a \texttt{FittedBox} widget. This widget lets us define a child widget called \texttt{SizedBox} on line 6, in which we can properly define the size of the box. This avoids the exception.
We also updated the method for accessing the details of a pictogram.
This method was changed to get the data from the \texttt{PictogramImage} to ensure it was properly retrieved even if they were shown through the \texttt{FittedBox} widgets on the week plan screen.
This method can be seen on \autoref{lst:getPictogram}.
\begin{lstlisting}[caption={The getPictogram method to retrieve the details of a pictogram},label={lst:getPictogram},language={[Sharp]C}]
  Widget _getPictogram(ActivityModel activity) {
    final PictogramImageBloc bloc = di.getDependency<PictogramImageBloc>();
    bloc.loadPictogramById(activity.pictogram.id);
    return StreamBuilder<Image>(
      stream: bloc.image,
      builder: (BuildContext context, AsyncSnapshot<Image> snapshot) {
        if (snapshot.data == null) {
          return const CircularProgressIndicator();
        }
        return Container(
          child: snapshot.data, 
          key: const Key('PictogramImage'));
      },
    );
  }
\end{lstlisting}
\autoref{lst:getPictogram} makes use of the \texttt{BLoC} created for pictograms to load one by its id.
It then returns the corresponding data as a \texttt{container} widget or shows a progress indicator that the data retrieval is in progress.