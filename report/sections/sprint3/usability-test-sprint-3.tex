\section{Usability test May 2nd}\label{usability-test-sprint-3}
On the 2nd of May we held a usability test for the customers of the GIRAF project.
Due to our experiences from the previous sprint, we sent an email to the participating customers in the project on March 19th with the date, sent reminders and called the customers to confirm their participation as the date approached.
No representatives from the kindergarten Birken could be present at this test, but Emil from the school Egebakken was available.

\subsection{The usability test}
The purpose of the usability test was to have Emil perform a certain set of tasks.
These tasks would be performed mostly on his own without our help, and he would narrate his thoughts as he performed them.
This was done to generate an impression of how intuitive the application was, and how easily he would be able to complete these fundamental tasks.
We provided the required information for tasks that needed it, such as username and password when logging in. 
The tasks can be seen on \autoref{table:usability_tasks}.
\begin{table}[H]
    \small
    \begin{tabular}{|p{1.3cm}|p{12cm}|}
    \hline
    Task number      &Task                                                                                                                \\ \hline
    1 & Log into the application                                                                                                           \\ \hline
    2 & Navigate to a certain citizen's week plan and determine the activities they would do on a certain day                              \\ \hline
    3 & Add an activity that symbolizes going to the toilet to this citizen's week plan on Monday                                           \\ \hline
    4 & Move the newly added activity to Thursday, between the breakfast and playground activities                                          \\ \hline
    5 & Mark the newly added activity as completed                                                                                             \\ \hline
    6 & Determine the activities another citizen has to partake in on the Wednesday in a certain week plan connected to this new citizen      \\ \hline
    7 & Make a new week plan for this citizen                                                                                                \\ \hline
    8 & Add a normal plan for a regular Monday as you might do at your job right now                                                            \\ \hline
    9 & Log out                                                                                                                                \\ \hline
    \end{tabular}
    \caption{Tasks for the usability test and the time it took to complete the tasks.}\label{table:usability_tasks}
\end{table}

\noindent
Emil did generally not struggle with these assignments.
He made a mistake during assignment 4 by pressing the activity once, opening the details page, rather than doing a \textit{long press} and dragging the activity.
However, when narrating his thoughts as the assignment was being performed, he said that he expected to complete this assignment by performing a \textit{long press}.
\\\\
Assignment 7 caused the most issues. 
The screen to add a new week plan slightly confused him.
When adding a new week plan, the application presents the option to add a blank week plan or use a template.
This confused him a bit, as the assignment did not specify which he should pick.
After resolving the issue, he was confused about the image that has to be chosen for a new week plan.
This image is just used as an illustration for the week plan, but he thought it might be seen as a pictogram causing confusion.
To solve these issues we removed the template button as it did not have any functionality, and by adding error messages.

\subsection{Clarifying questions}
Upon completing the test we posed some clarifying questions to examine his opinion of the application.
The overall design was intuitive according to Emil.
He felt the assignments were easily completed because of the intuitiveness of the program.
In terms of improvements he would like to see changes to the drag and drop functionality of moving activities around.
He felt it was hard to determine where the activity would land if it were to be dragged between two other activities.
A visual representation of where it would end up, such as a line indicating it, would be a solution to this problem.
\\\\
Choice boards were considered for implementation by the previous year on the GIRAF project.
These boards should allow for multiple activities to be added, and then the citizen would select one of them.
Emil thought this would be important to teach the citizens that they also have an influence, rather than just having them do what they are told all the time.
As such, this could be a feature to implement in the future as it was still deemed attractive.
\\\\
In relation to the timer functionality, we were uncertain whether the citizen or the guardian would be the one to change the visual representation.
Emil said that this should be done by guardians.
Once started, whether or not a citizen should be able to pause a timer was posed as a question.
He believed that both options had their merits. 
As such, he would like for a type of lock to be implemented on the timer.
The guardian should be able to choose that the timer should be locked when adding it.

\subsection{Prototype feedback}
After the test and the questions, we presented the updated and new prototypes from the sprint.
These can be seen in \autoref{sprint-3-prototypes}.
Emil thought that they all looked adequate and simple.
As we made two prototypes for copying multiple activities to other days, we asked him which one he preferred.
These prototypes can be seen on \autoref{fig:mark_mode_copy_dropdown_and_checkbox}.
He preferred \autoref{fig:mark_mode_copy_checkbox}, as it was easier to copy to multiple days.
The prototype for adding new pictograms from the phone gallery generated a new discussion.
Emil thought it would be useful to have a priority ranking system on pictograms when one searches for them.
You might have a large amount of pictograms, but realistically a core few would be heavily used. 
As such, he would like to see a priority list in searches based on the amount of times pictograms had been used.
