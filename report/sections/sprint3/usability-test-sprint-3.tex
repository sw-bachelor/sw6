\section{Usability test May 2nd}\label{usability-test-sprint-3}
On the 2nd of May we held a usability test for the customers of the GIRAF project.
This was done to garner information relating to the development of the project.
We sent an email to the participating customers in the project on March 19th with the date, sent reminders and called the customers to confirm their participation as the date approached.
No representatives from the kindergarten Birken could be present at this test, but Emil from the school Egebakken was available.

\subsection{The usability test}
The purpose of the usability test was to have Emil perform a certain set of tasks.
These tasks would be performed mostly on his own without our help, and he would narrate his thoughts as he performed them.
This was done to generate an impression of how intuitive the program was, and how easily he would be able to complete these fundamental tasks.
We provided the required information for tasks that needed it, such as username and password when logging in. 
The tasks can be seen on \autoref{table:usability_tasks}.
\begin{table}[H]
    \small
    \begin{tabular}{|p{1.3cm}|p{10cm}|p{1.8cm}|}
    \hline
    Task number      &Task                                                                                                                & Time      \\ \hline
    1 & Log into the system                                                                                                               & 15 seconds  \\ \hline
    2 & Navigate to a certain citizen's week plan and determine the activities they would do on a certain day                              & 15 seconds  \\ \hline
    3 & Add an activity that symbolizes going to the toilet to this citizen's week plan on Monday                                          &  15 seconds \\ \hline
    4 & Move the newly added activity to Thursday, between the breakfast and playground activitites                                        & 30 seconds   \\ \hline
    5 & Mark the newly added activity as completed                                                                                            & 12 seconds   \\ \hline
    6 & Determine the activities another citizen has to partake in on the Wednesday in a certain week plan connected to this new citizen     & 17 seconds   \\ \hline
    7 & Make a new week plan for this citizen                                                                                                & 1 minute   \\ \hline
    8 & Add a normal plan for a regular Monday as you might do at your job right now                                                          & 3 minutes   \\ \hline
    9 & Log out                                                                                                                              & 30 seconds   \\ \hline
    \end{tabular}
    \caption{Tasks for the usability test and the time it took to complete the tasks.}\label{table:usability_tasks}
\end{table}

\noindent
Emil did generally not struggle with these assignments.
Most of the issues took around 15 seconds, with only assignments 4, 7, 8 and 9 deviating from this.
\\
Assignment 4 took him 30 seconds to complete.
He made a mistake by pressing the activity once, opening the details page, rather than doing a \textit{long press} and dragging the activity.
However, when narrating his thoughts as the assignment was being performed, he said that he expected to complete this assignment by performing a \textit{long press}.
\\\\
Assignment 7 caused the most issues. 
He initially did it correctly, navigating to the proper place to add a week plan. 
The screen to add a new week plan slightly confused him, though.
When adding a new week plan, the system presents the option to add a blank week plan or use a template.
This confused him a bit, as the assignment did not specify which he should pick.
After resolving the issue, he was confused about the image that has to be chosen for a new week plan.
This image is just used as an illustration for the week plan, but he thought it might be seen as a pictogram causing confusion.
After this hurdle, he performed the task without issues.
\\\\
For assignment 8, he did not have any major problems with completing his tasks even though it took around 3 minutes to complete.
He did however have some issues finding the specific pictograms that he wanted.
If he was not able to find a pictogram that he thought look appopiate, he was good at searching on another keyword to get another result. 
An example could be that he searched for iPad, where there were no good results, but then he quickly changed the search to tablet, and found a fitting pictogram.
He added 18 pictograms to monday, and explained that 18-20 pictograms probably would be around the amount of pictograms that would be in a day.
He also told us, that the following days in the week typically would be very identical only with small adjustments.
\\\\
For assignment 9, he simply did not notice the log out button initially, causing it to take a little longer.

\subsection{Clarifying questions}
Upon completing the test we posed some clarifying questions to examine his opinion of the program.
The overall design was intuitive according to Emil. He thought it was relatively simple, and not too busy so as to be overwhelming for the citizens.
He felt the assignments were easily completed because of the intuitiveness of the program, and for the assignments that caused trouble he mainly blamed the issues on it being the first time he used the program, rather than it being poorly implemented.
In terms of improvements he would like to see changes to the drag and drop functionality of moving activities around.
He felt it was hard to determine where the activity would land if it were to be dragged between two other activities.
A visual representation of where it would end up, such as a line indicating it, would therefore be preferable.
\\\\
Because we were uncertain, we asked Emil how many citizens he or his colleagues would generally create week plans for.
He responded by saying that, at the moment, he only has one citizen, but the absolute maximum at Egebakken would be four.
Choice boards were considered for implementation by the previous year on the GIRAF project.
These boards should allow for multiple activities to be added, and then the citizen would select one of them.
Emil thinks this would be important to teach the citizens that they also have an influence, rather than just having them do what they are told all the time.
As such, this could be a feature to implement in the future as it was still deemed attractive.
\\\\
Emil was initially a proponent for the implementation of a timer functionality for the GIRAF project, as seen in \autoref{interview-with-emil}.
We were uncertain whether the citizen or the guardian would be the one to change representation shown for the citizen.
Emil thought that this should generally be done by guardians.
The citizens can have issues in telling time intervals apart, and choosing the proper representation would require some knowledge that only the guardian would have.
Once started, whether or not a citizen should be able to pause a timer was posed as a question.
He thought that both options had their merits. 
As such, he would like for a type of lock to be implemented on the timer.
The guardian would be able to choose that the timer should be locked when adding it, making it unable to be paused.

\subsection{Prototype feedback}
After the test and the questions, we presented the updated and new prototypes from the sprint.
These can be seen in \autoref{sprint-3-prototypes}.
Emil thought that they all looked adequate and simple, and did not have many comments in regards to improvements.
None of the prototypes were denied or described as unacceptable.
As we made two protoypes for copying multiple activities to other days, we asked him which one he prefered.
These prototypes can be seen on \autoref{fig:mark_mode_copy_dropdown_and_checkbox}.
The difference on the prototypes is that one has a dropdown menu, and the other has a checkbox for each day.
He prefered \autoref{fig:mark_mode_copy_checkbox}, as it was easier to copy to multiple days.
The prototype for adding new pictograms from the phone gallery generated a new discussion.
Emil thought it would be useful to have a priority ranking system on pictograms when one searches for them.
You might have a large amount of pictograms, but realistically a core few would be heavily used.
As such, he would like to see a priority list in searches based on the amount of times pictograms had been used.
