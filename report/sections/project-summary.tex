\section{Summary of the project}\label{appendix:project-summary}
This is the sixth semester project for software engineering students at Aalborg University 2019.
It is a collaboration between seven software groups with 4-6 members.
This project is based around GIRAF, an application designed to help schools and kindergartens structure weeks for autistic children, and to help these children in their everyday.
The project collaborates with multiple institutions, ensuring customer contact is an essential part of the project.
An important aspect of the project is to get experience working with an existing codebase. 
The GIRAF project has been ongoing for multiple years, and as such, an understanding of previous years work has to be achieved.
In a previous year a lot of the back end had been reworked, making the front end incompatible. 
The focus of GIRAF 2019 is to update the front end to function with the back end again and make it more stable and aesthetically pleasing.
\\\\
For this semester the groups function as full stack groups.
This means each group has knowledge within all areas of the application.
To facilitate this, skills groups are established, consisting of a representative from each group.
These groups have an area of expertise, and it is the responsibility of the skill group member to convey the knowledge gained from the skill group to their official group.
There is also a dedicated process group, concerning itself with the processes used across the semester, and a dedicated \textit{product owner} group, concerning itself with customer contact and ensuring the application is made to their specifications.
This report is made by the product owner group, and as such, focuses on customer contact as well as implementation.
\\\\
This semester is split into four sprints, for which we prepare a list of user stories and prioritize them based on customer contact.
Each sprint contains an initial planning meeting, where the user stories are presented and each group is assigned at least one user story.
Regular stand up meetings are held in each sprint to keep everyone up to date with the progress of the different groups.
When a sprint ends, a release candidate is created with the new functionality.
This release is then brought to a usability test with the customers, where they interact with the system to complete tasks we define.
The feedback we receive is then used to prepare the next sprint.
A sprint retrospective ends a sprint, and different aspects of the process are discussed to determine points of improvement.
\\\\
The project experienced difficulties initially, but after a change of framework to Flutter proceeded smoothly.
A new version of the GIRAF application was finalized after sprint 4 and uploaded to the \textit{Google Play Store}.
The product is a usable version of the GIRAF application, that allows a user to use pictograms to structure a day of activities.
