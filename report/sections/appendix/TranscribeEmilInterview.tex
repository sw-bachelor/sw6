\chapter{Transcription of the interview with Emil}\label{appendix:trans-emil}
\textbf{PO:}\\
Okay så, vi har ligesom fået delt spørgsmålene ind i sådan tre kategorier. 
De første de bliver sådan lidt med for at forstå hvordan i lige nu benytter IT i hverdagen, og så kommer der sådan noget specifikt til GIRAF projektet, og så har vi lige nogle sådan rent proces relaterede - vores gruppe og jer som kunde imellem.
Det første vi har tænkt på, det er, sådan i store træk, hvordan benytter i IT i hverdagen lige nu?   
\\\\
\textbf{Emil:} \\
Ja, og det er jo, der skal vi lige sådan snævre det lidt ind. 
Jeg fortalte jo også hvor bredt vi egentligt benytter IT på skolen, ikke også, indenfor alle områder. 
Det gennemsyrer jo en arbejdsplads som en specialskole, men er det med særlig fokus på det her omkring struktur og kommunikation, eller?
\\\\
\textbf{PO:} \\
Ja, lige præcis.
Som støttemiddel til eleverne.
\\\\
\textbf{Emil:}\\
Ja.
Hvordan bruger vi det der?
Jamen vi bruger det ikke sådan, hvad kan man sige, ensrettet.
Det er meget fra klasse til klasse, med det personale der tilfældigvis er i klasserne hvad de har med sig af erfaring og hvad det er de har for nogle konkrete elever og hvad det ligesom kan løse af udfordringer i hverdagen.
Det er med det perspektiv hvor det er.
Så det er på den måde er det meget forskelligartet hvordan det bliver brugt.
Vi har nogle klasser hvor at de kører for eksempel, dagstruktur bliver kørt sådan fælles på en tavle, så alle elever får det samme, ikke også, som sådan et fælles opmærksomhedspunkt.
Og andre har hver deres system hvor måske nogle af dem har det sådan elektronisk. 
Så, ja.
\\\\
\textbf{PO:}\\
Lige for at følge op til det der - er det sådan at hver klasse, er det kun delt ind i sådan årgang og sådan nogle ting der eller er det også hvordan de er som personer?
\\\\
\textbf{Emil:}\\
Det er et godt spørgsmål.
Selvfølgelig så har, så går dem der går til afgangsprøve jo ikke med nogle som ikke har talesprog, det er jo ikke sjov undervisning.
Så, vi har på skolen delt dem op i tre afdelinger.
En afdeling som der faktisk ikke engang ligger i Hammer Bakker, men som ligger ude i Sulsted der hedder Agernhuset for vores højt begavede, normal begavede unge som der følger almindelige skolematerialer.
De går ude i den afdeling, og så har vi sådan en hovedafdeling hvor der er en mellemgruppe, både for dem der er så, ja, dem der ikke lige passer direkte i nogen af dem, og så er der dem der næsten ingen talesprog har.
Mellemgruppen den er rimelig bred. 
Der er nogle som hvor deres sprog det er sådan rimeligt upåfaldende, men de har måske nogle massive autismevanskeligheder omkring det sociale, for eksempel, og ja.
Der kan være mange ting i det, men så hver afdeling har så tre trin.
En udskolingsgruppe, og en indskolingsgruppe og et mellemtrin. 
Så man finder ud af sådan hvor passer de henne sådan kognitivt og hvad for nogle skolematerialer kan de følge, ikke også, og så placerer man dem i afdelinger efter det, og så placerer man dem i klasse efter alder.
Det skal være en af de tre trin. 
Så der er rimelig stor spredning i vores, det er klart når der er nogle der går i både børnehaveklasse med nogle der også går i tredje.
Så, ja, men sådan er det.
Det er jo små grupper til gengæld, så vi differentierer jo hæftigt i hvad de får af undervisning.
\\\\
\textbf{PO:}\\
Ja okay.
Har de sådan, alle eleverne, har de iPad eller tablet til rådighed?
\\\\
\textbf{Emil:}\\
Ja.
Alle elever har iPads til rådighed. 
Det er helt fast, og nogle har endda mere end en iPad til rådighed.
\\\\
\textbf{PO:}\\
Er det noget de sådan er rimelig kompetente i at bruge?
\\\\
\textbf{Emil:} \\
Alle er 100\% kompetente i iPads, ja.
Jeg tror faktisk der er en enkelt klasse der så ikke har iPads ude ved Agernhuset, men de har fået PCer fordi de skal bruge det i forhold til deres afgangsprøve, og man kan kun få en enhed.
Generelt så er det iPads der, for 85\% af alle vores elever. 
Og dem kan alle finde ud af at bruge, selv de aller dårligste kan finde ud af at trykke på en iPad.
Ja.
Det er noget af et fremskridt i forhold til da det var på PC med mus, den der hvor du skal styre noget og klikke op på en skærm.
Der var det ikke alle der kunne bruge computer, men alle kan bruge en iPad.
\\\\
\textbf{PO:} \\
Er det sådan nogle specielle områder du sådan i hverdagen har tænkt at, det kunne være smart hvis det her det var digitaliseret?
\\\\
\textbf{Emil:}\\
Ja. 
Godt spørgsmål.
Altså, vi er jo sådan en arbejdsplads og et område med rigtig mange traditioner også, hvor at man jo gør meget man plejer og har erfaring for at fungere. 
Så det, man kan jo sige at specialområdet er jo generelt, selvom de også er med på at udvikle mange tinge, så er de også meget traditionsbundne. 
Så det kan gøre det lidt svært at rykke.
Altså, jeg synes egentligt, altså det er nogle af de områder jeg har været inde på omkring kommunikation og struktur hvor jeg tænker at der er, det er mest meningsfult at bruge det i nogle særlige situationer.
Vil jeg sige.
Det er, der kan man sige, de programmer der kan altid gøres bedre end det de er. 
Der er ulemper ved dem alle sammen. 
De har hver deres fordele og hver deres ulemper.
så vi er hele tiden sådan på afsøgning af, hvad er det, hvor er det vi får mest for pengene.
Hvad, ja, hvad rammer dem bedst.
Og det skifter også hele tiden.
Det gør det.
Jeg tror egentligt umiddelbart at vores største udfordring, det er personalet som brugere, umiddelbart.
Så jo mere nemt det er at gå til for personalet, jo større vil impact kan du få på det.
\\\\
\textbf{PO:}\\
Må jeg lige høre, nu når du siger det der, er det sådan at lærer personalet op i hvordan man bruger de der apps der?
\\\\
\textbf{Emil:}\\
Ja.
Det er forskelligt også, nogle, der er jo nogle de kan jo gå til det med det samme ikke? 
Og så har vi også nogle der har rigtigt svært ved det.
Det er bare hvis du skal bruge noget, hvis du skal støtte en elev på noget der er så vigtigt som at de trives i deres hverdag og kan have en forudsigelighed og struktur, ikke også, så er det jo vigtigt at alle voksne omkring barnet kan finde ud af at tilgå det her og støtte barnet med det.
Det nytter ikke noget det kun er halvdelen.
Så er det bedre at bruge noget andet.
Så derfor, så kan man sige, er det meget brugervenligt så er der stor sandsynlighed for at alle vil kunne gå ind og gøre det.
\\\\
\textbf{PO:}\\
Er det noget, sådan de apps som i har nu for eksempel, de gør noget for at få, eller sådan gør det nemmere for jer?
Der er Selvfølgelig brugervenlighed i appsne, men er det også sådan guides inde i appsne eller sådan noget?
\\\\
\textbf{Emil:} \\
Faktisk så vil jeg sige, at den der som vi nu lige forsøger at tilkøbe, det irriterer mig enormt meget at jeg ikke kan huske det.
Hvad er det den hedder?
Det er ikke Showmyday.
Nå, den vandt faktisk, vi havde, vi har haft en større afsøgning siden efteråret hvor vi så, hvad er der egentligt overhovedet af muligheder?
Og så skriver vi det ind.
Vi havde to tilbage til sidst. 
Dene ene den var billig, og den anden var sådan en mellemklasse app.
Vi valgte den vi valgte primært fordi at der er så ekstremt mange guides på YouTube til hvordan man skal bruge det.
Så det er mega nemt for personalet at gå ind, den er simpelthen, når du åbner den så er den utrolig intuitiv og nem at gå til for de fleste, og har man brug for støtte er støtten rigtig nem at få.
Det vil også være i forhold til forældre og sådan noget, ikke også, at de kan, man kan henvise til har du set der er YouTube klip nummer 35, der kan du se lige præcis det der du spørger om.
Så det gør det nemt at få det, få hele barnets familie med omkring det, så det ikke er en person i teamet der skal vejlede både kollegaer og forældre og altså det bliver en kæmpe arbejdsbelastning.
Så det var faktisk det den vandt på. 
Den anden den tabte især på at den kun kørte på Android. 
Det er også lidt en ulempe. 
Eller også så kørte den kun på Apple, og så var der nogle af vores elever der havde... 
Nej den kørte faktisk kun på iOS, og så var det sådan at vi har nogle elever der ville have rigtig stor gavn af den som har Android telefoner og ikke ville kunne bruge den der.
Hvor den anden vi købte den kører på begge systemer.
Så det er et stort plus at der er den fleksibilitet. 
\\\\
\textbf{PO:}\\
Så det er en stor bonus hvis der var noget dokumentation, for eksempel i videoformat, tilgængeligt for GIRAF?
\\\\
\textbf{Emil:}\\
Ja, klart.
Noget guide.
Altså man skal tænke at det skal være nemt at gå til og der skal være, du skal kunne hente støtte nogen steder, ja.
\\\\
\textbf{PO:}\\
Er det sådan hovedsageligt kun for guardians, eller er det også sådan for citizens? At de skal have det der i videoformat, eller er det simpelthen personalet man helst bare skal?
\\\\
\textbf{Emil:}\\
Det kan jo være, det kommer jo an på hvordan man laver det.
Så det behøver det jo ikke være.
Det kan også bare være hjælpefunktioner der er gode og så noget derinde, så det synes jeg er svært at svære entydigt på.
Hjælpen skal bare tænkes ind.
Man skal tænke ind at lærere og pædagoger er ikke nødvendigvis gode til IT.
Overhovedet.
Det er bedre at tænke det modsatte. 
Og så er der nogle der har rigtigt nemt ved det, men sådan er det.
Det skal designes til ens mormor.
Så noget jeg egentligt også synes vi mangler lidt, det er apps der kan, jeg viste det der med timeren i så med uret, det er jo sådan en app, det er jo en købe-app, jeg tror den koster en halvtredser eller sådan noget, og den er god fordi den ligner rigtig meget det ur vi har.
Nogle forskellige repræsentationsformer af tiden der går, det er faktisklidt en mangelvare synes jeg.    
Jeg synes egentligt også det er halvdyrt for et program der er meget, meget simpelt at lave. 
Det der det kan satanedme ikke tage ret meget tid at lave det kode til og køre den der ned.
Det synes jeg faktisk at vi har haft tænkt lidt ind i GIRAF, hvordan skulle tiden repræsenteres.
Det er der tænkt lidt for lidt over synes jeg.  
Det er faktisk et ret vigtigt issue for vores elever. 
\\\\
\textbf{PO:}\\
Hvordan tænker du så man kunne repræsentere tid? 
Jeg tænker umiddelbart at det var noget der ville kunne være svært.
\\\\
\textbf{Emil:}\\
Det kunne jo være alt fra, der er sådan lidt forskellige systemer man bruger, der er nogle der arbejder med sådan noget der hedder quarter-dots hvor det er sådan nogen røde prikker der ligesom forsvinder væk.
Du kender det jo også, det kunne også være sådan en bar som du har når et stykke software det loader som der går ned.    
Så det kan tænkes på mange måder, det er bare, det behøver ikke nødvendigvis være den der røde cirkel.
\\\\
\textbf{PO:}\\
Vi har også, jeg ved at der har været snak om lige nøjagtigt den feature med sidste års studerende, hvor de blandt andet har haft noget timeglasrepræsentation eller bare som digital tid.
\\\\
\textbf{Emil:}\\
Ja, vi har også nogle der bruger almindelige timeglas, det er der også nogen der er glade for.
Det vil være rimeligt meningsfuldt at der var flere måder at gøre det på.
Det ser jeg i hvert fald ikke i ret mange software, at de har det. 
\\\\
\textbf{PO:}\\
Er det så sådan for hvert barn at man skal kunne gp ind og vælge hvilken form de gerne vil se?
\\\\
\textbf{Emil:}\\
Ja.
Det kunne også være at man havde dem, det er jo igen det der med hvor meget skal det være flettet sammen, og hvor meget skal det være selvstændige apps. 
Hvor man kan sige, for nogle børn ville det, hvis man bare havde en app der hed tidstageren eller hvad fanden man kunne kalde den, så havde du den gennem dagen på din iPad, hvor du kunne se nu skal du lige, nu skal du vænne dig, vi ved bare den her med batteriet den er super god til dig fordi du kender den fra din iPad.
Så den tæller ned eller hvad den gør, ikke?
Ja, eller quarter-dots eller en time timer så man ligesom har en lille pakke der med forskellige tidsrepræsentationer.  
Så kan det også være meningsfuldt selvfølgelig hvis du kan koble det sammen med aktiviteterne, at man kan se det i forbindelse med kalenderen. 
Ja, det er hvor integreret det skal være.   
Men tidsdelen, den synes jeg den mangler.
\\\\
\textbf{PO:}\\
Der er et spørgsmål der relaterer sig til farver, hvor vi fik at vide at det var en international standard, er der sådan andre standarder man skal være opmærksom på, du lige kender til?
\\\\
\textbf{Emil:} \\
Nej.
Standarden er jo bare at det skal kunne tilpasses. 
Eller, farverne ja, det er en standard, det vil jeg sige, det nok er det. 
Så er der noget som er hyppigt brugt, som for eksempel ikonerne, hvor du tit bruger et system der hedder Boardmaker, som jo ikke er det eneste der findes på markedet, men som er rigtig, rigtigt udbredt indenfor autismeområdet.
Men altså, der kan tales godt og dårligt om dem, de bliver bare rigtigt meget brugt og der er også rigtigt mange elever der lærer dem at kende.
Det er jo så en anden ting, ikke, at de skal lære og forstå, hvad vil det sige, det der ikon som der viser en der for eksempel holder pause, at det ser sådan ud, og så skal jeg gøre sådan.
Hvor at det der med at forstå symboler det er jo lidt abstrakt og faktisk noget vores elever har svært ved, så der er meget udenadslære i det.
Så derfor er det ikke sådan noget man bare lige kan skifte rundt i mellem, men til gengæld tager du nogle af vores normalt begavede unge så synes de jo de er utroligt barnlige og grimme, og det er de jo også, det vil jeg give dem ret i.
Så jeg gjorde meget det i det tidligere projekt, der søgte jeg bare ikonerne på nettet og fandt nogle der var lidt mere pæne, cleane i deres, ja, den visualisering der var lavet. 
Det var mere spiseligt, vil jeg sige.
Det blev der taget godt imod. 
Så det er jo sådan den ene halv-standard man har, det er det Boardmaker der. 
\\\\
\textbf{PO:}\\
I forhold til farver, hvordan, udover der er en standard, er det sådan noget de går op i, eller er det bare sådan at det skal ikke ændre sig?
\\\\
\textbf{Emil:}\\
Nogle gør.
Nogle går op i det, nogle går ikke op i det.
Det andet, altså hvor kraftige de er i farven er nok mindre vigtigt.
Nu kan du se dem jeg lige har med der, ikke også, de er sådan meget farvefyldte eller farvemættede eller hvad man kan kalde det.
Men om de er mere nedtonede eller sådan, det tror jeg er mindre vigtigt.  
Så man kan godt arbejde med nuancerne tænker jeg, men farverne de er nu som de er.
\\\\
\textbf{PO:}\\
Så sådan, rent specifikt for GIRAF projektet her, den primære platform det benyttes på, vil det blive iPads?
\\\\
\textbf{Emil:}\\
Ja.
Hos os får de iPads. 
Det er ikke, jeg ved ikke om Birken har iPads.
Det er jeg ikke sikker på.
Jeg tror, sidste gang hvor jeg snakkede med Birken, det var sidste år, der tror jeg nok at de havde iPads til enkelte, altså de havde fem iPads til udlån blandt alle eleverne.
Så det var ikke deres egen iPad, det var noget de brugte og så kunne de spille på dem eller de kunne arbejde med dem eller hvad det nu var.   
Det var ikke sådan de havde deres egne, men det har vores, og det er lidt specielt, for det er sådan lidt dyrt at have det til hver elev, ikke?
Men det tror jeg ikke der er nogle af de andre deltagere i projektet der har.
\\\\
\textbf{PO:}\\
Hvordan, der har tidligere været udtrykt et ønske om en PC version af GIRAF, hvordan forholder det sig til jeres situation lige nu?
\\\\
\textbf{Emil:}\\
Det vil være mindre vigtigt på Egebakken.
Men det kan godt være der er andre der syntes det var vigtigt.
Må jeg lige spørge, kommer i til at holde sådan de her kundemøder der hvor der er flere repræsentanter end mig, eller var det bare fordi det var smart at jeg lige var her?
\\\\
\textbf{PO:}\\
Det er fordi de andre har vi først mulighed for at mødes med her, jeg mener det er den 27. 
Så det er bare for at vi lige har noget at gå ud fra til at starte med.
\\\\
\textbf{Emil:}\\
Okay.
Fint nok, det er godt.
\\\\
\textbf{PO:}\\
Men jo, det kommer til at være blandet fremover.
Der er selvfølgelig også for at få et opdateret overblik over hvordan tingene foregår og så videre.
Okay, sådan normaltvist, nu snakkede du under oplægget om at, hvis der går noget galt når i er på ekskursion eller lignende, er det sådan, de iPads i har, er der mulighed for at have internetadgang derpå altid, eller skal det fungere offline?
\\\\
\textbf{Emil:}\\
Det skal fungere offline.
Det skal det helt sikkert.
Vi har haft net der går ned nogen gange. 
Der er så meget, altså, når nettet går ned på vores skole, så er der kaos i alle rum, kan jeg godt fortælle dig.
Jeg havde faktisk en video jeg ville viser jer som jeg havde fundet på min iPad, hvor jeg ligesom havde forberedt, at når den kom på skulle jeg sørge for at skrue lyden ned fordi det var faktisk en dag hvor der ikke var internet, hvor jeg sad og arbejdede med en dreng, med noget kommunikationsprogrammer der, og der var en helvedes larm i baggrunden. 
Der er to der skriger helt vildt fordi at der ikke er internet.
Så, det skal køre offline.
\\\\
\textbf{PO:}\\
Der har også tidligere været udtrykt et ønske om mulighed for at have flere sprog i systemet. Er det noget i stadig ser som relevant ude ved jer, eller er det primært dansksprogede?
\\\\
\textbf{Emil:}\\
Det er primært dansksprogede.
Jeg vil godt nok sige det er tiltagende med flersprogede børn, det er det.
Så er det ligesom også, man skal sige hvor skal man så stoppe henne?
Måske engelsk, ikke også, har vi mange der også bliver gode til, også nogle af dem der er i GIRAF målgruppen, de sidder meget med iPad og ser meget på YouTube på engelsk, og de bliver faktisk rimeligt gode til det.
Ja, men det er nok også det. 
For ellers så har vi alt muligt blandet, og så kan der være indonesisk til vietnamesisk og, ja.
Så jeg har svært ved at så hvor man skal stoppe henne.
\\\\
\textbf{PO:}\\
Så er der sådan lidt en prioriterings, som der også har været inde på under spørgsmålene her.
\\\\
\textbf{Emil:}\\
Nej, må jeg lige sige en ting til med det her medsproget faktisk?
Jeg tænker, det er nok vigtigere at man tænker ind at det i høj grad skal kunne bruges nonverbalt.
Altså der skal være så lidt sprog i det som muligt. 
Fordi sproget er noget af det de har svært ved.
Så strukturen skal ligge i visualisering, og ikke så meget i sprog.
Det skal i tænke ind, i hvert fald.
\\\\
\textbf{PO:}\\
Så er der, som der blev nævnt tidligere, om der er sådan fokus på, nu har vi jo, der er week planneren, der er som du selv nævnte kategorispillet, pictodraw, voice game.
Der har jo sådan været flere ting ind over efterhånden.
Hvis du skulle, jeg kan finde en oversigt her med dem, om jeg kan få dig til sådan at udvælge hvad i vil sætte mest fokus på, eller om der er nogle nye fokusområder?
\\\\
\textbf{Emil:}\\
Altså launcheren den kørte jo, det var den som det kørte under.
Det var jo sådan, ja, styresystemet eller hvad man kan kalde det.
Jamen, category tool og category game er det ikke det samme? 
\\\\
\textbf{PO:}\\
Category tool er til at administrere kategorierne af piktogrammerne.
\\\\
\textbf{Emil:}\\
Nå.
Det var fordi det kan jeg simpelthen ikke huske det der.
\\\\
\textbf{PO:}\\
Det er også sidst opdateret i 2015.
\\\\
\textbf{Emil:}\\
Det er jo det ikke.
Jeg tror det var første gang jeg var med der, i det år.
Pictodraw det er jo selvfølgelig til at lave billederne til kalenderen.
Altså, egentligt så tænker jeg, det der med at lave billederne, jeg synes der er, det fungerer rigtigt godt de steder hvor at programmerne har forskellige muligheder når du skal have billeder, og du kan for eksempel hente dem direkte fra Google af.
Det er ved at være så udbygget, så at når jeg, for eksempel, skal lave et stykke kommunikationssoftware, og jeg skriver mund, for eksempel, hvis det er det jeg skal have på, og så går ind og googler mouth, så får jeg 99\% sikkert et hit som er brugbart.
Som jeg ikke ville give den ekstra indsats det ville være at sidde selv og tegne og justere og lave en hel masse.
\\\\
\textbf{PO:}\\
Så det vil egentligt være rart hvis det var let i weekplanneren at kunne søge direkte på Google?
\\\\
\textbf{Emil:}\\
Ja.
Jeg synes jeg har så gode oplevelser med de steder hvor du bare kan hente dem ind, at det tror jeg bare jeg ville satse på at få implementeret godt, og så lade være med at man skal lave dem selv.
Jeg er også bange for at der er nogle der vil være uenig med mig, men under alle omstændigheder giver den jo ikke, pictodraw giver jo ikke ret meget mening i sig selv.
Det er jo et addon til et andet program.
Så det kræver at det program det er hæftet på det er værdifult, ellers så giver det ingen mening.
Så derfor vil jeg også syntes at det var vigtigere at man lagde energi i primær programmet, umiddelbart.
\\\\
\textbf{PO:}\\
Som i det her tilfælde vel er weekplanneren?
\\\\
\textbf{Emil:}\\
Ja, nemlig.
Så er der sequence, hvad var det nu det var, det er jeg ikke sikker på?
\\\\
\textbf{PO:}\\
Jeg mener det er, ja, som laver en sekvens af piktogrammer i stedet for.
\\\\
\textbf{Emil:}\\
Ja, det er så når du, er det ikke hvor du går ind og vælger, er det der hvor du laver et underskema, eller er det der hvor du går ind og vælger mellem forskellige aktiviteter?
Det er jeg altså en lille smule usikker på.
\\\\
\textbf{PO:}\\
Jeg mener det er hvor, den hvor du vælger mellem aktiviteter det hedder choice board eller noget, og er en del af weekplanneren.
Jeg tror umiddelbart det der det er ligesom det der du også viste med "jeg vil have", så en sekvens af de her billeder simpelthen skal bygge en sætning.
\\\\
\textbf{Emil:}\\
Nå, ja, det er rigtigt. Det er der jo nogle der, prøv lige at se her en gang.
Jeg tror det fungerer lidt sådan her.
Den har jeg jo ikke implementeret på den her.
Jeg tror det er der hvor den ligger sig op på sådan en linje der.
Nu gør vi lige sådan her.
Hvor der er sådan en linje heroppe hvor man ligger dem på, og så siger man, for eksempel, bum, og så kan den læse op.
Det kunne også være man var, for eksempel, ja på Egebakken, og så skulle Johanne noget med, det kan være hun er glad.
Altså sådan at sætte dem sammen, ikke?
Jeg tror det er sådan noget der den har tænkt sig måske så, hvis det er sådan en sammenkobling af piktogrammer.
Jeg kan ikke huske specifikt hvad den ellers er.
Voice game, ja det kan jeg godt huske hvad er.
Den har vi prøvet, hvor jo højere du snakker, så skal du køre forbi nogle kasser og sådan noget, ja.
Den er meget fin, men det er jo også sådan en selvstændig app.
Nu er vi igen mere ovre i sådan noget undervisning, den kompenserer jo ikke for noget, vel?
Altså, ja, så det er hvad man ligesom har fokus på. 
Men noget som man kan sige, på nogle elever vil det her bare være fin at downloade og bruge, men det ville være enkelte elever det handler om, måske en ud af tyve.
\\\\
\textbf{PO:}\\
Ja, okay.
Så hvis vi skulle rangere det med prioritering af, hvis vi skulle lave mere end at arbejde på weekplanneren?
\\\\
\textbf{Emil:}\\
Så synes jeg det skulle være kategorier.
Kategori delen, helt klart.
\\\\
\textbf{PO:}\\
Så category game?
\\\\
\textbf{Emil:}\\
Ja det synes jeg helt klart.
Det har der ikke været arbejdet på i lang tid, og det er faktisk noget som mange synes kunne være, det er egentlig min kollega nede fra børnehaven der er talepædagog dernede, det er hendes gamle ide, og den holder synes jeg også.
\\\\
\textbf{PO:}\\
Okay, så der har tidligere været et ønske om at man
\\\\
\textbf{Emil:}\\
Undskyld, det var lige i forhold til den anden, hvis nu der er rangerede, hvis ikke man tænker den der med tidsdelen ind som en selvstændig app, det var noget man kunne overveje.
\\\\
\textbf{PO:}\\
Ja, det er faktisk noget der er på planen for at vi skal lave allerede her i starten til weekplanneren.
\\\\
\textbf{Emil:}\\
Okay, ja, men i kunne overveje lidt om hvordan det ville være at lave den også som den her voice game kører for sig.
Altså, man bare kan hente ned, om det ville kunne laves på den måde.
Hvis det er meget besværligt skal i selvfølgeligt ikke gøre det, men hvis altså det er nemt så kunne man jo overveje at gøre det.
Så havde man noget man faktisk kunne, den tror jeg ville få en høj brugsfrekvens, hvis den var god, i et relativt lille stykke software.
Hvor de andre har konkurrenter, så har det ingen konkurrenter.
\\\\
\textbf{PO:}\\
Okay, der har tidligere været et ønske om at kunne skifte sådan tilstand i programmet, fordi de her farver godt kunne blive for meget for nogle af brugerne, og så sætte den hen i gråtoner i stedet for.
Er det sådan en normal brugscase?
\\\\
\textbf{Emil:}\\
Ja.
Det så vi jo også, der var en der ikke havde farver på hans dimmer der, hvor det bare var hvidt. 
Det er jo, det er der mange der kører med. 
\\\\
\textbf{PO:}\\
Hvis nu man siger den får mulighed for at du kendte billeder ind fra Google af, vil det så være en fordel at den selv kan lave billederne i gråtoner når den viser det, eller er det ikke nede på det niveau?
\\\\
\textbf{Emil:}\\
Nej, det er ikke så vigtigt.
Det er mere vigtigt at det er tydeligt. 
Altså, at der ikke er, det er igen det der med at de har svært ved at optage for meget visuel information på en gang, så det skal være enkelt. 
Det tror jeg det er det i skal tænke ind, og hvad der så vil virke mest, det tror jeg det vil være jeres vurdering.
\\\\
\textbf{PO:}\\
I forhold til når vi fokuserer på weekplanneren, hvis du skal sige den vigtigste del heri, hvad er det så, det vigtigste funktionalitet af weekplanneren?
\\\\
\textbf{Emil:}\\
Det er svært med sådan et relativt stort system jo, der kan mange ting.
Jeg synes det er svært at tage en ting ud og så sige at det er den vigtigste, det er jo deres samspil der på en eller anden måde giver værdi.
Mon ikke det er, trods alt, sådan noget som at man kan variere i hvordan den visuelt skal vise dagene.
\\\\
\textbf{PO:}\\
Så tilpasningsmulighederne?
\\\\
\textbf{Emil:}\\
Ja.
Det vil jeg nok sige det nok er det vigtigste.
\\\\
\textbf{PO:}\\
Okay. Så er det bare sådan lidt, hvad hedder det, praktiske ting. 
Sådan med foretrukken kommunikation, hvis vi skal have fat i dig?
\\\\
\textbf{Emil:}\\
Nå ja.
Jamen det er også hvad i synes.
\\\\
\textbf{PO:}\\
Så det er mail, telefon eller?
\\\\
\textbf{Emil:}\\
Ja, jeg har det hele.
Jeg kigger mail tit. 
Ikke så meget i weekenden, men flere gange om dagen.
Altså på min arbejdsmail hvis i har den, nu ved jeg ikke?
\\\\
\textbf{PO:}\\
Jo det er jeg ret sikker på.
\\\\
\textbf{Emil:}\\
Den var den i skrev, jeg fik en mail på her for nylig.
\\\\
\textbf{PO:}\\
Jo, det har været Ulrik. Nu skal jeg lige se, vi har en liste herinde.
Og nummeret fungerer også?
\\\\
\textbf{Emil:}\\
Ja, nummeret fungerer også, men det kræver jeg er på kontoret, og det er langt fra altid jeg er det.
Det er en tredjedel af tiden eller sådan noget, sådan.
\\\\
\textbf{PO:}\\
Men hvis du tjekker mailen så ofte, så hvis det er akut tænker jeg vi prøver telefonen ellers sender vi en mail.
\\\\
\textbf{Emil:}\\
Ja, i kan altid ringe, altså skriv en mail og ring så får i fat i mig.
\\\\
\textbf{PO:}\\
Fornemt.
\\\\
\textbf{Emil:}\\
I kan også få mit mobilnummer privat, det er ligegyldigt, det vil jeg gerne give. 
Så kan i få fat i mig hele tiden.
Det kan i bare skrive.
\\\\
\textbf{PO:}\\
Det tænker jeg vi kan finde ud af hvis det bliver relevant senere.
\\\\
\textbf{Emil:}\\
Jamen det er i orden, der er mange arbejdsrelationer der får det også så det er ligegodt.
\\\\
\textbf{PO:}\\
Okay, vi tænker sådan med, jeg ved ikke om der tidligere har været sådan at i er blevet draget med ind over når der er blevet lavet de der user stories til hvert sprint vi skal i gang med, sådan til hver iteration af udviklingen?
Sådan med det er de her features vi gerne vil lave, ligesom at få jeres godkendelse af det også.
\\\\
\textbf{Emil:}\\
Jamen det har været lidt forskelligt hvor meget vi har været med jo. Nogle år har vi jo bare været med til de her opstartsmøder, og så var der et år hvor der også var lidt tvivl, der var noget der gik i ged i hvert fald kan jeg huske et år, hvor der nærmest ikke blev lavet noget.
Der var en hel masse problemer. 
Det var ikke sidste år, det var forrige år eller året før, må det have været.
Ja det var rigtigt rodet kan jeg huske, de var selv vildt utilfredse med den måde det havde kørt på. 
Så der har vi jo ikke været så meget med, og så er der andre år hvor vi har været rigtigt meget med sådan løbende til, jeg ved ikke, fem møder eller sådan noget, eller seks møder i løbet af projektet.
Så det er op til jer synes jeg, hvad i synes, hvornår i har brug for feedback på det i laver og ja.
Men det vil være lidt forskelligt hvor meget vi kan være med, for jeg har en kollega, hun er meget bundet op på sin undervisning hvor jeg har meget tid selv at bestemme over, så jeg kan næsten altid være med.
Hun kunne ikke være med i dag, for eksempel, men hun vil gerne være med her, men ja. 
\\\\
\textbf{PO:}\\
Jo tanken er lidt, den her gang, nu lavede de en masse af det bagved om de sidste par år, så vi kommer primært til at fokusere på at nu skal der være noget funktionalitet der fungerer igen. 
Hvor vi tænker det kunne være rart sådan løbende at få jer som kunder med ind over, og også få lavet nogle test af hvordan vurderer i systemet.
\\\\
\textbf{Emil:}\\
Ja, det kunne være fint.
\\\\
\textbf{PO:}\\
Både set som, ja barn og som voksen.
\\\\
\textbf{Emil:}\\
Men der er jo selvfølgelig lige, der er jo lige etikken i det, at hvis der er nogle ting vi skal prøve af på elever, så kræver det jo sådan rent etisk at vi synes det har en kvalitet der gør at vi vil præsentere det for barnet.
Fordi vi kan jo ikke have sådan nogle forsøgskaniner gående vi bare prøver alt muligt af på.
Så det vil vi lige vurdere professionelt, om vi synes det her har en kvalitet der gør at det vil vi egentligt gerne afprøve.
\\\\
\textbf{PO:}\\
Selvfølgeligt.
Ja, og så ved de møder tænker vi også at i er med til at prioritere. 
Vi laver sådan en liste sammen med jer med de her features vil vi gerne have lavet på sigt, hvad er vigtigst at få lavet lige nu.
\\\\
\textbf{Emil:}\\
Ja, det har vi også prøvet før. 
\\\\
\textbf{PO:}\\
Okay super, det var i hvert fald noget som vi så fra de sidste rapporter også, at der var nogle der efterspurgte at de var mere med til den her prioritering. 
Sådan at de fik det diskuteret ordentligt i gennem.
\\\\
\textbf{Emil:}\\
Det lyder fint, men det er nok i hvert fald vigtigt at der også er en, at vi så er en samlet gruppe der kan respondere på det.
For det er jo utroligt svært at gøre det som en.
\\\\
\textbf{PO:}\\
Ja, det er det. 
\\\\
\textbf{Emil:}\\
Det er jo på den måde, jeres kundegruppe er jo meget diverse i forhold til hvad i ellers kan komme ud for.
Vi har mange forskellige interesser og helt forskellige hverdage, og ting vi sidder i, så det er jo en meget stor udfordring for jer at finde nogle fællesting der gør at det er det her i ligesom synes giver mest mening.
\\\\
\textbf{PO:}\\
Det tager vi som en udfordring.
\\\\
\textbf{Emil:}\\
Ja, det vil være en udfordring tænker jeg.
Det er noget helt andet end skulle lavet et stykke software til en eller anden målenhed der som en virksomhed har udviklet.
De ved lige hvad de vil have, og det skal bare se sådan der ud.
Her er vi faktisk nogle kunder der måske er lidt uenige i hvordan vi gerne vil have det. 
\\\\
\textbf{PO:}\\
Jeg tror det var det. Det var mere eller mindre alle spørgsmålene. Jamen tak for det.
