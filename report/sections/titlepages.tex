\pdfbookmark[0]{English title page}{label:titlepage_en}
\aautitlepage{%
  \englishprojectinfo{
    Title%title
  }{%
     Bache %theme
  }{%
    Spring Semester 2019 %project period
  }{%
    SW610f19 % project group
  }{%
    %list of group members
    Andreas Stenshøj\\ 
    Daniel Moesgaard Andersen\\
    Frederik Valdemar Schrøder\\
    Jens Petur Tróndarson\\
    Rasmus Bundgaard Eduardsen\\
    Mathias Møller Lybech
  }{%
    %list of supervisors
    Chenjuan Guo\\
  }{%
    1 % number of printed copies
  }{%
    May 28, 2018 % date of completion
  }%
}{%department and address
  \textbf{Department of Computer Science}\\
  Aalborg University\\
  Selma Lagerlöfs Vej 300\\
  9220 Aalborg East, DK\\
  \href{www.cs.aau.dk}{www.cs.aau.dk}
}{% the abstract
The Graphical Interface Resource for Autistic Folk (GIRAF)
 is an application suite designed to help institutions for citizens with autism spectrum disorder. 
 Its main functionality is to provide an easy way to schedule activities for a given week.
 It is a multi-project being built by sixth semester software engineering students at Aalborg University. \\
 This report focuses on the product owner aspect of the GIRAF project, with a focus on gathering information from customers, building prototypes, and usability testing. \\
 In addition to the responsibilites as a product owner, it will describe the processes used for the application.\\
 The process uses full stack development teams to facilitate cooperation between developers.
 The result is a functional verison of the application, usable on both Android and iOS.
 According to the customers, it is an intuitive application compared to what they have previously tried.}

%\cleardoublepage
%{\selectlanguage{danish}
%\pdfbookmark[0]{Danish title page}{label:titlepage_da}
%\aautitlepage{%
%  \danishprojectinfo{
%    Rapportens titel %title
%  }{%
%    Semestertema %theme
%  }{%
%    Efterårssemestret 2010 %project period
%  }{%
%    XXX % project group
%  }{%
%    %list of group members
%    Forfatter 1\\ 
%    Forfatter 2\\
%    Forfatter 3
%  }{%
%    %list of supervisors
%    Vejleder 1\\
%    Vejleder 2
%  }{%
%    1 % number of printed copies
%  }{%
%    \today % date of completion
%  }%
%}{%department and address
%  \textbf{Institut for Elektroniske Systemer}\\
%  Fredrik Bajers Vej 7\\
%  DK-9220 Aalborg Ø\\
%  \href{http://es.aau.dk}{http://es.aau.dk}
%}{% the abstract
%  Her er resuméet
%}}
