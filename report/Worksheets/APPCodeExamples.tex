\chapter{Code examples}\label{APP:examples}
This appendix will describe some examples of the differences between Phal and Arduino code, in order to illuminate the differences.\\


\section{Example 1}
\noindent The first example, seen on Listings~\ref{code:code1Phal} and \ref{code:code1Arduino}, will declare 5 lights and repeatedly turn them on and off for 5 seconds.
\begin{lstlisting}[caption={Code example 1 in Phal}, label={code:code1Phal}]
setup{
  lightbulb lb1 := pin 1
  lightbulb lb2 := pin 2
  lightbulb lb3 := pin 3
  lightbulb lb4 := pin 4
  lightbulb lb5 := pin 5

  group lights{
    lb1
    lb2
    lb3
    lb4
    lb5
  }
}

repeat{
  lights := on
  wait for 5 seconds
  lights := off
  wait for 5 seconds
}
\end{lstlisting}

\begin{lstlisting}[caption={Code example 1 in Arduino}, label={code:code1Arduino}]
void setup(){
  pinMode(1, OUTPUT);
  pinMode(2, OUTPUT);
  pinMode(3, OUTPUT);
  pinMode(4, OUTPUT);
  pinMode(5, OUTPUT);
}

void loop(){
  digitalWrite(1, HIGH);
  digitalWrite(2, HIGH);
  digitalWrite(3, HIGH);
  digitalWrite(4, HIGH);
  digitalWrite(5, HIGH);
  delay(5000);
  digitalWrite(1, LOW);
  digitalWrite(2, LOW);
  digitalWrite(3, LOW);
  digitalWrite(4, LOW);
  digitalWrite(5, LOW);
  delay(5000);
}
\end{lstlisting}

\section{Example 2}
The second code example will open a window with a motor if the temperature reaches above a set temperature limit or close the window if it reaches below a set limit.
\begin{lstlisting}[caption={Code example 2 in PHAL}, label={code:code2PHAL}]
setup{
  motor openWindow := 1
  motor closeWindow := 2

  temperatureSensor temp := 3

  number tempHigh := 30
  number tempLow := 20
}

repeat{
  if(temp.reading is greater than tempHigh) then {
    call turnOnMotor with (openWindow)
  }
  if(temp.reading is less than tempLow) {
    call turnOnMotor with (closeWindow)
  }
}

define turnOnMotor with (motor m) returnType(none) {
  m := on
  wait 5 seconds
  m := off
}
\end{lstlisting}

\begin{lstlisting}[caption={Code example 2 in Arduino}, label={code:code2Arduino0}]
#include <dht.h>
dht DHT;

void setup(){
  pinMode(1, OUTPUT);
  pinMode(2, OUTPUT);
}

void loop(){
  int chk = DHT.read11(3);
  
  if(DHT.temperature > 30) {
  turnOnMotor(1);
  }
  if(DHT.temperature < 20) {
    turnOnMotor(2);
  }
}

void turnOnMotor(int m){
  digitalWrite(m, HIGH);
  delay(5000);
  digitalWrite(m, LOW);
}
\end{lstlisting}
\section{Example 3}
An example that show how loop in PHAL are different than loops in Arduino code
\begin{lstlisting}[caption={Code example 3 in PHAL}, label={code:code3PHAL}]
setup{
  lightbulb lb := pin 1
 
  number i := 0
}

repeat{
  loop 8 times {
  lb := on
  wait 1 seconds
  lb := off
  }

  loop until i greater than 8 increase i by 1 {
  lb := on
  wait 1 seconds
  lb := off
  }
}
\end{lstlisting}

\begin{lstlisting}[caption={Code example 3 in Arduino}, label={code:code3Arduino}]
void setup(){
  pinMode(1, OUTPUT);
}

void loop(){
  int i;
  
  for(i = 0; i < 8; i++){
  digitalWrite(1, HIGH);
  delay(1000);
  digitalWrite(1, LOW);
  }
  i = 0;
  while(i <= 8){
  digitalWrite(1, HIGH);
  delay(1000);
  digitalWrite(1n LOW);
  i++;
  }
}
\end{lstlisting}
%\begin{lstlisting}[caption={}, label={}]
 
%\end{lstlisting}

%\begin{lstlisting}[caption={}, label={}]

%\end{lstlisting}