\chapter{Introduction}\label{ch:Introduction}
Technology is constantly changing how we live our lives, do our work and interact with each other. One of the ways technology influences our everyday lives is through home automation. Home automation is becoming a bigger part of mainstream consumers lives due to it being a convenient way to automate certain tasks in the home that could be considered tedious. Some of the first to adapt to new technology are hobbyists \cite{TheHobbyistRenaissance}. The hobbyist uses the technology because of excitement and an interest in the subject, and not because of any professional reasons.
\\\\
The output of this project is a new programming language named PHAL, which is short for Personal Home Automation Language, which will help hobbyists build their home automation system, and a compiler designed to compile this new language to create usable programs. In order to accomplish this, this report will cover an initial problem analysis based on the topic of home automation, in which the necessary functions for home automation are considered. After the initial design considerations are concluded, the language analysis takes place where the language criteria and essential features are discussed and related to the target demographic, and finally some examples of PHAL code will be defined. 
\\\\
The report is split into five parts, each corresponding to a part in the process of crafting a compiler.
\\\\
In the \textit{Analysis} part, an analysis of the problem will be conducted where a platform will be chosen and the targeted users will be specified. The goal of part one is to get a clear understanding of the problem and how it can be solved.
\\\\
The second part, \textit{Designing PHAL}, deals with the definition of the PHAL language where we will delve into what it takes to develop a programming language and how the PHAL language should look.
The goal here is to define the requirements for the language, which will be done using the MosCoW model and give the reader an idea about the ideas that define PHAL.
\\\\
In the third part, \textit{Compiler design}, the language will be formally defined using context free grammars and we will delve further into what is required for creating a programming language in more technical terms.
\\\\
In the fourth part, \textit{Implementation}, the concrete implementations of PHAL will be explained using code examples. 
This part will include more information about how to generate an abstract syntax tree (AST), using ANTLR4 and how to later traverse this tree to analyse the code.
Likewise, it will delve into the construction of making a binding visitor, type checker and finally the code generation for PHAL.
\\\\
In the final part, \textit{Conclusion}, we will conclude on the results of this project, including a discussion about future work to further improve upon the PHAL language. 