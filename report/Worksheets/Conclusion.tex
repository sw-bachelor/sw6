\chapter{Conclusion}\label{ch:conclusion}
In order to conclude the project, the final result will be reviewed in relation to the MOSCOW model outlined in Section~\ref{section:moscow}.
\\
Everything from the \textit{must have} section was fulfilled, with the choice of supporting LEDs instead of lights to increase simplicity and allow further expansion of the component library later.
\\
Likewise, the points from \textit{should have} are roughly accomplished. 
The programmer can apply the on or off operation on a group in order to toggle the status of the members of the group.
\\
From the \textit{could have} and \textit{won't have} sections, the ability to include third part libraries was included.
\\\\
In the current state of PHAL, a programmer will be able to make some minimalistic home automation systems using various control structures and in turn be able to control lights and motors.
\\ 
Thus we have deemed PHAL to be functional as a platform which can be expanded upon with the development of further component libraries, and to hopefully simplify the process of creating a home automation system for users with a limited prior knowledge of programming.
\\
\section{Comparisons of PHAL and Arduino code}
Here is a little example that shows the difference between PHAL code and Arduino code

\begin{lstlisting}[caption={Code example in Phal}, label={code:codePhal}]
setup{
  temperatureSensor temp := pin 12
  motor fan := 3
}

repeat{
  if(temp.reading greater than 23) then{
    fan := on
  }
  fan := off
}
\end{lstlisting}

\begin{lstlisting}[caption={Code example in Arduino}, label={code:codeArduino}]
#include <dht.h>
dht DHT

void setup{
  pinMode(12, OUTPUT);
}

void loop(){
  int chk = DHT.read11(4);
  int temperature = DHT.temperature;

  if (temperature > 23) {
    digitalWrite(12, HIGH);
  } 
  digitalWrite(12, LOW);
}
\end{lstlisting}
The PHAL code has a higher level of readability than the Arduino code because it works with components as types instead of refering to their pin number. This means that the programmer does not have to remember which components are connected to which pin but can instead just refer to their ids. The ability to toggle the components on or off is likewise a more natural way to work with them instead of having to use the digitalWrite function.
More comparisons of PHAL and Arduino code can be found in Appendix~\ref{APP:examples}.

\section{Future work}
In this section we will present some of the future work that could be done to improve this project, in addition to the points mentioned in the MoSCoW model, that were not accomplished. 
These are some improvements that would have been nice to have, but were not prioritised due to the limited time available.

\subsection*{Complex operations for components}
At the moment, most of the components that are supported by PHAL only have the ability to be turned on and off. 
It would be ideal to have even more operations that could be used for each specific component to allow for more customisation, such as dimming for lights or the ability to set an angle for a servo motor.

\subsection*{Easy to include components}
The components implemented in PHAL are defined directly in the grammar, which makes it difficult to introduce new components to the language without making major changes. 
An improvement could be to have the components not be dependent on the grammar specification, but instead be included with the library inclusions, so that the programmer easily could include any components.
In this project, we have implemented support for three different components. Allowing users to include third party libraries for components, would mean that more diverse programs could be created through PHAL. 
For example, support for the motion sensor component would be ideal, because it would increase the automation of the program allowing for example light bulbs to be switched on based on the motion sensor input.

\subsection*{Reference keyword}
In PHAL, the function calls are call by value. We would like the programmer to have the option to also make calls by reference. Therefore we would have liked to have a reference keyword implemented that can be put on parameters to pass a reference instead of a value.


\subsection*{Number type improvements}
The number type is not fully optimised at the moment when compiling to either integers or floating point values. 
In the current state, when a number variable is assigned to either a function call or an element in a list we chose to set it as a float. 
The reason for this is that we did not have the time to implement functionality that could predict the outcome of these.
With better optimisation on the number type, the overall size of a PHAL program could be reduced.

\subsection*{Text improvement}
The text type is not very useful at the moment. 
Something that could be helpful would be to iterate through every letter in a text which would give the programmer some opportunity to print the text in an easier way.\\
In addition to this, the text type will be more useful if output modules such as a display are implemented, where the text can be shown.

\subsection*{Possibillity to compile down to multiple platforms}
Another functionality that could be added to the compiler in the future would be the possibility of compiling for multiple platforms. 
At the moment it is only possible generate code for Arduino. 
It would beneficial to be able to compile to some of the other platforms mentioned in Subsection \ref{PlatformsForHomeAutomation}. 
This would likewise expand the possible user base of PHAL and make the language more flexible for it's users.