\chapter{Syntax definition}\label{ch:LanguageDefinitions}
In order for PHAL to be realised and to create a compiler to function as expected for the language, the language has to be defined. This is done through a combination of informal and formal definitions in the following chapter.

\section{Informal definition}
The aim for PHAL is to be relatively simple in comparison to other programming languages. 
It is intended for the Arduino hardware platform for home automation purposes, without needing extensive programming knowledge.

\subsection{Types} \label{Def:Types}
The programming language offers multiple data types:
\begin{itemize}
    \item The \textit{text} type consists of a sequence of characters.
    \item The \textit{number} type consists of a sequence of numbers. This type allows both integers and decimal values. 
    \item The \textit{bool} type is either \textit{true}, \textit{false}, \textit{on} or \textit{off}.
    \item The \textit{group} type is a complex type, consisting of multiple advanced data types.
    \item The \textit{list} type is a complex type, consisting of multiple elements of the same data type.
\end{itemize}
These types have been chosen in accordance with the considerations relating to the target demographic and the features discussed in Chapter~\ref{ch:ProblemAnalysis}.
The integer and float types found in languages such as C have been condensed into the single type called number, including both the floating point functionality and integer functionality. 
The \textit{text} type is functionally the same as the \textit{string} type found in languages such as C\#, but named more akin to what the users would use in their everyday speech for consistency in relation to the number type. 
\\\\
The bool type is maintained from languages such as C\#, where it is defined through the use of conditions that have the values true or false. 
When the name of the type was considered, candidates such as \textit{truthvalue} and \textit{boolean} were suggested, but the name was not changed as a suitable replacement was not found.
\\
As an extension to the regular usage of boolean values, PHAL will accept \textit{on} and \textit{off} as aliases for respectively true and false values.
\\\\
The group type allows grouping of multiple component types. 
An example of this could be kitchen appliances. 
Eg. a user has three lights in the kitchen, and wants them to automatically turn on at a specific time in the morning to facilitate waking up. 
In the language, the programmer would be able to combine these lights using the group type, creating an umbrella term for these lights - this could for example be a group named \textit{kitchenLights}. 
The programmer would then do automation on this group called \textit{kitchenLights} rather than each separate light one at a time. 
Another example would be starting a motor to open a window at the same time as the lights turn on - this could be accomplished in a group called
\textit{kitchenComponents} for instance. 
This means the group can also consist of multiple data types as well as nested groups.
\\\\
A list allows the user to group multiple elements of the same type together. 
The intended functionality of a list, compared to the group type, is that it allows the elements to be grouped together, but still be interacted with as individual elements.

\subsection{Keywords}
PHAL has a number of reserved keywords. 
The first set of keywords are related to selective control structures. 
As shown in Listings~\ref{code:If-statement} and \ref{code:Switch-statement}, we reserve the keywords \textit{if, then, else, switch, default} and \textit{case}. 
For \textit{if-statements} we combine the curly brackets of \textit{C} with the \textit{then} keyword for simplicity, and likewise the switches are kept in the style of \textit{C}, but without the need for a \textit{break} statement to avoid fall-through.
\\\\
For iterative control structures, shown in Listings~\ref{code:Loopstatement1}, \ref{code:Loopstatement2} and \ref{code:Loopstatement3}, we reserve the keywords \textit{loop, until, increase, decrease, by} and \textit{times}. 
These keywords are needed for the different loop variations, and are used in an effort to make loops simpler.
\\\\
Creating a function is done with the keywords \textit{define, with, none, return} and \textit{returnType} as shown on Listing~\ref{code:MethodDeclParam}. 
As seen on the listing, the \textit{with} keyword is used as a way of generating context in combination with the general parentheses notation for formal parameters of other languages. 
On top of this, the \textit{returntype} is defined after the parameters. 
In the body of the function the actual value returned is specified through the \textit{return} keyword.
\\
For calling functions we reserve the keywords \textit{call, with} and \textit{none}. 
As shown on Listings~\ref{code:MethodCall1} and \ref{code:MethodCall2}, the \textit{call} keyword is used for starting the call, and the \textit{with} keyword is used for the same reason as when defining functions. 
\\\\
For comparing statements the language reserves the use of mathematical operators. 
The demographic, even if they are not familiar with programming, are expected to be familiar with the mathematical operators \textit{<, >, <=, >=} and \textit{=}. 
On top of this, the language also allows the use of the following keywords \textit{or, is, and} and \textit{is not}. 
These keywords will replace the \textit{C} operators \&\& and || for simplicity.
\\\\
In terms of mathematical operators, the language reserves the use of the common algebraic operators \textit{+, -, /} and \textit{*}. 
These keywords will function the same as they do in mathematical operations. 
Finally, the \textit{+} operator will also be overloaded to serve as the operator for concatenating two strings when used on text variables.
\\\\
For the language we reserve certain keywords related to types as described in Subsection~\ref{Def:Types}. 
These keywords are \textit{text, number, bool, group} and \textit{list}. 
These keywords are reserved for definition of types for variables, and as such are not allowed for use outside of this.
\\\\
Parentheses and curly brackets are also reserved as apparent on the various listings cited in this section, mainly for use of encapsulating the different statements.

\subsection{Variables}
Variables are declared by specifying the variable's type prior to the name of it. 
PHAL allows variable names to consist of all letters from the roman alphabet.
\\
Variables are case-sensitive, but PHAL allows the user to use a variable without explicitly declaring it beforehand.
\\\\
For the naming of variables, we wish to have the variables start with a letter or an underscore, and the rest of the characters of the string can be either numbers, letters or underscores. 
The name of the variables cannot contain any other symbol except for numbers, letters and underscore.

\section{Formal definition} \label{FormalDefinition}
In this section we will describe the PHAL language with a formal definition. 
The informal definition generates a basic understanding but can be vague. 
The formal definition makes for a more precise representation of the language. 
We will for example present PHAL's CFG, regular expressions for PHAL and also the type rules.

\subsection{Lexemes and tokens}
In order to define the PHAL language we make use of lexemes and tokens. 
An input can be divided into these two categories, where lexemes are \textit{"words"} in the input, for example keywords, operators or identifiers. 
The token is a data structure for the lexemes and additional information. As a simple example for PHAL, the input \textit{a := 2 + 9.8} would be divided into lexemes and tokens in the following way:
\\
\textbf{Tokens:} id, assign, number, plus, number
\\
\textbf{Lexemes:} a, :=, 2, +, 9.8
\subsection{Token definition} 
All tokens with the exception of the keywords will be specified through regular expressions. 
Scanners are generally automatically generated through external software, where regular expressions are necessary because they are converted through algorithms.
The tokens for PHAL are described as follows:
\begin{itemize}
    \item Keyword
    \item Identifier
    \item Literals
    \item Operation
\end{itemize}
\subsubsection{Keywords}
A keyword can be one of the following:
\begin{center}
\begin{tabular}{c   c    c}
    and & bool & by \\
    call & case & decrease \\
    default & define & else \\
    for & group & if \\
    increase & is & list \\
    loop & none & not \\
    number & or & pin \\
    repeat & return & returnType \\
    setup & with & switch  \\
    text & then & times  \\
    until & &
\end{tabular}
\end{center}

\subsubsection{Identifiers}
Identifiers will be used for naming variables and classes. The identifiers must be formatted in such a way that they start with either an underscore or a letter, which can be followed by zero or more letters, digits or underscores.\\
As a regular expression, this would be expressed as $[\_a-zA-Z][\_a-zA-Z0-9]*$.
In addition to the format, they must be different from the reserved keywords mentioned above.


\subsubsection{Literals}
Literals are a notation for representing values in code. PHAL uses the following types of literals:
\begin{itemize}
    \item boolean-literal
    \item number-literal
    \item text-literal
\end{itemize}
\textbf{Boolean-literals}
\\
There are two boolean literal values: \textit{true} and \textit{false}. The type of a boolean-literal is bool.
These two values have a set of aliases \textit{on} and \textit{off}. These have the same functionality as the default boolean values, and are solely included for readability in context.
\\\\
\textbf{Number-literal}
\\
As per the definition of the \textit{number} type in PHAL in Section \ref{Def:Types}, the number-literal is a combination of integer-literals and real-literals.
\\\\
An integer literal is a string in the source code that is recognised as an integer by the lexical analyser. 
Integers can be any string made of a combination of $[0-9]+$. An integer literal can also have prefixes to represent a negative value.
\\\\
The real-literal is a string in the source code that is recognised as a float. Floats are of the form $[0-9]+.[0-9]+$ meaning that has the form of two numbers divided by a period. 
\\\\
\textbf{Text-literal}
\\
Text-literals are the strings in the source code encapsulated by quotation marks. A text-literal can be any combination of characters with the exception of a few characters. Something that should be addressed in the compiler is the problem that arises if a string contains other quotation marks, this is solved by simply disallowing the use of the quotation mark character in a text-literal. 


\section{Structural template}
The main structure of a PHAL program has two functions: A setup and a repeat function.
\\\\
The setup function is called when a program starts. 
It is for example used to initialise variables, set up modules and which pins they use or include libraries. It will only run once.
\\\\
The repeat function does as the name implies; repeats while the program is running. 
It is used to actively interact with connected components.
\\\\
Both of the primary functions are required, and do not return anything.
